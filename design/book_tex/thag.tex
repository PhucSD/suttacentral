\documentclass[10pt, openany]{book}

%PACKAGES%
\usepackage[inner=1.3in, width=4in, top=1in, height=7in, papersize={6in, 9in}]{geometry}
\usepackage{graphicx}
\usepackage{fontspec}
%\usepackage{letterspace}
\usepackage{enumitem}
\usepackage{sectsty}
\usepackage[compact]{titlesec}
\titlespacing{\subsubsection}{0pt}{*}{*-1}
\usepackage{soulutf8}
\usepackage{verse}
\usepackage{fix-cm}%font size\\
\usepackage{lettrine}
\usepackage[hyphens]{url}
\usepackage[tocflat, toctextentriesindented]{tocstyle}
\usepackage{tocloft}
\usepackage[unicode, hidelinks, pdfauthor={Bhikkhu Sujato}, pdftitle={Verses of the Senior Monks}, pdfsubject={Buddhism}, pdfkeywords={Buddhism, bhikkhu, monks, sutta, tipitaka, tripitaka, sutra}, pdfproducer={LuaTeX  beta-0.70.1}, pdfcreator={LaTeX2e}]{hyperref} %ADDS METADATA%
%\usepackage{pagegrid}
%PACKAGES%
%\pagegridsetup{top-left, step=3.435in}

%ADD BLANK PAGES AT END OF BOOK%
\newcommand{\blankpage}{\\
\newpage\\
\thispagestyle{empty}
\mbox{}
\newpage\\
}
%ADD BLANK PAGES AT END OF BOOK%
%TABLE OF CONTENTS%
%NOTE: STYLING OF TOC IS DONE WITH TOCSTYLE. CONTROL MARGINS OF TOC BY PLACING \NEWGEOMETRY AND\,\\textsc{re}STO\textsc{re}GEOMETRY AROUND\,TOC
\settocstylefeature[]{leaders}{\hfill}%ELIMINATES DOTS%
\settocstylefeature[0]{entryvskip}{0.4em}%VERTICAL SPACE BEFO\textsc{re}\,CHAPTER ENTRIES%

%TABLE OF CONTENTS%

%LINESPACE%
\usepackage{setspace}
\setstretch{1.12}

%LINESPACE%


\newfontfamily\Chapfont[]{Skolar PE}
\chapterfont{\Chapfont\scshape\centering\Large\mdseries}
\newfontfamily\Secfont[Numbers=OldStyle]{Skolar PE Semibold}
\sectionfont{\Secfont\large}
\newfontfamily\Subsubsecfont[Numbers=OldStyle]{Skolar PE Semibold}
\subsubsectionfont{\Subsubsecfont}



%HEADINGS%
\renewcommand*{\cftchapfont}{\mdseries}%use tocloft
\renewcommand*{\cftchappagefont}{\mdseries}%use tocloft
\renewcommand*{\cfttoctitlefont}{\Chapfont\Large}%use tocloft
%FONTS%
\setmainfont[Numbers=OldStyle]{Skolar PE}
\setsansfont[Scale = MatchLowercase,BoldFont={Source Sans Pro Semibold}]{Source Sans Pro Regular}
\setmonofont{Skolar PE}

\usepackage{realscripts}

%HEADER%
\usepackage{fancyhdr}
\setlength{\headheight}{15pt}
\pagestyle{fancy}
\renewcommand{\chaptermark}[1]{\markboth{\thechapter.\ #1}{}}
\renewcommand{\sectionmark}[1]{\markright{#1}{}}
\fancyhf{}
\fancyhead[LE,RO]{\thepage}
\fancyhead[CE]{\headcaps{\MakeUppercase{Verses of Senior Monks}}}
\fancyhead[CO]{\headcaps{\MakeUppercase{\leftmark}}}
\renewcommand{\headrulewidth}{0pt}
\fancypagestyle{plain}{ %
\fancyhf{} % remove everything
\renewcommand{\headrulewidth}{0pt}
\renewcommand{\footrulewidth}{0pt}}
\newfontfamily\headcapsfont[RawFeature=+c2sc]{Skolar PE}%THIS DEFINES THE SMALL CAPS FONT WITH SMALLCAPS NUMBER. IT TURNS CAPS INTO SMALL CAPS WHILE LEAVING LOWERCASE ALONE
\newcommand\headcaps[1]{{\headcapsfont #1}}

%HEADER%

\newfontfamily\SkolarLight{Skolar PE Light}





%HANGING LEFT%
\newcommand*{\vleftofline}[1]{\leavevmode\llap{#1}}
%HANGINGLEFT%

%WIDOWS & ORPHANS%
\widowpenalty=10000
\clubpenalty=10000
%WIDOWS & ORPHANS%

\usepackage{microtype}
\frenchspacing

%ENUMERATE PARAGRAPHS%
\setlist[enumerate,1]{label=\protect\raisebox{0.06em}{\arabic*}, leftmargin=0em, font=\SkolarLight\textsubscript, labelsep=1em, itemindent=0em}
%ENUMERATE PARAGRAPHS%
\usepackage{realscripts}


%ASTERISM%

%DOCUMENT INFO. NOT USED IN TEXT.%
\title{Verses of the Senior Monks}
\author{Bhikkhu Sujato \& Jessica Walton}
\date{}
%DOCUMENT INFO. NOT USED IN TEXT.%


\begin{document}

\frontmatter

\pagestyle{empty}

\maketitle

\newpage

\setlength{\parindent}{0cm}
\begin{small}

Published by SuttaCentral.

\medskip

First edition: 2014\\

\textbf{Creative Commons Zero (CC0 1.0 Universal)}

This work is public domain. To the extent possible under law, the authors Bhikkhu Sujato and Jessica Walton have waived all copyright and related or neighboring rights to \emph{Verses of the Senior Monks}.

\bigskip

\begin{center}{\sffamily suttacentral.net}\end{center}


ISBN: 978-1-921842-17-7

\end{small}

\newpage



\begin{center}\end{center}

\begin{center}

{\LARGE Verses of the Senior Monks}

\vfill

\emph{\large A new translation of the Theragāthā by}

\medskip

\MakeLowercase{\caps{\Large Bhikkhu Sujato \&}}

\medskip

\MakeLowercase{\caps{\Large Jessica Walton}}


\vfill
\emph{Published by}

\medskip

\caps{\large SuttaCentral}


\end{center}

\setlength{\parindent}{1.5em}

\tableofcontents 

\mainmatter

\pagestyle{fancy}

\chapter*{An Approachable Translation}
\addcontentsline{toc}{chapter}{An Approachable Translation}
\markboth{An Approachable Translation}{An Approachable Translation}

\setlength{\parskip}{0cm}

The Theragāthā is a classic Pali collection of verses by early Buddhist monks. The work consists of 1289 verses, collected according to the monk with whom they were traditionally associated. These poems speak from the personal experience of monks living in or near the time of the Buddha. More than any other text we find here a range of voices expressing the fears, inspirations, struggles, and triumphs of the spiritual search.

This new translation for SuttaCentral is released under Creative Commons Zero, which effectively dedicates the translation to the public domain. You are encouraged to do whatever you want with the text. Take it, change it, adapt it, print it, republish it in whatever way you wish. If you find any mistakes, or have any suggestions for the translation, I’d appreciate it if you were to let me know.

It is customary when making a new translation to acknowledge one’s debt to former translators, and to explain the need for a new one—and this case is no different. The Theragāthā has been fully translated into English twice before, both times published by the Pali Text Society. The first translation was by Caroline A.F. Rhys Davids in 1913, and the second by K.R. Norman in 1969. The efforts of the former translators is utterly indispensable, and their work makes each succeeding attempt that much easier. Nevertheless, the limitations of these earlier translations are well known. The Rhys Davids translation employs highly archaic language and poetic styles, as well as being based on a dated sensibility regarding both Pali and Buddhism. Norman’s translation, while exemplary in terms of Indological linguistics, employs what Norman himself described as “a starkness and austerity of words which borders on the ungrammatical”.

Moreover, neither of the former translations is freely available. To my knowledge, this is the first translation of the Theragāthā to be fully available on the internet.

Both of the earlier translations were based on the Pali Text Society’s edition by Hermann Oldenberg and Richard Pischel of 1883. The current translation, by contrast, is based on the Mahāsaṅgīti edition of the Pali canon, as published on SuttaCentral. It numbers 1289 verses as opposed to the 1279 of the PTS editions. The extra verses arise, not from a difference in substance, but from the inclusion of repetitions that were absent from the PTS editions. The first set of extra verses is at verse 1020 and the second at verse 1161. Up to verse 1020, therefore, the numbering is the same in the SuttaCentral and PTS editions.

\section*{What is an approachable translation?}

My aim was to make a translation that is first and foremost readable, so that this astonishing work of ancient spiritual insight might enjoy the wider audience it so richly deserves.

I’ve been thinking about the standard trope that introduces the prose suttas: a person “approaches” the Buddha to ask a question or hear a teaching. It’s so standard that we usually just pass it by. But it is no small thing to “approach” a spiritual teacher. It takes time, effort, curiosity, and courage; many of those people would have been more than a little nervous.

How, then, would the Buddha respond when approached? Would he have been archaic and obscure? Would he use words in odd, alienating ways? Would you need to have an expert by your side, whispering notes into your ear every second sentence—“He said this; but what he really meant was…”?

I think not. I think that the Buddha would have spoken clearly, kindly, and with no more complication than was necessary. I think that he would have respected the effort that people made to “approach” his teachings, and he would have tried the best he could, given the limitations of language and comprehension, to explain the Dhamma so that people could understand it.

Of course, the Theragāthā is not, with a few small exceptions, attributed to the Buddha; but the basic idea is the same. Most of the verses in the Thera\-gāthā are, like the bulk of the early texts, straightforward and didactic. Though formally cast as verse, their concern is not primarily with poetic style, but with meaning. They employed their literary forms solely in order to create an understanding in the listener, an understanding that leads to the letting go of suffering.

An approachable translation expresses the meaning of the text in simple, friendly, idiomatic English. It should not just be technically correct, it should sound like something someone might actually say.

Which means that it should strive to dispense entirely with the abomination of Buddhist Hybrid English, that obscure dialect of formalisms, technicalities, and Indic idioms that has dominated Buddhist translations, into which English has been coerced by translators who were writing for Indologists, linguists, and Buddhist philosophers. Buddhist Hybrid English is a Death by a Thousand Papercuts; with each obscurity the reader is distanced, taken out of the text, pushed into a mode of acting on the text, rather than being drawn into it.

That is not how those who listened to the Buddha would have experienced it. They were not being annoyed by the grit of dubious diction, nor were they being constantly nagged to check the footnotes. They were drawn inwards and upwards, fully experiencing the transformative power of the Dhamma as it came to life in the words of the Awakened. We cannot hope to recapture this experience, but we can try to make things no worse than they need to be.

At each step of the way I asked myself, “Would an ordinary person, with little or no understanding of Buddhism, be able to read this and understand what it is actually saying?” To this end, I have favored the simpler word over the more complex; the direct phrasing rather than the oblique; the active voice rather than the passive; the informal rather than the formal; and the explicit rather than the implicit. With this, my first substantive attempt at translating Pali, I feel I am a long way from achieving my goal; but perhaps a few small steps have been made.

\section*{This translation}

The process of creating the translation was this. In assembling the texts for SuttaCentral, I have been keen to create a complete online set of translations for early Buddhist texts. I find it astonishing that the early Buddhist texts are not all freely available on the internet, and I would like to change that. In 2013 I was approached by Jessica Walton (then Ayyā Nibbidā), a student of mine, who wanted a project to help learn Pali. I suggested that she work on the Thera/Theri-gāthā, in the hope that we could create a freely available translation.

Of course, this is a terrible job for a student—these are some of the most difficult texts in the Pali canon. But I hoped that it would prove useful, and so it has. I suggested that Jessica use Norman’s translation side by side with the Pali and work on creating a more readable rendering. She did this, mostly working on her own.

When she was happy with that, she passed the project over to me, and when I got the chance I took it up. I then went over the text in detail, modifying virtually every one of Jessica’s lines, while still keeping many of her turns of phrase. Without her work, this translation would not have been completed.

I also referred heavily to Norman’s translation, which enabled me to make sense of the many obscurities of vocabulary and syntax found in the text. Only rarely have I departed from Norman’s linguistic interpretations, and I have adopted his renderings on occasions when I felt I couldn’t do better.

There are, however, many occasions when Norman’s work is limited by his purely linguistic approach. There is no better example of this than Thag 411. The Pali begins \emph{uṭṭhehi nisīda}, on which Norman notes:

\begin{quote}The collocation of “stand up” and “sit down” is strange and clearly one or other of the words is used metaphorically.\end{quote}

He then renders the verse thus:

\begin{quote}Stand up, Kātiyāna, pay attention; do not be full of sleep, be awake. May the kinsman of the indolent, king death, not conquer lazy you, as though with a snare.\end{quote}

But to any meditator there is nothing strange about this at all; it just means to get up and meditate. I render the verse:

\begin{quote}Get up, Kātiyāna, and sit!\\
Don’t sleep too much, be wakeful.\\
Don’t be lazy, and let the kinsman of the heedless,\\
The king of death, catch you in his trap.\end{quote}

In addition to Norman’s translation, I have consulted translations by Bhikkhu Thanissaro and Bhik\-khu Bodhi for a few verses. I have, however, not consulted the Rhys Davids translation at all.

I should also acknowledge as influences in this translation my fellow monks, who I was living with while making this, especially Ajahn Brahm and Ajahn Brahmali. Both of these monks have influenced the translation greatly. It is from Ajahn Brahm that I have learned the virtue of plain English; of the kindness of speaking such that people actually understand. For years he has advocated the idea that translations should be based on the meaning of sentences, rather than the literal rendering of words. 

And with Ajahn Brahmali, who has been working on Vinaya translations at the same time, I have had many illuminating discussions about the meaning of various words and phrases. He said one thing that stuck in my mind: a translation should mean something. Even if you’re not sure what the text means, we can be sure that it had some meaning, so to translate it based purely on lexically correspondences is to not really translate it at all. Say what you think the text means, and if you’re wrong, fine, fix it up later.

\section*{About the Theragāthā}

I’d like to give a very brief and non-technical introduction to the text. If you are interested in a more detailed technical analysis, you can read Norman’s long introduction, which specially focusses on the metrical styles of the text.

Each of the verses of the Thera\-gāthā is collected under the name of a certain monk. (There is a parallel collection of nuns’ verses, the Therīgāthā, which I hope to translate in the future.) In many cases the verses were composed by, or at least were supposed to be composed by, these monks. Generally speaking I see no reason why the bulk of the verses should not be authentic. 

However, not all the verses can be ascribed to the monks in question. Sometimes the verses are in a dialogue form; or they may be teaching verses addressed to a monk; or they may be verses about a monk; in some cases they have been added by later redactors. In many cases, the verses are in a vague third person, which leaves it ambiguous whether it was meant to be by the monk or about him. And sometimes verses are repeated, both within the Thera\-gāthā and in other Buddhist texts, so a speaker of a verse is not always its composer. It is best, then, to consider the collection as “Verses associated with the senior monks”.

I have used the term “senior monk” rather than “elder” to render \emph{thera} for a couple of reasons. First, it will make it easier to distinguish the collection from the Therīgāthā. More importantly, not all the monks here are really “elders” in the sense of being wizened old men. Usually in Sangha usage a \emph{thera} is simply one who has completed ten years as a monk, so a monk of thirty years of age, while hardly an “elder”, may be a \emph{thera}.

As well as being collected according to the name of the associated monk, the texts are organized by number (the \emph{aṅguttara} principle). That is, the first sets of verses are those where a monk is associated with only one verse; then two, three, and so on. There is, in addition, an occasional connection of subject matter or literary style from one verse to the other; and, rarely, a thin narrative context (eg. Thag 16.1).

The numbering of the collections needs a little attention. The texts may be referenced by three means, all of which are available on SuttaCentral; either by simple verse count, or by chapter and verse, or by the page number of the PTS Pali edition.

The primary system used in SuttaCentral is the chapter and verse, as this collects all the verses associated with a given monk in one place. This chapter and verse system is not used in the PTS editions, but it is used in the Mahāsaṅgīti text on which the translation is based. However this system can be a little confusing—or at least, I was confused by it! From the ones to the fourteens there is no problem. There is no set of fifteen verses, so we skip from the fourteens to the sixteens. Here the numbering of the sections goes out of alignment with the number of verses: the fifteenth section (Thag 15.1) consists of a set of sixteen verses. The sixteenth section (Thag 16.1 etc.) then consists of sets of twenty or more verses, and so on.

In terms of dating, the Thera\-gāthā belongs firmly to the corpus of early Buddhist texts. Most of the monks are said to have lived in the time of the Buddha, and there seems no good reason to doubt this. In a few cases, due to the content of the text, the vocabulary or metre, or the statements in the commentary, the verses appear to date from as late as the time of king Ashoka. Norman suggests a period of composition of almost 300 years; however, if we adopt, as it seems we should, the “median chronology” that places the death of the Buddha not long before 400 \caps{bce}, then the period of composition would be closer to 200 years.

As with all Pali texts, the Thera\-gāthā is passed down in the tradition alongside a commentary, in this case written by Dhammapāla approximately 1,000 years after the text itself. As well as providing the normal kinds of linguistic and doctrinal analysis, the Thera\-gāthā commentary gives background stories for the lives of the monks, many of whom we know little about apart from the Thera\-gāthā itself. In some cases, the stories provide context to make sense of the verses, and there seems little doubt that these verses, as is the normal way in Pali, were passed down from the earliest times with some form of narrative context and explanation. Like the Jātakas, the Dhammapada, or the Udāna, the verses formed the emotional and doctrinal kernel of the story. However, in the form that we have it today, the commentary clearly speaks to a set of concerns and ideas that date long after the Thera\-gāthā itself. While the commentary is invaluable in understanding what the meaning of these texts was for the Theravadin tradition, it is probably in only rare cases that it provides genuine historical information about the monks. I have consulted the commentary only in cases where the meaning of the verse was unclear to me.

What is striking to me is just how clear-cut the demarcation of Pali texts really is. The Thera\-gāthā sits firmly on the far side of a dividing line in Pali literature that stems from the time of Ashoka or thereabouts. It is concerned with seclusion, meditation, mindfulness, and above all, liberation. Later texts were concerned with glorifying the Buddha, and especially with encouraging acts of merit for attaining heaven or enlightenment in future lives. Such concerns are notable for their absence from the Thera\-gāthā; when they are present, such as Sela’s verses extolling the Buddha, they remain grounded in human experience, rather than the elaborate fantasies of later days. 

There are a very few exceptions to this, such as Thag 1.96 Khaṇ\-ḍasumana, which says how after offering a flower he rejoiced in heaven for 800 million years, and then attained \emph{nibbāna} with the leftovers. But this just feels so out of place. Among the countless verses that speak of retreating to solitude, of devotion to jhana, of renouncing everything in the world, such sentiments seem as if from a different world of thought; a different religion even.

The classical Thera\-gāthā verse, as I mention above, is a song of liberation, rejoicing in a simple life lived with nature. Here’s a typical example, from Thag 1.22, the verse of Cittaka:

\begin{quote}Crested peacocks with beautiful blue necks\\
Cry out in Karaṃvī.\\
Aroused by a cool breeze,\\
They awaken the sleeper to practice \emph{jhāna}.\end{quote}

But the verses embrace a wide range of subjects; straightforward doctrinal statements, lamentations of the decline of the Sangha, eulogy of great monks, or simple narrative.

While the texts are mostly direct and clear heart\-ed, some of the most interesting verses are those that speak from the mind’s contradictions, the longings that accompany a full-blooded commitment to the spiritual life. Nowhere has this very human ambiguity been expressed better than in the extended set of verses by Tālapuṭa (Thag 19.1). Employing an unusually sophisticated poetic style—only exceeded in this regard by Vaṅgīsa, in whose verses we can discern the beginnings of the decadent poetics of later generations—and addressing his recalcitrant mind in an unusual second person, he berates it for its inconstancy:

\begin{quote}Oh, when will the winter clouds rain freshly\\
As I wear my robe in the forest,\\
Walking the path trodden by the sages?\\
When will it be? …

For many years you begged me,\\
\vleftofline{“}Enough of living in a house for you!”\\
Why do you not urge me on, mind,\\
Now I’ve gone forth as an ascetic?\end{quote}

Of all the texts in the Pali canon, it is in the verses of these senior monks and nuns that we come closest to the personal experience of living in the time of the Buddha, struggling with, and eventually overcoming, the causes of suffering that are so captivating. I hope that this new translation can help bring these experiences to life for a new audience.

\chapter*{Chapter One}
\addcontentsline{toc}{chapter}{Chapter One}
\markboth{Chapter One}{Chapter One}

\subsubsection*{1.1 Subhūti}

\vspace{0.78em}

\begin{enumerate}

\item My little hut is roofed and pleasant, \\
Sheltered from the wind:\\
So rain, sky, as you please!\\
My mind is serene and freed,\\
I practice whole-heartedly: so rain, sky!

\subsubsection*{1.2 Mahākoṭṭhika}

\item Calm and quiet,\\
Wise in counsel and steady;\\
Shaking off bad qualities,\\
As the wind shakes leaves off a tree.

\subsubsection*{1.3 Kaṅkhārevata}

\item See this wisdom of the Tathāgatas!\\
Like a fire blazing in the night,\\
Giving light, giving vision,\\
Dispelling the doubt of those who’ve come here.

\subsubsection*{1.4 Puṇṇa}

\item You should only associate with the wise, \\
Those intent upon good, seeing the goal.\\
Being wise, heedful, and discerning,\\
They realise the goal, so great, profound,\\
Hard to see, subtle, and fine.

\subsubsection*{1.5 Dabba}

\item Once hard to tame, now tamed themselves,\\
Worthy, content, crossed over doubt.\\
Victorious, with fears vanished,\\
Dabba is steadfast, and has realized \emph{nibbāna}.

\subsubsection*{1.6 Sītavaniya}

\item The monk who went to Sītavana is alone,\\
Content, practicing \emph{samādhi},\\
Victorious, with goosebumps vanished,\\
Guarding mindfulness of the body, resolute.

\subsubsection*{1.7 Bhalliya}

\item He has swept away the army of the king of death,\\
Like a great flood sweeping away\\
A fragile bridge of reeds.\\
Victorious, with fears vanished,\\
He is tamed and steadfast, and has realized \emph{nibbāna}.

\subsubsection*{1.8 Vīra}

\item Once hard to tame, now tamed himself,\\
A hero, content, with doubt overcome,\\
Victorious, with goosebumps vanished,\\
Vīra is steadfast, and has realized \emph{nibbāna}.

\subsubsection*{1.9 Pilindavaccha}

\item It was welcome, not unwelcome,\\
The advice I got was good.\\
Of things which are shared,\\
I encountered the best.

\subsubsection*{1.10 Puṇṇamāsa}

\item One who is accomplished in knowledge,\\
Peaceful and restrained,\\
Doesn’t expect to dwell in this world or the next.\\
Without clinging to anything,\\
They know the arising and passing of the world.

\subsubsection*{1.11 Cūḷavaccha}

\item A monk with much joy \\
In the Dhamma taught by the Buddha\\
Would realise the peaceful state:\\
The stilling of activities, bliss.

\subsubsection*{1.12 Mahāvaccha}

\item Empowered by wisdom, \\
Endowed with virtue and vows,\\
Possessing \emph{samādhi}, delighting in \emph{jhāna}, mindful,\\
Eating suitable food,\\
One should bide one’s time here, free of desire.

\subsubsection*{1.13 Vanavaccha}

\item They look like blue-black storm clouds, glistening, \\
Cooled with the waters of clear-flowing streams,\\
And covered with ladybird beetles:\\
These rocky crags delight me!

\subsubsection*{1.14 Novice Sivaka}

\item My preceptor said:\\
\vleftofline{“}Let’s go from here, Sīvaka.”\\
My body lives in the village,\\
But my mind has gone to the wilderness.\\
I’ll go there even if I’m lying down;\\
There’s no tying down one who understands.

\subsubsection*{1.15 Kuṇḍadhāna}

\item Five should be cut off, five should be abandoned,\\
Five more should be developed.\\
A monk who has overcome five attachments\\
Is called “One who has crossed the flood”.

\subsubsection*{1.16 Belaṭṭhasīsa}

\item Just as a fine thoroughbred\\
Proceeds with ease,\\
Tail and mane flying in the wind;\\
So my days and nights \\
Proceed with ease,\\
Full of spiritual joy.

\subsubsection*{1.17 Dāsaka}

\item One who is drowsy, a glutton,\\
Fond of sleep, rolling as they lie,\\
Like a great hog stuffed with food:\\
That fool is reborn again and again.

\subsubsection*{1.18 Siṅgālapitu}

\item There was an heir of the Buddha,\\
A monk in Bhesakaḷā forest,\\
Who suffused the entire earth \\
With the perception of “bones”.\\
I think he will quickly abandon sensual desire.

\subsubsection*{1.19 Kula}

\item Irrigators lead water,\\
Fletchers shape arrows,\\
Carpenters shape wood;\\
The disciplined tame themselves.

\subsubsection*{1.20 Ajita}

\item I do not fear death;\\
Nor do I long for life.\\
I’ll lay down this body,\\
Aware and mindful.

\subsubsection*{1.21 Nigrodha}

\item I’m not afraid of fear.\\
Our teacher is skilled in the deathless;\\
Monks proceed by the path\\
Where no fear remains.

\subsubsection*{1.22 Cittaka}

\item Crested peacocks with beautiful blue necks\\
Cry out in Karaṃvī.\\
Aroused by a cool breeze,\\
They awaken the sleeper to practice \emph{jhāna}.

\subsubsection*{1.23 Gosāla}

\item I’ll eat honey and rice in Veḷugumba,\\
And then, skilfully scrutinizing \\
The rise and fall of the aggregates,\\
I’ll return to my forest hill,\\
And devote myself to seclusion.

\subsubsection*{1.24 Sugandha}

\item I went forth after the rainy season—\\
See the excellence of the Dhamma!\\
I’ve attained the three knowledges\\
And fulfilled the Buddha’s instructions.

\subsubsection*{1.25 Nandiya}

\item Dark One, if you attack such a monk, \\
Whose mind is full of light,\\
And has arrived at the fruit,\\
You’ll fall into suffering.

\subsubsection*{1.26 Abhaya}

\item Having heard the wonderful words\\
Of the Buddha, the Kinsman of the Sun,\\
I penetrated the subtle truth,\\
Like a hair-tip with an arrow.

\subsubsection*{1.27 Lomasakaṅgiya}

\item With my chest I’ll thrust aside \\
The grasses, vines, and creepers,\\
And devote myself to seclusion.

\subsubsection*{1.28 Jambugāmikaputta}

\item Aren’t you obsessed with clothes?\\
Don’t you delight in jewellery?\\
Is it you—not anyone else—\\
Spreading the scent of virtue?

\subsubsection*{1.29 Hārita}

\item Straighten yourself,\\
Like a fletcher straightens an arrow.\\
When your mind is upright, Hārita, \\
Demolish ignorance!

\subsubsection*{1.30 Uttiya}

\item When I was ill in the past,\\
I was mindful.\\
Now I am ill once more—\\
It’s time to be heedful.

\subsubsection*{1.31 Gahvaratīriya}

\item Bitten by ticks and mosquitoes\\
In the wilderness, the ancient forest;\\
One should endure mindfully,\\
Like an elephant at the head of the battle.

\subsubsection*{1.32 Suppiya}

\item I’ll exchange old age for the un-ageing,\\
Burning for extinguishing:\\
The ultimate peace,\\
The unexcelled safety from the yoke.

\subsubsection*{1.33 Sopāka}

\item Just as a mother would be good \\
To her beloved and only son;\\
So, to creatures all and everywhere,\\
Let one be good.

\subsubsection*{1.34 Posiya}

\item For one who understands\\
It’s always better not to mix with such women.\\
I went from the village to the wilderness;\\
From there I entered the house.\\
Though I was there to be fed, \\
I stood up and left without taking leave.

\subsubsection*{1.35 Sāmaññakāni}

\item Whoever is seeking happiness \\
Will find it through this practice,\\
Get a good reputation, and grow in renown:\\
Develop the noble eightfold, straight, direct path \\
For the realisation of the deathless.

\subsubsection*{1.36 Kumāputta}

\item Learning is good, wandering is good,\\
Homeless life is always good.\\
Questions on the goal,\\
Actions that are skilful,\\
This is the ascetic life for one who has nothing.

\subsubsection*{1.37 Kumāputtasahāyaka}

\item Some travel to different regions,\\
Wandering unrestrained.\\
If they lose their stillness,\\
What is the point\\
Of wandering around the countries?\\
So you should dispel pride,\\
practising \emph{jhāna} without distraction.

\subsubsection*{1.38 Gavampati}

\item His psychic power \\
Made the river Sarabhu stand still;\\
Gavampati is unbound and unperturbed.\\
The gods bow to that great sage,\\
Who has left behind all attachments, \\
And gone beyond rebirth in any state of existence.

\subsubsection*{1.39 Tissa}

\item As if struck by a sword,\\
As if their head was on fire,\\
A monk should go forth mindfully,\\
To abandon desire for sensual pleasures.

\subsubsection*{1.40 Vaḍḍhamāna}

\item As if struck by a sword,\\
As if their head was on fire,\\
A monk should go forth mindfully,\\
To abandon desire to be reborn \\
In any state of existence.

\subsubsection*{1.41 Sirivaḍḍha}

\item Lightning flashes down\\
On the cleft of Vebhāra and Paṇḍava.\\
But in the mountain cleft, the son of the inimitable\\
Is absorbed in \emph{jhāna}, equanimous.

\subsubsection*{1.42 Khadiravaniya}

\item Cāla, Upacāla and Sīsupacāla:\\
Be mindful!\\
I’ve come to you like a hair-splitter.

\subsubsection*{1.43 Sumaṅgala}

\item Well freed! Well freed!\\
I’m very well freed from three crooked things:\\
My sickles, my ploughs, my little hoes.\\
Even if they were here, right here—\\
I’d be done with them, done!\\
Practice \emph{jhāna} Sumaṅgala!\\
Practice \emph{jhāna}  Sumaṅgala!\\
Stay heedful, Sumaṅgala!

\subsubsection*{1.44 Sānu}

\item Mum, they cry for the dead,\\
Or for one who is alive but has disappeared.\\
I’m alive and you can see me,\\
So Mum, why do you weep for me?

\subsubsection*{1.45 Ramaṇīyavihāri}

\item Just as an excellent throroughbred\\
Having stumbled, stands firm,\\
So I’m endowed with vision,\\
A disciple of the Buddha.

\subsubsection*{1.46 Samiddhi}

\item I went forth out of faith\\
From the home life into homelessness.\\
My mindfulness and wisdom have grown,\\
My mind is serene in \emph{samādhi}.\\
Make whatever illusions you want,\\
It doesn’t bother me.

\subsubsection*{1.47 Ujjaya}

\item Homage to the Buddha, the hero,\\
Freed in every way.\\
Abiding in the fruits of your practice,\\
I live without defilements.

\subsubsection*{1.48 Sañjaya}

\item Since I’ve gone forth\\
From home life into homelessness,\\
I’m not aware of any intention\\
That is ignoble and hateful.

\subsubsection*{1.49 Rāmaṇeyyaka}

\item Even with all the sounds,\\
The sweet chirping and cheeping of birds,\\
My mind doesn’t tremble,\\
For I’m devoted to oneness.

\subsubsection*{1.50 Vimala}

\item The rain falls and the wind blows on mother Earth,\\
While lightning flashes across the sky!\\
But my thoughts are stilled,\\
My mind is serene in \emph{samādhi}.

\subsubsection*{1.51 Godhika}

\item The sky rains down, like a beautiful song,\\
My little hut is roofed and pleasant, \\
Sheltered from the wind.\\
My mind is serene in \emph{samādhi}:\\
So rain, sky, as you please.

\subsubsection*{1.52 Subāhu}

\item The sky rains down, like a beautiful song,\\
My little hut is roofed and pleasant, \\
Sheltered from the wind.\\
My mind is serene in my body:\\
So rain, sky, as you please.

\subsubsection*{1.53 Valliya}

\item The sky rains down, like a beautiful song,\\
My little hut is roofed and pleasant, \\
Sheltered from the wind.\\
I dwell there, heedful:\\
So rain, sky, as you please.

\subsubsection*{1.54 Uttiya}

\item The sky rains down, like a beautiful song,\\
My little hut is roofed and pleasant, \\
Sheltered from the wind.\\
I dwell there without a partner:\\
So rain, sky, as you please.

\subsubsection*{1.55 Añjanavaniya}

\item I plunged into the  Añjana forest\\
And made a little hut to live in.\\
I’ve attained the three knowledges\\
And fulfilled the Buddha’s instructions.

\subsubsection*{1.56 Kuṭivihāri}

\item \vleftofline{“}Who is in this little hut?”\\
\vleftofline{“}A monk is in this little hut,\\
Free of lust, his mind serene in \emph{samādhi}.\\
My friend, you should know this:\\
Your little hut wasn’t built in vain.”

\subsubsection*{1.57 Dutiyakuṭivihāri}

\item This was your old hut,\\
But you still want a new hut.\\
Dispel desire for a hut, monk!\\
A new hut will only bring more suffering.

\subsubsection*{1.58 Ramaṇīyakuṭika}

\item My little hut is pleasing, delightful,\\
A gift given in faith.\\
I’ve no need of girls:\\
Go, ladies, to those in need!

\subsubsection*{1.59 Kosalavihāri}

\item I went forth out of faith\\
And built a little hut in the wilderness.\\
I’m heedful, ardent, \\
Aware, and mindful.

\subsubsection*{1.60 Sīvali}

\item My intentions, the purpose \\
Of entering this hut, have prospered.\\
Abandoning the tendency to conceit,\\
I’ll realise knowledge and liberation.

\subsubsection*{1.61 Vappa}

\item One who sees\\
Sees those who see and those who don’t.\\
One who doesn’t see\\
Sees neither.

\subsubsection*{1.62 Vajjiputta}

\item We dwell alone in the wilderness,\\
Like a log rejected in a forest.\\
Lots of people are jealous of me,\\
Like beings in hell are jealous \\
Of someone going to heaven.

\subsubsection*{1.63 Pakkha}

\item They died and fell;\\
Fallen but still greedy, they return.\\
What had to be done has been done,\\
What had to be enjoyed has been enjoyed,\\
Happiness has been realised through happiness.

\subsubsection*{1.64 Vimalakoṇḍañña}

\item I arose from the one named after a tree,\\
I was born of the one whose banner shines.\\
The banner killer has destroyed the great banner,\\
By means of the banner itself.

\subsubsection*{1.65 Ukkhepakatavaccha}

\item Vaccha has tossed away\\
What he built over many years.\\
Sitting comfortably, uplifted with joy,\\
He teaches this to householders.

\subsubsection*{1.66 Meghiya}

\item He counselled me, the great hero,\\
The one who has gone beyond all things.\\
When I heard his teaching \\
I stayed close by him, mindful.\\
I’ve attained the three knowledges\\
And fulfilled the Buddha’s instructions.

\subsubsection*{1.67 Ekadhammasavanīya}

\item My defilements have been burnt away \\
By practising \emph{jhāna};\\
Rebirth into all states of existence is over,\\
Transmigraton through births is finished,\\
Now there is no more rebirth \\
Into any state of existence.

\subsubsection*{1.68 Ekudāniya}

\item A sage with higher consciousness, heedful,\\
Training in the ways of silence,\\
At peace and always mindful:\\
Such a one has no sorrow.

\subsubsection*{1.69 Channa}

\item Hearing the sweet Dhamma taught by the master,\\
Who understands all, and whose knowledge excels,\\
I’ve entered the path to realise the deathless.\\
He’s skilled in the road to safety from the yoke.

\subsubsection*{1.70 Puṇṇa}

\item Virtue is the highest here,\\
But understanding is supreme.\\
A person with both virtue and understanding\\
Is victorious among men and gods.

\subsubsection*{1.71 Vacchapāla}

\item Though \emph{nibbāna} is very refined and subtle,\\
It is not difficult to realize for one who sees the goal,\\
Skilled in thought, humble in manner,\\
Cultivating the virtuous conduct of the Buddha.

\subsubsection*{1.72 Ātuma}

\item A young bamboo is hard to trample\\
When the point is grown and it’s become woody;\\
That’s how I feel with the wife \\
Who was arranged for me.\\
Give me permission—now I’ve gone forth. 

\subsubsection*{1.73 Māṇava}

\item Seeing an old person,\\
One suffering from disease,\\
And a corpse, come to the end of life,\\
I went forth, becoming a wanderer,\\
And abandoning the pleasures of the senses.

\subsubsection*{1.74 Suyāmana}

\item Sensual desire, ill will,\\
Dullness and drowsiness,\\
Restlessness, and doubt\\
Are not found in a monk at all.

\subsubsection*{1.75 Susārada}

\item Good is the sight of those who’ve practised well;\\
Doubt is cut off, and intelligence grows.\\
Even a fool becomes wise;\\
Therefore meeting with such people is good.

\subsubsection*{1.76 Piyañjaha}

\item Settle down when others spring up;\\
Spring up when others settle down;\\
Remain when others have departed;\\
Be without delight when others delight.

\subsubsection*{1.77 Hatthārohaputta}

\item In the past my mind wandered\\
How it wished, where it liked, as it pleased.\\
Now I’ll carefully guide it,\\
As a rutting elephant is guided \\
By a trainer with a hook.

\subsubsection*{1.78 Meṇḍasira}

\item Transmigrating through countless births,\\
I’ve journeyed without end.\\
I’ve suffered, but now:\\
The mass of suffering has collapsed.

\subsubsection*{1.79 Rakkhita}

\item All my lust is abandoned,\\
All my hate is undone,\\
All my delusion is gone;\\
I’m cooled, quenched.

\subsubsection*{1.80 Ugga}

\item Whatever actions I have performed,\\
Whether trivial or important,\\
Are all completely exhausted;\\
Now there is no more rebirth \\
Into any state of existence.

\subsubsection*{1.81 Samitigutta}

\item Whatever evil I have performed\\
In previous births,\\
It is to be experienced right here,\\
And not in any other place.

\subsubsection*{1.82 Kassapa}

\item Go, child,\\
Where there’s plenty of food,\\
Safe and fearless—\\
May you not be overcome by sorrow!

\subsubsection*{1.83 Sīha}

\item Dwell heedful, Sīha,\\
Don’t be lazy by day or by night.\\
Develop skilful qualities,\\
And quickly discard this mortal frame.

\subsubsection*{1.84 Nīta}

\item Sleeping all night,\\
Fond of socializing by day,\\
When will the fool\\
Make an end of suffering?

\subsubsection*{1.85 Sunāga}

\item Skilled in the characteristics of the mind,\\
Understanding the sweetness of seclusion,\\
Practising \emph{jhāna}, disciplined, mindful:\\
Such a person would realize spiritual happiness.

\subsubsection*{1.86 Nāgita}

\item Outside of here there are many other doctrines;\\
Those paths don’t lead to \emph{nibbāna}, but this one does.\\
Indeed, the Blessed One himself counsels the Saṅgha;\\
The Teacher shows the palms of his hands.

\subsubsection*{1.87 Paviṭṭha}

\item The aggregates are seen in accordance with reality,\\
Rebirth in all states of existence is torn apart,\\
Transmigration through births is finished,\\
Now there is no more rebirth \\
Into any state of existence.

\subsubsection*{1.88 Ajjuna}

\item I was able to lift myself up\\
From the water to the shore.\\
I’ve penetrated the truths,\\
Like one swept along on a powerful flood.

\subsubsection*{1.89 Devasabha}

\item I’ve crossed the marshes,\\
I’ve avoided the cliffs,\\
I’m freed from floods and fetters,\\
And I've destroyed all conceit.

\subsubsection*{1.90 Sāmidatta}

\item The five aggregates are fully understood;\\
They remain with the root cut off.\\
Transmigration is finished,\\
Now there is no more rebirth \\
Into any state of existence.

\subsubsection*{1.91 Paripuṇṇaka}

\item What I consumed today is considered better\\
Than pure food of a hundred flavors:\\
The Dhamma taught by the Buddha,\\
Gotama of infinite vision.

\subsubsection*{1.92 Vijaya}

\item The one whose defilements are dried up,\\
Who’s not attached to food,\\
Whose resort is the liberation\\
That is signless and empty:\\
Their track is hard to trace,\\
Like that of birds in the sky.

\subsubsection*{1.93 Eraka}

\item Sensual pleasures are suffering, Eraka!\\
Sensual pleasures aren’t happiness, Eraka!\\
One who enjoys sensual pleasures \\
Enjoys suffering, Eraka!\\
One who doesn’t enjoy sensual pleasures \\
Doesn’t enjoy suffering, Eraka!

\subsubsection*{1.94 Mettaji}

\item Homage to that Blessed One,\\
The glorious son of the Sakyans!\\
When he realised the highest state,\\
He taught the highest Dhamma well.

\subsubsection*{1.95 Cakkhupāla}

\item I’m blind, my eyes are ruined,\\
I’m travelling a desolate road.\\
Even if I have to crawl I’ll keep going—\\
Though not with wicked companions.

\subsubsection*{1.96 Khaṇḍasumana}

\item I offered a single flower,\\
And then amused myself in heavens\\
For 800 million years;\\
With what’s left over I’ve realized \emph{nibbāna}.

\subsubsection*{1.97 Tissa}

\item Giving up a valuable bronze bowl,\\
And a precious golden one, too,\\
I took a bowl made of clay:\\
This is my second anointing.

\subsubsection*{1.98 Abhaya}

\item If you focus on the pleasant aspect\\
Of sights that you see, you’ll lose your mindfulness.\\
Experiencing it with a lustful mind,\\
You keep holding on.\\
Your defilements grow,\\
Leading to the root of rebirth \\
In some state of existence.

\subsubsection*{1.99 Uttiya}

\item If you focus on the pleasant aspect\\
Of sounds that you hear, \\
You’ll lose your mindfulness.\\
Experiencing it with a lustful mind,\\
You keep holding on.\\
Your defilements grow,\\
Leading to transmigration.

\subsubsection*{1.100 Devasabha}

\item Accomplished in the four right strivings,\\
With establishment of mindfulness as your safe place,\\
Festooned with the flowers of liberation,\\
You’ll realise \emph{nibbāna} without defilements.

\subsubsection*{1.101 Belaṭṭhānika}

\item He’s given up the household life, \\
But he has no purpose,\\
Like a big pig that chomps on grain,\\
Using his snout as a plough, living for his belly, lazy:\\
That idiot comes to the womb again and again.

\subsubsection*{1.102 Setuccha}

\item Deceived by conceit,\\
Defiled by conditions,\\
Distressed by gain and loss,\\
They don’t realise \emph{samādhi}.

\subsubsection*{1.103 Bandhura}

\item I don’t need this—\\
I’m satisfied and pleased with the sweet Dhamma.\\
I’ve drunk the best, the supreme nectar:\\
I won’t go near poison.

\subsubsection*{1.104 Khitaka}

\item Hey! My body is light,\\
Full of so much rapture and happiness.\\
My body feels like it’s floating,\\
Like cotton on the wind.

\subsubsection*{1.105 Malitavambha}

\item Dissatisfied, one should not stay;\\
Happy, one should depart.\\
One who sees clearly wouldn’t stay\\
In a place that was not conducive to the goal.

\subsubsection*{1.106 Suhemanta}

\item When the meaning has a hundred aspects,\\
And carries a hundred characteristics,\\
The fool sees only one factor,\\
While the sage sees a hundred.

\subsubsection*{1.107 Dhammasava}

\item After investigating, I went forth\\
From the home life into homelessness.\\
I’ve attained the three knowledges\\
And fulfilled the Buddha’s instructions.

\subsubsection*{1.108 Dhammasavapitu}

\item At 120 years old\\
I went forth into homelessness.\\
I’ve attained the three knowledges\\
And fulfilled the Buddha’s instructions.

\subsubsection*{1.109 Saṃgharakkhita}

\item He’s gone on retreat, \\
But he doesn’t yet heed the counsel\\
Of the one with supreme compassion \\
For his welfare.\\
He lives with unrestrained faculties,\\
Like a young deer in the woods.

\subsubsection*{1.110 Usabha}

\item The trees on the mountain-tops have grown well,\\
Freshly sprinkled by towering clouds.\\
For Usabha, who loves seclusion, \\
And who thinks only of wilderness,\\
Goodness arises more and more.

\subsubsection*{1.111 Jenta}

\item Going forth is hard, living at home is hard,\\
Dhamma is profound, \\
And money is hard to come by.\\
Getting by is difficult \\
For we who accept whatever comes,\\
So we should always remember impermanence.

\subsubsection*{1.112 Vacchagotta}

\item I have the three knowledges, I’m a great meditator,\\
Skilled in serenity of mind.\\
I’ve realized my own true goal,\\
And fulfilled the Buddha’s instructions.

\subsubsection*{1.113 Vanavaccha}

\item The water is clear and the gorges are wide,\\
Monkeys and deer are all around;\\
Festooned with dewy moss,\\
These rocky crags delight me!

\subsubsection*{1.114 Adhimutta}

\item When your body is uncomfortably heavy,\\
While life is running out;\\
Greedy for physical pleasure,\\
How can you find happiness as an ascetic?

\subsubsection*{1.115 Mahānāma}

\item By Mount Nesādaka,\\
With its famous covering\\
Of many shrubs and trees,\\
You’re found deficient.

\subsubsection*{1.116 Pārāpariya}

\item I’ve abandoned the six spheres of sense-contact,\\
My sense-doors are guarded and well restrained;\\
I’ve ejected the root of misery,\\
And attained the end of defilements.

\subsubsection*{1.117 Yasa}

\item I’m well-anointed and well-dressed,\\
Adorned with all my jewellery.\\
I’ve attained the three knowledges\\
And fulfilled the Buddha’s instructions.

\subsubsection*{1.118 Kimila}

\item Old age falls like a curse;\\
It’s the same body, but it seems like someone else’s.\\
I remember myself as if I was someone else,\\
But I’m still the same, I haven’t been away.

\subsubsection*{1.119 Vajjiputta}

\item You’ve gone to the jungle, the root of a tree,\\
Putting \emph{nibbāna} in your heart.\\
Practice \emph{jhāna}, Gotama, don’t be heedless.\\
What is this hullabaloo to you?

\subsubsection*{1.120 Isidatta}

\item The five aggregates are fully understood,\\
They remain, but their root is severed.\\
I have realized the end of suffering,\\
And attained the end of defilements.

\chapter*{Chapter Two}
\addcontentsline{toc}{chapter}{Chapter Two}
\markboth{Chapter Two}{Chapter Two}

\subsubsection*{2.1 Uttara}

\item No life is permanent,\\
And no conditions last forever.\\
The aggregates are reborn\\
And pass away, again and again.

\item Knowing this danger,\\
I’m not interested in being reborn \\
Into any state of existence.\\
I’ve escaped all sensual pleasures,\\
And attained the end of defilements.

\subsubsection*{2.2 Piṇḍolabhāradvāja}

\item You can’t live by fasting,\\
But food doesn’t lead to peace of heart.\\
Seeing how the body is sustained by food,\\
I wander, seeking.

\item They know it’s a swamp,\\
This worship and homage from respectable families;\\
A subtle dart, hard to pull out;\\
It’s hard for a corrupt person to give up honour.

\subsubsection*{2.3 Valliya}

\item A monkey went up to the little hut\\
With five doors.\\
He circles around, knocking\\
On each door, again and again.

\item Stand still monkey, don’t run!\\
Things are different now;\\
You’ve been caught by wisdom—\\
You won’t go far.

\subsubsection*{2.4 Gaṅgātīriya}

\item My hut on the bank of the Ganges\\
Is made from three palm leaves.\\
My alms-bowl is a funeral pot,\\
My robe is castoff rags.

\item In my first two rainy seasons\\
I spoke only one word.\\
In my third rainy season,\\
The mass of darkness was torn apart.

\subsubsection*{2.5 Ajina}

\item Even someone with the three knowledges,\\
Who has conquered death, \\
And is without defilements,\\
Is looked down on for being unknown\\
By fools without wisdom.

\item But a person who gets food and drink\\
Is honored by them,\\
Even if they are of bad character.

\subsubsection*{2.6 Meḷajina}

\item When I heard the Teacher\\
Speaking Dhamma,\\
I wasn’t aware of any doubt\\
In the all-knowing, unconquered one,

\item The caravan leader, the great hero,\\
The most excellent of charioteers.\\
I have no doubt\\
In the path or practice.

\subsubsection*{2.7 Rādha}

\item Just as rain seeps into\\
A poorly roofed house,\\
Lust seeps into\\
An undeveloped mind.

\item Just as rain doesn’t seep into\\
A well roofed house,\\
Lust doesn’t seep into\\
A well-developed mind.

\subsubsection*{2.8 Surādha}

\item Rebirth is ended for me,\\
The conqueror’s instruction is fulfilled,\\
What they call a “net” is abandoned,\\
The attachment to being reborn \\
In any state of existence is undone.

\item I’ve arrived at the goal\\
For the sake of which I went forth\\
From the home life into homelessness:\\
The ending of all fetters.

\subsubsection*{2.9 Gotama}

\item Sages sleep happily\\
When they’re not attached to women;\\
For the truth is hard to find among them,\\
And one must always be guarded.

\item Sensual pleasure, you’ve been slain!\\
We’re not in your debt any more.\\
Now we go to \emph{nibbāna},\\
Where there is no more sorrow.

\subsubsection*{2.10 Vasabha}

\item First one kills oneself,\\
Then one kills others.\\
One kills oneself, really dead,\\
Like one who kills birds using a dead bird as a decoy.

\item A holy man’s color is not on the outside;\\
A holy man is colored on the inside.\\
Whoever does bad deeds\\
Such a one is truly dark, Sujampati.

\subsubsection*{2.11 Mahācunda}

\item It is from wishing to learn that learning grows;\\
When you are learned, understanding grows;\\
Through understanding, you know the goal;\\
Knowing the goal brings happiness.

\item Make use of secluded lodgings!\\
Practice to be released from fetters!\\
If you don’t find enjoyment there,\\
Live in the Saṅgha, guarded and mindful.

\subsubsection*{2.12 Jotidāsa}

\item People who act harshly—\\
Attacking people,\\
Tying them up,\\
Hurting them in all kinds of ways—\\
They’re treated in the same way;\\
Their deeds don’t vanish.

\item Whatever deeds a person does,\\
Whether for good or for bad,\\
They are the heir to each\\
And every deed that they perform.

\subsubsection*{2.13 Heraññakāni}

\item The days and nights rush by,\\
And then life is cut short.\\
The vitality of mortals wastes away,\\
Like the water in tiny streams.

\item But while doing bad deeds\\
The fool doesn’t realize—\\
It’ll be bitter later on;\\
Yes, the result will be bad for him.

\subsubsection*{2.14 Somamitta}

\item If someone lost in the middle of the ocean,\\
Were to clamber up on a little log, they’d sink;\\
In the same way, even a good person would sink\\
If they rely on a lazy person.\\
So avoid those who are lazy, lacking energy.

\item Instead, dwell with the wise—\\
Secluded, noble,\\
Resolute, practising \emph{jhāna},\\
And always energetic.

\subsubsection*{2.15 Sabbamitta}

\item People are attached to people;\\
People are dependent on people;\\
People are hurt by people;\\
And people hurt people.

\item What’s the point of people,\\
Or the things people make?\\
Go, leave these people,\\
Who’ve hurt so many people.

\subsubsection*{2.16 Mahākāḷa}

\item There’s a big black woman who looks like a crow.\\
She broke off thigh-bones, first one then another;\\
She broke off arm-bones, first one then another;\\
She broke off a skull like a curd-bowl, and then—\\
She assembled them all together \\
And sat down beside them.

\item When an ignorant person builds up attachments,\\
That idiot returns to suffering, again and again.\\
So let one who understands not build up attachments:\\
May I never again lie with a broken skull!

\subsubsection*{2.17 Tissa}

\item When your head is shaven, \\
And you’re wrapped in the outer robe,\\
You’ll have many enemies\\
When you receive food and drink,\\
Clothes and lodgings.

\item Knowing this danger,\\
This great fear in honours,\\
A monk should go forth mindfully,\\
With few possessions, and not full of desire.

\subsubsection*{2.18 Kimila}

\item In Pācīnavaṃsa grove\\
The companions of the Sakyans,\\
Having given up much wealth,\\
Are satisfied with whatever is put in their bowls.

\item Energetic, resolute,\\
Always strong in striving;\\
Having given up mundane satisfaction,\\
They delight in the satisfaction of Dhamma.

\subsubsection*{2.19 Nanda}

\item I used my mind unwisely,\\
I was addicted to ornamentation.\\
I was vain, fickle,\\
Tormented by desire for sensual pleasures.

\item But with the help of the Buddha,\\
The Kinsman of the Sun, so skilled in means,\\
I practiced wisely,\\
And extracted any attachment \\
To being reborn from my mind.

\subsubsection*{2.20 Sirimā}

\item If they praise someone\\
Who doesn’t have \emph{samādhi},\\
The praise is in vain,\\
As they don’t have \emph{samādhi}.

\item If they rebuke someone\\
Who does have \emph{samādhi},\\
The rebuke is in vain,\\
As they do have \emph{samādhi}.

\subsubsection*{2.21 Uttara}

\item I’ve fully understood the aggregates,\\
I’ve undone craving;\\
I’ve developed the factors of awakening,\\
And I’ve realized the ending of defilements.

\item Having fully understood the aggregates,\\
Having plucked out the weaver of the web,\\
Having developed the factors of awakening,\\
I’ll realize \emph{nibbāna}, without defilements.

\subsubsection*{2.22 Bhaddaji}

\item That king was named Panāda,\\
Whose sacrificial post was golden.\\
Its height was sixteen times its width,\\
And the top was a thousand-fold.

\item With a thousand panels, and a hundred ball-caps,\\
Adorned with banners, made of gold;\\
There, the seven times six hundred\\
Gods of music danced.

\subsubsection*{2.23 Sobhita}

\item As a monk, mindful and wise,\\
Resolute in power and energy,\\
I recollected five hundred aeons\\
In one night.

\item Developing the four establishments of mindfulness,\\
The seven factors of awakening, \\
And the eightfold path,\\
I recollected five hundred aeons\\
In one night.

\subsubsection*{2.24 Valliya}

\item The duty of one whose energy is strong;\\
The duty of one intent on awakening:\\
That I’ll do, I won’t fail—\\
See my energy and effort!

\item Teach me the path,\\
The road that culminates in the deathless.\\
I’ll know it with wisdom,\\
As the Ganges knows the ocean.

\subsubsection*{2.25 Vītasoka}

\item The barber approached\\
To shave my head.\\
I picked up a mirror\\
And looked at my body.

\item My body looked vacant;\\
I was blind, but the darkness left me.\\
My fancy hairdo has been cut off:\\
Now there is no more rebirth\\
Into any state of existence.

\subsubsection*{2.26 Puṇṇamāsa}

\item I abandoned the five hindrances\\
So I could realise security from the yoke;\\
And I picked up the Dhamma as a mirror,\\
For knowing and seeing myself.

\item I checked over this body\\
All of it, inside and out.\\
Internally and externally,\\
My body looked vacant.

\subsubsection*{2.27 Nandaka}

\item Though a fine thoroughbred stumbles\\
It soon stands firm again;\\
It gains even more spirit,\\
And draws its load undeterred.

\item Even so, I am one endowed with vision,\\
A disciple of the Buddha!\\
You should remember me as a thoroughbred,\\
The Buddha’s rightful son.

\subsubsection*{2.28 Bharata}

\item Come Nandaka, let’s go\\
To visit our preceptor.\\
We’ll roar the lion’s roar\\
Before of the best of Buddhas.

\item The sage gave us the going forth\\
Out of compassion, so that we could realize\\
The ending of all fetters—\\
Now we have reached that goal.

\subsubsection*{2.29 Bhāradvāja}

\item This is how the wise roar:\\
Like lions in mountain caves,\\
Heroes, triumphant in battle,\\
Having vanquished Māra and his army.

\item I’ve attended on the teacher;\\
I’ve honoured the Dhamma and the Saṅgha;\\
I’m happy and joyful,\\
Because I’ve seen my son free of defilements.

\subsubsection*{2.30 Kaṇhadinna}

\item I sat close by wise people,\\
And learnt the Dhamma many times.\\
What I learnt, I practiced,\\
Entering the road that culminates in the deathless.

\item I’ve slain the desire to be reborn \\
In any state of existence,\\
Such desire won’t be found in me again.\\
It was not, and it won’t be in me,\\
And it isn’t found in me now.

\subsubsection*{2.31 Migasira}

\item When I became a monk\\
In the teaching of the Buddha,\\
Letting go, I rose up;\\
I escaped the realm of sensual pleasures.

\item Then, under the supervision \\
Of the supreme Buddha,\\
My mind was freed.\\
I know that my freedom is unshakeable,\\
Because all fetters have ended.

\subsubsection*{2.32 Sivaka}

\item All houses are impermanent;\\
Again and again, in all kinds of realms,\\
I’ve searched for the house-builder—\\
Rebirth again and again is suffering.

\item I’ve seen you, house-builder!\\
You won’t build a house again.\\
All your rafters are broken,\\
Your ridgepole is split.\\
My mind is released from limits:\\
It’ll fall apart in this very life.

\subsubsection*{2.33 Upavāṇa}

\item The Worthy One, the world’s Holy One\\
The sage is afflicted by winds.\\
If there’s hot water,\\
Give it to the sage, brahmin.

\item I wish to bring it to the one\\
Who is honoured by those worthy of honour,\\
Revered by those worthy of reverence,\\
And respected by those worthy of respect.

\subsubsection*{2.34 Isidinna}

\item I’ve seen lay disciples who have memorized discourses,\\
Saying “Sensual pleasures are impermanent”;\\
But they are passionately enamoured \\
Of jewelled earrings,\\
Desiring children and wives.

\item To be honest, they don’t know Dhamma,\\
Despite saying “Sensual pleasures are impermanent”;\\
They don’t have the power to cut their lust,\\
So they’re attached to children, wives, and wealth.

\subsubsection*{2.35 Sambulakaccāna}

\item The sky rains, the sky groans,\\
I’m staying alone in a frightful hole.\\
But while I’m staying alone in that frightful hole,\\
I’ve no fear, no dread, no goosebumps.

\item This is my normal state,\\
When I’m staying alone in a frightful hole:\\
I’ve no fear, no dread, no goosebumps.

\subsubsection*{2.36 Nitaka}

\item Whose mind is like rock,\\
Steady, not trembling?\\
Free of desire among desirable things,\\
Not agitated among agitating things?\\
For one whose mind is developed in this way,\\
From where will suffering come?

\item \emph{My} mind is like rock,\\
Steady, not trembling,\\
Free of desire among desirable things,\\
Not agitated among agitating things.\\
For me, whose mind is developed in this way,\\
From where will suffering come?

\subsubsection*{2.37 Soṇapoṭiriya}

\item Night, with its garland of stars,\\
Is not just for sleeping.\\
Those who are conscious will know\\
That night is also for waking.

\item If I were to fall from the back of an elephant\\
And be trampled by the tuskers that follow,\\
Better for me to die in battle,\\
Than to live on in defeat.

\subsubsection*{2.38 Nisabha}

\item One who has gone forth \\
From the home life out of faith,\\
Leaving behind the five kinds of sensual pleasures,\\
So pleasant seeming, delighting the mind—\\
Let them put an end to suffering.

\item I don’t long for death;\\
I don’t long for life;\\
I await my time,\\
Aware and mindful.

\subsubsection*{2.39 Usabha}

\item I arranged a robe on my shoulder,\\
The colour of young mango sprouts;\\
Then I entered the village for alms,\\
While sitting on an elephant’s neck!

\item But when I dismounted from the elephant,\\
I was moved by inspiration—\\
At first I was burning, but then I was peaceful;\\
I realized the end of defilements.

\subsubsection*{2.40 Kappaṭakura}

\item This fellow, “Rag-rice”, he sure is a rag.\\
This place has been made for practising \emph{jhāna},\\
Like a crystal vase filled to the brim\\
With the nectar of the deathless,\\
Into which enough Dhamma has been poured.

\item Don’t nod off, Rag—\\
I’ll smack your ear!\\
Nodding off in the middle of the Saṅgha?\\
You haven’t learnt a thing.

\subsubsection*{2.41 Kumārakassapa}

\item Oh, the Buddhas! Oh, the Dhammas!\\
Oh, the perfections of the Teacher!\\
Where a disciple may see\\
Such a Dhamma for themselves.

\item Through countless aeons\\
They obtained an identity;\\
This is the end,\\
Their last body;\\
Transmigrating through births and deaths,\\
Now there is no more rebirth \\
Into any state of existence.

\subsubsection*{2.42 Dhammapāla}

\item The young monk\\
Who is devoted to the teaching of the Buddha,\\
Wakeful among those who sleep—\\
His life isn’t in vain.

\item So let the wise devote themselves\\
To faith, virtuous behaviour,\\
Confidence, and insight into Dhamma,\\
Remembering the teachings of the Buddhas.

\subsubsection*{2.43 Brahmāli}

\item Whose faculties have become serene,\\
Like horses tamed by a charioteer?\\
Who has abandoned conceit and defilements,\\
Becoming such that even the gods envy them?

\item \emph{My} faculties have become serene,\\
Like horses tamed by a charioteer?\\
\emph{I} have abandoned conceit and defilements,\\
Becoming such that even the gods envy me.

\subsubsection*{2.44 Mogharāja}

\item \vleftofline{“}Your skin is nasty but your heart is good,\\
Mogharāja, you always have \emph{samādhi}.\\
But in the nights of winter, so dark and cold,\\
How will you get by, monk?”

\item \vleftofline{“}I’ve heard that all the Magadhans\\
Have an abundance of grain.\\
I’ll make my bed under a thatched roof,\\
Just like those who live in comfort.”

\subsubsection*{2.45 Visākhapañcālaputta}

\item One should not suspend others from the Saṅgha,\\
Nor raise objections against them;\\
And neither disparage nor raise one’s voice \\
Against one who has crossed to the further shore.\\
One should not praise oneself among the assemblies,\\
But be without conceit, measured in speech, \\
And of good conduct.

\item For one who sees the goal, so very subtle and fine,\\
Who has wholesome thoughts and humbleness,\\
And cultivates the Buddha’s ethical conduct,\\
It’s not hard to gain \emph{nibbāna}.

\subsubsection*{2.46 Cūḷaka}

\item The peacocks cry out\\
With their fair crests and tails,\\
Their lovely blue necks and fair faces,\\
Their beautiful song and their call.\\
This broad earth is lush with grass and dew,\\
And the sky’s full of beautiful clouds.

\item A person who is practising \emph{jhāna} is happy in mind,\\
And their appearance is uplifting;\\
Going forth in the teaching of the Buddha \\
Is easy for a good person.\\
You should realise that supreme, unchanging state,\\
So very pure, subtle, and hard to see.

\subsubsection*{2.47 Anūpama}

\item The conceited mind, addicted to pleasure,\\
Impales itself on its own stake.\\
It goes only where\\
There’s a stake, a chopping block.

\item I declare you the demon mind!\\
I declare you the insidious mind!\\
You’ve found the teacher so hard to find—\\
Don’t lead me away from the goal.

\subsubsection*{2.48 Vajjita}

\item Transmigrating for such a long time,\\
I’ve evolved through various states of rebirth,\\
Not seeing the noble truths,\\
A blind, unenlightened person.

\item But when I became heedful\\
Transmigrating from birth to birth was disintegrated;\\
All states of rebirth were cut off;\\
Now there is no more rebirth \\
Into any state of existence.

\subsubsection*{2.49 Sandhita}

\item Beneath the Bodhi Tree,\\
Bright green and growing,\\
Being mindful, my perception\\
Became one with the Buddha.

\item It was thirty one aeons ago\\
That I gained that perception;\\
And it is due to that perception\\
That I’ve realized the ending of defilements.

\chapter*{Chapter Three}
\addcontentsline{toc}{chapter}{Chapter Three}
\markboth{Chapter Three}{Chapter Three}

\subsubsection*{3.1 Aṅgaṇikabhāradvāja}

\item Seeking purity the wrong way,\\
I worshipped the sacred fire in a grove.\\
Not knowing the path to purity,\\
I mortified my flesh in search of immortality.

\item I’ve gained \emph{this} happiness by means of happiness:\\
See the excellence of the Dhamma!\\
I’ve attained the three knowledges,\\
And fulfilled the Buddha’s instructions.

\item In the past I was related to Brahmā,\\
But now I’m really a brahmin:\\
I have the three knowledges, I’m cleansed,\\
I’m an initiate, \\
And I’m accomplished in sacred knowledge.

\subsubsection*{3.2 Paccaya}

\item I went forth five days ago,\\
A trainee, with my heart’s goal unfulfilled.\\
I entered my dwelling,\\
And an aspiration arose in my mind.

\item I won’t eat; I won’t drink;\\
I won’t leave my dwelling;\\
Nor will I lie down on my side—\\
Until the dart of craving is pulled out.

\item See my energy and effort\\
As I practice this way!\\
I’ve attained the three knowledges,\\
And fulfilled the Buddha’s instructions.

\subsubsection*{3.3 Bākula}

\item Whoever wishes to do afterwards\\
What they should have done before\\
They’ve lost the causes for happiness,\\
And afterwards they’re tormented with remorse.

\item You should only say what you would do;\\
You shouldn’t say what you wouldn’t do.\\
The wise will recognize\\
One who talks without doing.

\item Oh! \emph{nibbāna} is so very blissful,\\
As taught by the fully awakened Buddha:\\
Sorrowless, stainless, secure;\\
Where suffering all ceases.

\subsubsection*{3.4 Dhaniya}

\item One who hopes for the ascetic life,\\
Wishing to live in happiness,\\
Should not look down on the Saṅgha’s robe,\\
Or its food and drinks.

\item One who hopes for the ascetic life,\\
Wishing to live in happiness,\\
Should stay in the Saṅgha’s lodgings,\\
Like a snake in a mouse hole.

\item One who hopes for the ascetic life,\\
Wishing to live in happiness,\\
Should be satisfied with whatever is offered,\\
Developing this one quality.

\subsubsection*{3.5 Mātaṅgaputta}

\item \vleftofline{“}It’s too cold, too hot,\\
Too late,” they say.\\
Those who neglect their work like this—\\
Opportunities pass them by.

\item But one who considers hot and cold\\
To be nothing more than a blade of grass;\\
He does his manly duty,\\
And his happiness never fails.

\item With my chest I’ll thrust aside\\
The grasses, vines, and creepers,\\
And devote myself to seclusion.

\subsubsection*{3.6 Khujjasobhita}

\item \vleftofline{“}One of those monks who live in Pāṭaliputta—\\
Such brilliant speakers, and very learned—\\
Stands at the door:\\
The old man, Khujjasobhita.

\item One of those monks who live in Pāṭaliputta—\\
Such brilliant speakers, and very learned—\\
Stands at the door:\\
An old man, trembling in the wind.”

\item \vleftofline{“}By war well fought, by sacrifice well made,\\
By victory in battle;\\
By living the spiritual life:\\
That’s how this person flourishes in happiness.”

\subsubsection*{3.7 Vāraṇa}

\item Anyone among men\\
Who harms other creatures:\\
From this world and the next,\\
That person will fall.

\item But someone with a mind of loving-kindness,\\
Compassionate for all creatures:\\
\emph{That} sort of person\\
Gives rise to merit in abundance.

\item One should train in good speech,\\
In attending closely to ascetics,\\
In sitting alone in hidden places,\\
And in calming the mind.

\subsubsection*{3.8 Vassika}

\item I was the only one in my family\\
Who had faith and wisdom.\\
It’s good for my relatives that I’m\\
Firm in Dhamma, and my conduct is virtuous.

\item I rebuked my family out of compassion,\\
Reprimanding them because of my love\\
For my family and relatives.

\item They performed a service for the monks\\
And then they passed away,\\
To find happiness in the heaven of the Thirty-three.\\
There, my brothers and mother rejoice\\
With all the pleasures they desire.

\subsubsection*{3.9 Yasoja}

\item \vleftofline{“}With knobbly knees,\\
Thin, with veins matted on his skin,\\
Eating and drinking in moderation—\\
This person’s spirit is undaunted.”

\item \vleftofline{“}Pestered by gadflies and mosquitoes\\
In the awesome wilderness;\\
One should mindfully endure,\\
Like an elephant at the head of the battle.

\item A monk alone is like Brahmā;\\
A pair of monks are like devas;\\
Three are like a village;\\
And more than that is a rabble.”

\subsubsection*{3.10 Sāṭimattiya}

\item In the past you had faith,\\
Today you have none.\\
What’s yours is yours alone;\\
I’ve done nothing wrong.

\item Faith is impermanent, fickle—\\
So I have seen.\\
People’s passions wax and wane:\\
Why should a sage grow old worrying about that?

\item The meal of a sage is cooked\\
Bit by bit, in this family or that.\\
I’ll walk for alms,\\
For my legs are strong.

\subsubsection*{3.11 Upāli}

\item One newly gone forth,\\
Who has left their home out of faith,\\
Should associate with spiritual friends,\\
Whose livelihood is pure, and who are not lazy.

\item One newly gone forth,\\
Who has left their home out of faith,\\
A monk who stays with the Saṅgha,\\
Being wise, would train in monastic discipline.

\item One newly gone forth,\\
Who has left their home out of faith,\\
Skilled in what is appropriate and what is not,\\
Would wander undistracted.

\subsubsection*{3.12 Uttarapāla}

\item Sadly, I was intelligent and peaceful,\\
But only enough to critically analyse the meaning.\\
The five kinds of sensual pleasure in  the world,\\
So delusory, were my downfall.

\item Entering into Māra’s domain,\\
I was struck by a powerful dart;\\
But I was able to free myself\\
From the trap laid by the king of death.

\item All sensual pleasures have been abandoned,\\
Rebirth in all states of existence is torn apart,\\
Transmigration through births is finished,\\
Now there is no more rebirth \\
Into any state of existence.

\subsubsection*{3.13 Abhibhūta}

\item All my family gathered here,\\
Listen to me,\\
I’ll teach you Dhamma!\\
Being born again and again is suffering.

\item Rouse yourselves, let go!\\
Devote yourselves to the teachings of the Buddha!\\
Crush the army of death,\\
Like an elephant crushes a hut of reeds.

\item Whoever will live heedfully\\
In this Dhamma and discipline,\\
Will abandon transmigration through births,\\
And make an end to suffering.

\subsubsection*{3.14 Gotama}

\item Transmigrating, I went to hell;\\
Again and again, I went to the ghost realm;\\
Many times I dwelt long\\
In the suffering of the animal realm.

\item I was also reborn as a human;\\
From time to time I went to heaven;\\
I’ve stayed in the corporeal realms \\
And the incorporeal,\\
Among the percipient-nor-non-percipient, \\
And the non-percipient.

\item I understood these states of existence \\
To be worthless:\\
Conditioned, unstable, always in motion.\\
When I understood the origin \\
Of rebirth within myself,\\
Mindful, I found peace.

\subsubsection*{3.15 Hārita}

\item Whoever wishes to do afterwards\\
What they should have done before,\\
They’ve lost the causes for happiness,\\
And afterwards they’re tormented with remorse.

\item You should only say what you would do;\\
You shouldn’t say what you wouldn’t do.\\
The wise will recognize\\
One who talks without doing.

\item Oh! \emph{nibbāna} is so very blissful,\\
As taught by the Buddha:\\
Sorrowless, stainless, secure;\\
Where suffering all ceases.

\subsubsection*{3.16 Vimala}

\item Avoiding bad friends,\\
You should associate with the best of people.\\
Stick to the advice that he gave you,\\
Aspiring for unshakable happiness.

\item If someone lost in the middle of the ocean,\\
Were to clamber up on a little log, they’d sink;\\
In the same way, even a good person would sink\\
If they rely on a lazy person.\\
So avoid those who are lazy, lacking energy.

\item Instead, dwell with the wise—\\
Secluded, noble,\\
Resolute, practising \emph{jhāna},\\
And always energetic.

\chapter*{Chapter Four}
\addcontentsline{toc}{chapter}{Chapter Four}
\markboth{Chapter Four}{Chapter Four}

\subsubsection*{4.1 Nāgasamāla}

\item There’s a dancer along the highway,\\
Dancing as the music plays;\\
She’s adorned with jewellery and all dressed up,\\
With a garland of flowers \\
And perfume of sandalwood.

\item I entered for alms,\\
And while going along I glanced at her,\\
Adorned with jewellery and all dressed up,\\
Like a snare of death laid down.

\item Then the realization\\
Came upon me—\\
The danger became clear,\\
And I was firmly repulsed.

\item Then my mind was liberated—\\
See the excellence of the Dhamma!\\
I’ve attained the three knowledges,\\
And fulfilled the Buddha’s instructions.

\subsubsection*{4.2 Bhagu}

\item Overwhelmed by drowsiness,\\
I came out of my dwelling;\\
Stepping up on to the walking meditation path,\\
I fell to the ground right there.

\item I rubbed my limbs, and again\\
I stepped up on to the walking meditation path.\\
I walked meditation up and down the path,\\
Serene inside myself.

\item Then the realization\\
Came upon me—\\
The danger became clear,\\
And I was firmly repulsed.

\item Then my mind was liberated—\\
See the excellence of the Dhamma!\\
I’ve attained the three knowledges,\\
And fulfilled the Buddha’s instructions.

\subsubsection*{4.3 Sabhiya}

\item Others don’t understand\\
That here we come to our end.\\
Those that do understand this\\
Settle their quarrels because of that.

\item And when those who don’t understand\\
Behave as though they were immortal;\\
Those who understand the Dhamma\\
Are like the healthy among the sick.

\item Any lax act,\\
Or corrupt religious observance,\\
Or a spiritual life arousing suspicion,\\
Does not yield great fruit.

\item Whoever has no respect\\
For their companions in the spiritual life\\
Is as far from true Dhamma\\
As the sky is from the earth.

\subsubsection*{4.4 Nandaka}

\item Damn these stinking bodies!\\
They’re on Māra’s side, they ooze;\\
And bodies have nine streams\\
That are always flowing.

\item Don’t think much of bodies;\\
Don’t disparage the Tathāgathas.\\
They’re not even aroused by heaven,\\
Let alone by humans.

\item But those dumb fools,\\
With bad advisors, shrouded in delusion,\\
That kind of person is aroused by bodies,\\
When Māra has thrown the snare.

\item Those who have discarded\\
Lust, hatred, and ignorance:\\
They’ve cut the strings, they’re no longer bound—\\
Such people are not aroused by bodies.

\subsubsection*{4.5 Jambuka}

\item For fifty-five years\\
I wore mud and dirt;\\
Eating one meal a month,\\
I tore out my hair and beard.

\item I stood on one foot;\\
I rejected seats;\\
I ate dried-out dung;\\
I didn’t accept food that had been set aside for me.

\item Having done many actions of this kind,\\
Which lead to a bad destination,\\
As I was being swept away by the great flood,\\
I went to the Buddha for refuge.

\item See the going for refuge!\\
See the excellence of the Dhamma!\\
I’ve attained the three knowledges,\\
And fulfilled the Buddha’s instructions.

\subsubsection*{4.6 Senaka}

\item During the spring festival at Gayā,\\
It was so welcome for me\\
To see the Buddha,\\
Teaching the supreme Dhamma.

\item He was glorious, the teacher of a community,\\
Who had realised the highest, a leader,\\
Conqueror of the world with its gods;\\
His vision was unequalled.

\item A great being of power, a great hero,\\
A great light, without defilements.\\
With the utter ending of all defilements,\\
The teacher has no fear from any direction.

\item For a long time, sadly, I was corrupted,\\
Fettered by the bond of wrong view.\\
That Blessed One, Senaka,\\
Released me from all ties.

\subsubsection*{4.7 Sambhūta}

\item Hurrying when it’s time for going slowly;\\
Going slowly when it’s time to hurry;\\
That fool falls into suffering\\
Because of these muddled arrangements.

\item Their good fortune wastes away\\
Like the moon in the waning fortnight;\\
They become disgraced,\\
And alienated from their friends.

\item Going slowly when it’s time for going slowly;\\
Hurrying when it’s time to hurry;\\
That wise person comes into happiness\\
Because of these proper arrangements.

\item Their good fortune flourishes\\
Like the moon in the waxing fortnight;\\
They become famous and respected,\\
Not alienated from their friends.

\subsubsection*{4.8 Rāhula}

\item I am known as “Fortunate Rāhula”,\\
Because I’m endowed in both ways:\\
I am the son of the Buddha,\\
And I have the vision of the Dhammas.

\item Since my defilements have ended,\\
Since there is no more being reborn \\
In any state of existence—\\
I’m an Arahant, worthy of offerings,\\
With the three knowledges \\
And the vision of the deathless.

\item Blinded by sensual pleasures, trapped in a net,\\
They are smothered over by craving,\\
Bound by the Kinsman of the Negligent,\\
Like a fish caught in a funnel-net trap.

\item Having thrown off those sensual pleasures,\\
Having cut Māra’s bond,\\
Having pulled out craving, roots and all:\\
I’ve become cool, and realized \emph{nibbāna}.

\subsubsection*{4.9 Candana}

\item Covered with gold,\\
Surrounded by all her maids,\\
With my son upon her hip,\\
My wife came up to me.

\item I saw her coming,\\
The mother of my son,\\
Adorned with jewellery and all dressed up,\\
Like a snare of death laid down.

\item Then the realization\\
Came upon me—\\
The danger became clear,\\
And I was firmly repulsed.

\item Then my mind was liberated—\\
See the excellence of the Dhamma!\\
I’ve attained the three knowledges,\\
And fulfilled the Buddha’s instructions.

\subsubsection*{4.10 Dhammika}

\item \vleftofline{“}Dhamma really protects you if you practice Dhamma;\\
Dhamma well-practiced brings happiness.\\
If you practice Dhamma, this is the benefit—\\
You won’t go to a bad destination.

\item Dhamma and what is not Dhamma\\
Don’t both lead to the same results.\\
What is not Dhamma leads to hell,\\
While Dhamma takes you to a good destination.

\item So you should be enthusiastic \\
To perform acts of Dhamma,\\
Rejoicing in the Fortunate One, the poised.\\
Disciples of the best of Fortunate Ones \\
Are firm in Dhamma;\\
Those wise ones are led on, \\
Going to the very best of refuges.”

\item \vleftofline{“}The boil has been burst from its root,\\
The net of craving is undone.\\
He has ended transmigration, he has nothing,\\
Just like the full moon in a clear night sky.”

\subsubsection*{4.11 Sappaka}

\item When the crane with its beautiful white wings,\\
Startled by fear of the dark thundercloud,\\
Flees, seeking shelter—\\
Then the River Ajakaraṇī delights me.

\item When the crane, so pure and white,\\
Startled by fear of the dark thundercloud,\\
Seeks for a cave to shelter in, but can’t see one—\\
Then the River Ajakaraṇī delights me.

\item Who wouldn’t be delighted\\
By the rose-apple trees\\
That adorn both banks of the river there,\\
Behind my cave?

\item Rid of snakes, that death-mad swarm,\\
 The lazy frogs croak:\\
\vleftofline{“}Today isn’t the time to stray from mountain streams;\\
Ajakaraṇī is safe, pleasant, and delightful.”

\subsubsection*{4.12 Mudita}

\item I went forth to save my life;\\
But I gained faith\\
After receiving full ordination;\\
And I strove, strong in effort.

\item With pleasure, let this body be broken!\\
Let this lump of flesh be dissolved!\\
Let both my legs fall off\\
At the knees!

\item I won’t eat, I won’t drink,\\
I won’t leave my dwelling,\\
Nor will I lie down on my side,\\
Until the dart of craving is drawn out.

\item As I dwell like this,\\
See my energy and striving!\\
I’ve attained the three knowledges,\\
And fulfilled the Buddha’s instructions.

\chapter*{Chapter Five}
\addcontentsline{toc}{chapter}{Chapter Five}
\markboth{Chapter Five}{Chapter Five}

\subsubsection*{5.1 Rājadatta}

\item I, a monk, went to a charnel ground\\
And saw a woman left there,\\
Discarded in a cemetery,\\
Full of worms that devoured her.

\item Some men were disgusted,\\
Seeing her dead and rotten;\\
But sexual desire arose in me,\\
I was as if blind to her oozing body.

\item Quicker than the boiling of rice\\
I left that place,\\
Mindful and aware,\\
I sat down to one side.

\item Then the realization\\
Came upon me—\\
The danger became clear,\\
And I was firmly repulsed.

\item Then my mind was liberated—\\
See the excellence of the Dhamma!\\
I’ve attained the three knowledges,\\
And fulfilled the Buddha’s instructions.

\subsubsection*{5.2 Subhūta}

\item If a person, wishing for a certain outcome,\\
Applies themseves to a misguided endeavor,\\
And they don’t achieve what they have practiced for,\\
They say: “That’s a sign of my bad luck.”

\item When a misfortune \\
Has been plucked out and conquered,\\
To give it up in part would be like \\
The losing throw of the dice;\\
But to give up everything \\
Would be as if one was blind,\\
Not discerning the even and the uneven.

\item You should only say what you would do;\\
You shouldn’t say what you wouldn’t do.\\
The wise will recognize\\
One who talks without doing.

\item Just like a glorious flower\\
That’s colourful but lacks fragrance;\\
So is well-spoken speech fruitless\\
For one not acting in accordance.

\item Just like a glorious flower\\
Is both colourful and fragrant,\\
So is well-spoken speech fruitful\\
For one who acts in accordance.

\subsubsection*{5.3 Girimānanda}

\item The sky rains, like a beautiful song,\\
My little hut is roofed and pleasant, \\
Sheltered from the wind,\\
I dwell there peacefully\\
So rain, sky, as you please.

\item The sky rains, like a beautiful song,\\
My little hut is roofed and pleasant, \\
Sheltered from the wind,\\
I dwell there, with peaceful mind:\\
So rain, sky, as you please.

\item The sky rains, like a beautiful song,\\
My little hut is roofed and pleasant, \\
Sheltered from the wind,\\
I dwell there, free of lust:\\
So rain, sky, as you please.

\item The sky rains, like a beautiful song,\\
My little hut is roofed and pleasant, \\
Sheltered from the wind,\\
I dwell there, free of hate:\\
So rain, sky, as you please.

\item The sky rains, like a beautiful song,\\
My little hut is roofed and pleasant, \\
Sheltered from the wind,\\
I dwell there, free of delusion:\\
So rain, sky, as you please.

\subsubsection*{5.4 Sumana}

\item My mentor helped me to learn,\\
Hoping that I would practice those teachings;\\
Aspiring for the deathless,\\
I’ve done what was to be done.

\item I’ve arrived at the Dhamma,\\
And witnessed it for myself, not based on hearsay.\\
With purified knowledge, free of doubt,\\
I declare it in your presence.

\item I know my past life;\\
My clairvoyance is clarified;\\
I’ve realized my own true goal,\\
The Buddha’s instruction is fulfilled.

\item Being heedful in the training,\\
I have learned your teachings well.\\
All my defilements are ended;\\
Now there is no more rebirth \\
Into any state of existence.

\item You advised me in noble ways;\\
Compassionate, you helped teach me;\\
Your instruction was not in vain—\\
I, your student, am fully trained.

\subsubsection*{5.5 Vaḍḍha}

\item It was good, how my mother\\
Spurred me onwards.\\
When I heard her words,\\
Advised by my mother,\\
I became energetic, resolute—\\
I realised supreme awakening.

\item I’m an Arahant, worthy of offerings,\\
With the three knowledges \\
And the vision of the deathless;\\
I conquered Namuci’s army,\\
And now I live without defilements.

\item The defilements which I had,\\
Both internally and externally,\\
Are now all cut off without remainder;\\
They won’t arise again.

\item My skilful sister\\
said this to me:\\
\vleftofline{“}Now neither you nor I\\
Have any entanglements.”

\item Suffering is at an end;\\
This is the last body\\
To transmigrate through birth and death:\\
Now there is no more rebirth \\
Into any state of existence.

\subsubsection*{5.6 Nadīkassapa}

\item It was truly for my benefit\\
That the Buddha went to the river Nerañjara;\\
When I heard his teaching,\\
I rejected wrong view.

\item Previously, I performed the higher \\
And lower sacrifices;\\
I worshipped the sacred flame,\\
Thinking, “This is purity.”\\
I was a blind, unenlightened person.

\item Caught in the thicket of wrong view,\\
Deluded by misapprehension;\\
Thinking impurity was purity,\\
I was blind and ignorant.

\item I’ve abandoned wrong view,\\
Rebirth into any state of existence is torn apart,\\
I worship what is truly worthy of offerings:\\
I bow to the Tathāgata.

\item I’ve abandoned all delusion \\
Rebirth into any state of existence is torn apart,\\
Transmigration through births is finished,\\
Now there is no more rebirth \\
Into any state of existence.

\subsubsection*{5.7 Gayākassapa}

\item Three times a day—\\
Morning, midday, and evening—\\
I went down into the water at Gayā,\\
For the Gayā spring festival.

\item \vleftofline{“}Whatever bad things I’ve done\\
In previous births,\\
I’ll now wash away right here”—\\
This is the view I previously held.

\item Having heard the well-spoken words\\
Regarding the Dhamma and the goal,\\
I wisely reflected\\
On the true, essential goal.

\item I’ve washed away all bad things\\
I’m stainless, cleansed, pristine;\\
The pure heir of the pure one,\\
A rightful son of the Buddha.

\item When I plunged into the eight-fold stream,\\
All bad things were washed away.\\
I’ve attained the three knowledges,\\
And fulfilled the Buddha’s instructions.

\subsubsection*{5.8 Vakkali}

\item \vleftofline{“}Struck by a wind ailment,\\
While staying in a forest grove;\\
You’ve gone into a tough place for gathering alms:\\
How will you get by, monk?”

\item \vleftofline{“}Pervading my body\\
With lots of rapture and happiness,\\
Putting up with what’s tough,\\
I’ll dwell in the forest.

\item Developing the establishments of mindfulness,\\
The faculties and the powers,\\
Developing the factors of awakening,\\
I’ll dwell in the forest.

\item Having seen those who are energetic, resolute,\\
Always of strong effort,\\
Harmonious and serene,\\
I’ll dwell in the forest.

\item Recollecting the Buddha,\\
The highest, the tamed, who has \emph{samādhi};\\
Not lazy by day or by night,\\
I’ll dwell in the forest.”

\subsubsection*{5.9 Vijitasena}

\item I’ll cage you, mind,\\
Like an elephant in a stockade.\\
Born of the flesh, that net of sensual pleasures,\\
I won’t urge you to do bad.

\item Caged, you won’t escape,\\
Like an elephant who can’t find an open gate.\\
Demon-mind, you won’t wander again and again,\\
Bullying, and loving to do bad.

\item Just as a strong trainer with a hook,\\
Takes a wild, newly captured elephant\\
And wins it over against its will,\\
So I’ll win you over.

\item Just as a fine charioteer, \\
Skilled in the taming  of fine horses, \\
Tames a thoroughbred,\\
So, firmly established in the five powers,\\
I’ll tame you.

\item I’ll bind you with mindfulness,\\
I’m committed to taming you;\\
Restrained by harnessed energy,\\
Mind, you won’t go far from here.

\subsubsection*{5.10 Yasadatta}

\item With fault-finding mind, the dullard\\
Listens to the conqueror’s teaching.\\
They’re as far from true Dhamma,\\
As the earth is from the sky.

\item With fault-finding mind, the dullard\\
Listens to the conqueror’s teaching.\\
They decline in the true Dhamma,\\
Like the moon in the waning fortnight.

\item With fault-finding mind, the dullard\\
Listens to the conqueror’s teaching.\\
They wither away in the true Dhamma,\\
Like a fish in too little water.

\item With fault-finding mind, the dullard\\
Listens to the conqueror’s teaching.\\
They don’t thrive in the true Dhamma,\\
Like a rotten seed in a field.

\item But one with contended mind\\
Who listens to the conqueror’s teachings—\\
Having ended all defilements,\\
Having witnessed the unshakable,\\
Having arrived at the highest peace—\\
They realize \emph{nibbāna} without defilements.

\subsubsection*{5.11 Soṇakuṭikaṇṇa}

\item I’ve received full ordination,\\
I am liberated, without defilements,\\
I’ve seen the Blessed One myself,\\
And even stayed together with him.

\item The Blessed One, the teacher,\\
Spent much of the night in the open;\\
Then he, who is so skilled at dwelling in meditation,\\
Entered his dwelling.

\item Spreading out his outer robe,\\
Gotama made his bed;\\
Like a lion in a rocky cave,\\
With fear and dread abandoned.

\item Then, with lovely enunciation,\\
Soṇa, a disciple of the Buddha,\\
Recited the true Dhamma\\
In the presence of the best of Buddhas.

\item When he has fully understood the five aggregates,\\
Developed the straight path,\\
And arrived at the highest peace,\\
He will realize \emph{nibbāna} without defilements.

\subsubsection*{5.12 Kosiya}

\item Whatever wise person, \\
Understanding their teacher’s words,\\
Stays with them, their affection growing;\\
That wise person is indeed devoted—\\
Knowing about Dhammas, they’re distinguished.

\item When extreme stresses arise,\\
Whoever does not tremble, but reflects instead,\\
That wise person is indeed strong—\\
Knowing about Dhammas, they’re distinguished.

\item Steady as the ocean, imperturbable,\\
Their wisdom is deep, and they see the subtle goal;\\
That wise person is indeed immovable—\\
Knowing about Dhammas, they’re distinguished.

\item They’re very learned, and have memorized the Dhamma,\\
practising Dhamma in accordance with Dhamma;\\
That wise person is indeed such—\\
Knowing about Dhammas, they’re distinguished.

\item They know the meaning of what is said,\\
And having known, they act appropriately;\\
That wise person is indeed \\
One who has mastered the meaning—\\
Knowing about Dhammas, they’re distinguished.

\chapter*{Chapter Six}
\addcontentsline{toc}{chapter}{Chapter Six}
\markboth{Chapter Six}{Chapter Six}

\subsubsection*{6.1 Uruveḷakassapa}

\item When I saw the marvels\\
Of the renowned Gotama,\\
I didn’t immediately bow to him;\\
I was blinded by jealousy and conceit.

\item Knowing what I was thinking,\\
The trainer of men spurred me on;\\
And I was struck with a marvellous inspiration,\\
That gave me goose-bumps.

\item Rejecting my petty accomplishments\\
When I used to be a matted-hair ascetic,\\
I then went forth,\\
In the conqueror’s teaching.

\item I used to be content with sacrifice,\\
Giving priority to the realm of sensual pleasures,\\
But later I uprooted desire,\\
And hatred and also delusion.

\item I know my past life;\\
My clairvoyance is clarified;\\
I have psychic powers, \\
And I know the minds of others;\\
I have realised the divine ear.

\item I’ve attained the goal\\
For the sake of which I went forth\\
From home life into homelessness—\\
The ending of all fetters.

\subsubsection*{6.2 Tekicchakāri}

\item \vleftofline{“}The rice has been harvested,\\
And gathered on the threshing-floor—\\
But I don’t get any alms-food!\\
How will I get by?”

\item \vleftofline{“}Recollect the immeasurable Buddha!\\
Confident, your body pervaded with rapture,\\
You’ll always be full of joy.

\item Recollect the immeasurable Dhamma!\\
Confident, your body pervaded with rapture,\\
You’ll always be full of joy.

\item Recollect the immeasurable Saṅgha!\\
Confident, your body pervaded with rapture,\\
You’ll always be full of joy.”

\item \vleftofline{“}You stay in the open,\\
Though these winter nights are cold.\\
Don’t perish, overcome with cold;\\
Enter your dwelling, with its door shut fast.”

\item \vleftofline{“}I’ll realise the four immeasurable states,\\
And dwell happily with them.\\
I won’t perish, overcome with cold;\\
I’ll dwell unperturbed.”

\subsubsection*{6.3 Mahānāga}

\item Whoever has no respect\\
For their companions in the spiritual life\\
Falls away from the true Dhamma,\\
Like a fish in too little water.

\item Whoever has no respect\\
For their companions in the spiritual life\\
Doesn’t thrive in the true Dhamma,\\
Like a rotten seed in a field.

\item Whoever has no respect\\
For their companions in the spiritual life\\
Is far from \emph{nibbāna},\\
In the teaching of the Dhamma king.

\item Whoever does have respect\\
For their companions in the spiritual life\\
Doesn’t fall away from the true Dhamma,\\
Like a fish in plenty of water.

\item Whoever does have respect\\
For their companions in the spiritual life\\
Thrives in the true Dhamma,\\
Like a quality seed in a field.

\item Whoever does have respect\\
For their companions in the spiritual life\\
Is close to \emph{nibbāna},\\
In the teaching of the Dhamma king.

\subsubsection*{6.4 Kulla}

\item I, Kulla, went to a charnel ground\\
And saw a woman left there,\\
Discarded in a cemetery,\\
Full of worms that devoured her.

\item See this body, Kulla—\\
Diseased, filthy, rotten,\\
Oozing and trickling,\\
A fools’ delight.

\item Taking Dhamma as a mirror\\
For realizing knowledge and vision,\\
I reviewed this body,\\
Vacant, inside and out.

\item As this is, so is that;\\
As that is, so is this.\\
As below, so above;\\
As above, so below.

\item As by day, so by night;\\
As by night, so by day.\\
As before, so after;\\
As after, so before.

\item Not even music played by a five-piece band,\\
Can give such pleasure\\
As there is for one with unified mind,\\
Discerning the Dhamma rightly.

\subsubsection*{6.5 Mālukyaputta}

\item For a person who lives heedlessly,\\
Craving grows like a parasitic creeper.\\
They jump from here to there, like a monkey\\
That wants fruit in a forest grove.

\item Whoever is overcome by this wretched craving,\\
This attachment to the world,\\
Their sorrow grows,\\
Like grass in the rain.

\item But whoever overcomes this wretched craving,\\
This attachment to the world,\\
Their sorrows fall from them,\\
Like a water-drop from a lotus.

\item I say this to you, venerables,\\
All those who have gathered here:\\
Dig up the root of craving,\\
Like someone who is looking for roots \\
Will dig up the grass.\\
Don’t let Māra break you again and again,\\
Like a stream breaking a reed.

\item Act on the Buddha’s words,\\
Don’t let the moment pass you by.\\
Those who pass up the moment\\
Grieve when they end up in hell.

\item Heedlessness is always an impurity,\\
Impurity comes from heedlessness.\\
With heedfulness and knowledge,\\
Pluck out your own dart.

\subsubsection*{6.6 Sappadāsa}

\item In the twenty-five years\\
Since I went forth,\\
I have not found peace of mind,\\
Even for as long as a finger-snap.

\item Since I couldn’t get my mind unified,\\
I was tormented by sexual desire.\\
Wailing, with outstretched arms,\\
I burst out of my dwelling.

\item Should I … or should I take the knife?\\
What’s the point of living?\\
Rejecting the training,\\
How should one like me come to an end?

\item Then I picked up a razor;\\
And sat on a bench;\\
The razor was ready—\\
To cut my vein.

\item Then the realization\\
Came upon me—\\
The danger became clear,\\
And I was firmly repulsed.

\item Then my mind was liberated—\\
See the excellence of the Dhamma!\\
I’ve attained the three knowledges,\\
And fulfilled the Buddha’s instructions.

\subsubsection*{6.7 Kātiyāna}

\item Get up, Kātiyāna, and sit!\\
Don’t sleep too much, be wakeful.\\
Don’t be lazy, and let the kinsman of the heedless,\\
The king of death, catch you in his trap.

\item Like a wave in the mighty ocean,\\
Birth and old age overwhelm you.\\
Make a safe island of yourself,\\
For you have no other shelter.

\item The teacher has mastered this path,\\
Which transcends ties, \\
And the fear of birth and old age.\\
Be heedful all the time,\\
And devote yourself to dedicated practice.

\item Be free of your former bonds!\\
Wearing outer robe, \\
With shaven head, eating almsfood,\\
Don’t delight in play or sleep,\\
Devote yourself to \emph{jhāna}, Kātiyāna.

\item Practice \emph{jhāna} and conquer, Kātiyāna,\\
You’re skilled in the path to security from the yoke.\\
Attaining unexcelled purity,\\
You’ll be quenched, like a flame by water.

\item A lamp with feeble flames\\
Is bent down by the wind, like a creeper;\\
Just so, kinsman of Indra,\\
You shake off Māra, without grasping.\\
Free of lust for feelings,\\
Await your time here, cooled.

\subsubsection*{6.8 Migajāla}

\item It was well-taught by the one who sees,\\
The Buddha, Kinsman of the Sun,\\
Who has gone beyond all fetters,\\
And destroyed all rolling-on.

\item Leading to liberation, crossing over,\\
Drying up the root of craving,\\
Cutting off the root of poison, the slaughter-house,\\
And leading to \emph{nibbāna}.

\item By breaking the root of unknowing,\\
It smashes the mechanism of deeds,\\
And looses the thunderbolt of knowledge\\
On the taking up of consciousnesses.

\item Informing us about our feelings,\\
Freeing us from grasping,\\
Wisely contemplating all states of existence\\
As a pit of burning coals.

\item Very sweet, very deep,\\
Preventing birth and death,\\
Leading to the stilling of suffering, bliss—\\
It is the noble eightfold path.

\item Knowing deed as deed,\\
And result as result;\\
Seeing dependently originated phenomena\\
As if they were in a clear light;\\
Leading to great security, peace,\\
It’s excellent at the end.

\subsubsection*{6.9 Purohitaputtajenta}

\item I was intoxicated with the pride of birth,\\
And wealth and sovereignty,\\
I lived intoxicated\\
With the beauty and form of my body.

\item No-one was my equal or my better—\\
Or so I thought.\\
I was such an arrogant fool,\\
Stuck up, waving my own flag.

\item I didn’t pay respects to anyone:\\
Not my mother or father,\\
Nor others considered to be honorable.\\
I was stiff with conceit, and disrespectful.

\item When I saw the supreme leader,\\
The most excellent of charioteers,\\
Shining like the sun,\\
And revered by the monastic Saṅgha,

\item I discarded conceit and intoxication,\\
And, with a clear and confident heart,\\
I bowed down with my head\\
To the highest of all beings.

\item The conceit of superiority \\
And the conceit of inferiority\\
Have been abandoned and uprooted.\\
The conceit “I am” has been eradicated,\\
And every kind of conceit has been destroyed.

\subsubsection*{6.10 Sumana}

\item I had just gone forth,\\
I was seven years old,\\
When I overcome the dragon king, so mighty,\\
With my psychic powers.

\item And I brought water for my mentor\\
From the great lake Anotatta.\\
When he saw me,\\
My teacher said this:

\item \vleftofline{“}Sāriputta, see this\\
Young boy coming,\\
Carrying a water pot,\\
Serene inside himself.

\item His conduct inspires confidence,\\
He is of lovely deportment:\\
He is Anuruddha’s novice,\\
Excelling in psychic powers.

\item Made a thoroughbred by a thoroughbred,\\
Made good by the good,\\
Educated and trained by Anuruddha,\\
Who has completed his work.

\item Having attained the highest peace\\
And witnessed the unshakable,\\
That novice Sumana\\
Wants no-one to know about him.”

\subsubsection*{6.11 Nhātakamuni}

\item \vleftofline{“}Struck by a wind ailment,\\
While staying in a forest grove;\\
You’ve gone into a tough place for gathering alms:\\
How will you get by, monk?”

\item \vleftofline{“}Pervading my body\\
With lots of rapture and happiness,\\
Putting up with what’s tough,\\
I’ll dwell in the forest.

\item Developing the seven factors of awakening,\\
The faculties and the powers,\\
Endowed with subtle \emph{jhānas},\\
I’ll dwell without defilements.

\item Freed from stains,\\
My pure mind is undisturbed;\\
Frequently reviewing this,\\
I’ll dwell without defilements.

\item Those defilements that were found in me,\\
Internally and externally,\\
Are all cut off without remainder,\\
And will not arise again.

\item The five aggregates are fully understood;\\
They remain with their root cut off.\\
I have attained the ending of suffering,\\
Now there is no more rebirth \\
Into any state of existence.”

\subsubsection*{6.12 Brahmadatta}

\item For one without anger, tamed, living calmly,\\
Liberated by right knowledge,\\
At peace, poised:\\
Where would anger come from?

\item One who gets angry at an angry person\\
Just makes things worse.\\
One who doesn’t get angry at an angry person\\
Wins a battle hard to win.

\item When you know that the other is angry,\\
You act for the good of both\\
Yourself and the other,\\
If you are mindful, and stay calm.

\item Those who are unskilled in Dhamma\\
Consider one who heals both\\
Oneself and the other\\
To be a fool.

\item If anger arises in you,\\
Reflect on the simile of the saw;\\
If craving for flavours arises in you,\\
Remember the simile of the son’s flesh.

\item If your mind runs\\
Among sensual pleasures \\
And rebirth in various states of existence,\\
Quickly curb it with mindfulness,\\
As one would curb a greedy cow eating corn.

\subsubsection*{6.13 Sirimaṇḍa}

\item The rain saturates things that are covered up;\\
It doesn’t saturate things that are open.\\
Therefore you should open up a covered thing,\\
So the rain will not saturate it.

\item The world is crushed by death,\\
Surrounded by old age,\\
Struck by the dart of craving,\\
And ever obscured by desire.

\item The world is crushed by death,\\
Caged by old age,\\
Beaten constantly, without respite,\\
Like a thief being flogged.

\item Three things are coming, like a wall of flame:\\
Death, disease, and old age.\\
No power can stand before them,\\
And there is no speed to flee.

\item Don’t waste your day,\\
A little or a lot.\\
Every night that passes\\
Shortens your life by that much.

\item Walking or standing,\\
Sitting or lying down:\\
Your final night draws near.\\
You have no time to be heedless.

\subsubsection*{6.14 Sabbakāmi}

\item Though this two-legged body is dirty and stinking,\\
Full of different carcasses,\\
And oozing from various places,\\
Still it is cherished.

\item Like a hidden deer by a trick,\\
Like a fish by a hook,\\
Like a monkey by tar—\\
They trap an unawakened person.

\item Sights, sounds, tastes, smells,\\
And touches, all delighting the mind.\\
These five kinds of sensual pleasure\\
Are seen in a woman’s body.

\item Those unawakened people, their minds full of lust,\\
Who pursue those women;\\
They swell the horrors of the charnel ground,\\
Piling up more rebirth \\
Into various states of existence.

\item The one who avoids them,\\
Like a snake’s head with a foot,\\
Mindful, he transcends\\
Attachment to the world.

\item Seeing the danger in sensual pleasures,\\
And recognizing renunciation as safety,\\
I’ve escaped all sensual pleasures,\\
And attained the end of defilements.

\chapter*{Chapter Seven}
\addcontentsline{toc}{chapter}{Chapter Seven}
\markboth{Chapter Seven}{Chapter Seven}

\subsubsection*{7.1 Sundarasamudda}

\item She was adorned with jewellery and all dressed up,\\
With a garland of flowers \\
And perfume of sandalwood,\\
Her feet brightly rouged:\\
A courtesan wearing slippers.

\item She took off her slippers in front of me,\\
Her hands in \emph{añjalī},\\
And softly and sweetly\\
She spoke to me, smiling:

\item \vleftofline{“}You’re too young to have gone forth;\\
Come, stay in my teaching!\\
Enjoy human sensual pleasures,\\
I’ll give you riches.\\
I promise this is the truth—\\
I’ll swear by the Sacred Flame.

\item And when we’ve grown old together,\\
Leaning on sticks,\\
We’ll both go forth,\\
So we’ll win both ways.”

\item When I saw the courtesan seducing me,\\
Her hands in \emph{añjalī},\\
Adorned with jewellery and all dressed up,\\
Like a snare of death laid down.

\item Then the realization\\
Came upon me—\\
The danger became clear,\\
And I was firmly repulsed.

\item Then my mind was liberated—\\
See the excellence of the Dhamma!\\
I’ve attained the three knowledges,\\
And fulfilled the Buddha’s instructions.

\subsubsection*{7.2 Lakuṇḍakabhaddiya}

\item Bhaddiya has plucked out craving, root and all,\\
And in a jungle thicket\\
On the far side of Ambāṭaka park,\\
He practices \emph{jhāna}; he is truly well-favoured.

\item Some delight in drums,\\
In lutes, and in cymbals;\\
But here, at the foot of a tree,\\
I delight in the Buddha’s teaching.

\item If the Buddha were to grant me one wish,\\
And I were to get what I wished for,\\
I’d choose that the whole world\\
Be always mindful of the body.

\item Those who’ve judged me by my appearance,\\
And those who’ve followed me because of my voice,\\
They’re under the sway of desire and lust;\\
They don’t know me.

\item Not knowing what’s inside,\\
Not seeing what’s outside;\\
The fool, obstructed all around,\\
Is carried away by my voice.

\item Not knowing what’s inside,\\
But discerning what’s outside;\\
They too, seeing only the external fruits of practice,\\
Are carried away by my voice.

\item Understanding what’s inside,\\
And discerning what’s outside;\\
They, seeing without obstacles,\\
Are not carried away by my voice.

\subsubsection*{7.3 Bhadda}

\item I was an only child,\\
Loved by my mother and father.\\
They had me by practising\\
Many prayers and observances.

\item Out of compassion,\\
Wishing me well and seeking my welfare,\\
My mother and father\\
Took me up to the Buddha.

\item \vleftofline{“}We had this son with difficulty;\\
He is delicate, and has grown up in comfort.\\
We offer him to you, Lord,\\
To attend upon the conqueror.”

\item The teacher, having accepted me,\\
Declared to Ānanda:\\
\vleftofline{“}Quickly give him the going-forth—\\
This one will be a thoroughbred.”

\item After he, the teacher, had sent me forth,\\
The conqueror entered his dwelling.\\
Before the sun set,\\
My mind was liberated.

\item The teacher didn’t neglect me;\\
When he came out from seclusion,\\
He said: “Come Bhadda!”\\
That was my full ordination.

\item At seven years old\\
I received full ordination.\\
I’ve attained the three knowledges;\\
Oh, the excellence of the Dhamma!

\subsubsection*{7.4 Sopāka}

\item I saw the most excellent person,\\
Walking meditation in the shade of the terrace,\\
So I approached,\\
And bowed to the most excellent man.

\item Arranging my robe over one shoulder,\\
And clasping my hands together,\\
I walked meditation alongside that stainless one,\\
Most excellent of all beings.

\item The wise one, skilled in questions,\\
Questioned me.\\
Brave and fearless,\\
I answered the teacher.

\item When all his questions were answered,\\
The Tathāgata congratulated me.\\
Looking around the monastic Saṅgha,\\
He said this:

\item \vleftofline{“}It is a blessing for the people of Aṅga and Magadha\\
That this person enjoys their\\
Robe and almsfood,\\
Requisites and lodgings,\\
Their respect and service—\\
It’s a blessing for them,” he declared.

\item \vleftofline{“}Sopāka, from this day on\\
You are invited to come and see me.\\
And Sopāka, let this\\
Be your full ordination.”

\item At seven years old\\
I received full ordination.\\
I bear my final body—\\
Oh, the excellence of the Dhamma!

\subsubsection*{7.5 Sarabhaṅga}

\item I broke the reeds off with my hands,\\
Made a hut, and stayed there.\\
So I became known as “Reed-breaker”.

\item But now it’s not appropriate\\
For me to break reeds with my hands.\\
The training rules have been laid down for us\\
By Gotama the renowned.

\item Previously, I, Sarabhaṅga,\\
Didn’t see the disease in its entirety.\\
But now I have seen the disease,\\
Because I practised what was taught \\
By the one beyond the gods.

\item Gotama travelled by that straight road;\\
The same path travelled by Vipassī,\\
The same path as Sikhī, Vessabhū,\\
Kakusandha, Koṇāgamana, and Kassapa.

\item By these seven Buddhas, \\
Who plunged into the ending,\\
Free of craving, without grasping,\\
Having become Dhamma, poised,\\
This Dhamma was taught,

\item Out of compassion for living beings—\\
Suffering, origin, path,\\
And cessation, the ending of suffering.\\
In these four noble truths,

\item Suffering is stopped,\\
This endless transmigration.\\
When the body has broken up,\\
And life has come to an end,\\
There is no more rebirth into any state of existence:\\
I’m well-liberated in every way.

\chapter*{Chapter Eight}
\addcontentsline{toc}{chapter}{Chapter Eight}
\markboth{Chapter Eight}{Chapter Eight}

\subsubsection*{8.1 Mahākaccāyana}

\item Don’t get involved in lots of work,\\
Avoid people, and don’t try to get more requisites.\\
If you’re eager and greedy for flavours,\\
You’ll miss the goal that brings such happiness.

\item They know that this really is a bog,\\
This homage and veneration \\
Among respectable families.\\
Honor is a subtle dart, hard to extract,\\
And hard for a bad man to give up.

\item Your deeds aren’t bad\\
Because of what others do.\\
You yourself should not do bad,\\
For people have deeds as their kin.

\item You’re not a criminal \\
Because of what someone else says,\\
And you’re not a sage \\
Because of what someone else says;\\
But as you know yourself,\\
So the gods will know you.

\item Others don’t understand,\\
That here we come to an end.\\
Those who do understand this\\
Settle their quarrels.

\item A wise person lives on,\\
Even after their wealth is lost;\\
But without gaining wisdom,\\
Even a wealthy person doesn’t really live.

\item All is heard with the ear,\\
All is seen with the eye;\\
The wise would not think that all \\
That is seen and heard\\
Is worthy of rejection.

\item Though you have eyes, be as if blind;\\
Though you have ears, be as if deaf,\\
Though you have wisdom, be as if stupid,\\
Though you have strength, be as if feeble.\\
Then, when the goal has been realised,\\
You may lie on your death-bed.

\subsubsection*{8.2 Sirimitta}

\item Without anger or resentment,\\
Without deceit, and rid of slander,\\
Such a monk, poised,\\
Doesn’t sorrow after death.

\item Without anger or resentment,\\
Without deceit, and rid of slander,\\
A monk with sense doors guarded,\\
Doesn’t sorrow after death.

\item Without anger or resentment,\\
Without deceit, and rid of slander,\\
A monk of good virtue\\
Doesn’t sorrow after death.

\item Without anger or resentment,\\
Without deceit, and rid of slander,\\
A monk with good friends\\
Doesn’t sorrow after death.

\item Without anger or resentment,\\
Without deceit, and rid of slander,\\
A monk of good wisdom,\\
Doesn’t sorrow after death.

\item Whoever has faith in the Tathāgatha,\\
That is unshakable and firmly established,\\
Whose ethics are good,\\
Pleasing to the noble ones, and praiseworthy,

\item Who has confidence in the Saṅgha,\\
And whose vision is straight—\\
They’re called “free from poverty”;\\
Their life is not wasted.

\item Therefore a wise person would devote themselves\\
To faith, virtue, \\
Confidence, and the vision of Dhamma,\\
Remembering the teaching of the Buddhas.

\subsubsection*{8.3 Mahāpanthaka}

\item When I first saw the teacher,\\
Who was free of fear from any direction,\\
I was struck with awe,\\
Since I’d seen the best of men.

\item If you have the good luck\\
To find such a teacher,\\
But you push it all away,\\
You’ll lose your chance.

\item Then I left behind my children and wife,\\
My riches and my grain;\\
I cut off my hair and beard,\\
And went forth into homelessness.

\item Endowed with the monastic training and livelihood,\\
My sense faculties well-restrained,\\
Paying homage to the Buddha,\\
I dwelt undefeated.

\item Then a resolve occurred to me,\\
My heart’s truest wish:\\
I wouldn’t sit down, not even for a moment,\\
Until the dart of craving was pulled out.

\item As I dwell like this,\\
See my energy and striving!\\
I’ve attained the three knowledges,\\
And fulfilled the Buddha’s instructions.

\item I know my past life;\\
My clairvoyance is clarified;\\
I’m an Arahant, worthy of offerings,\\
Liberated, without attachments.

\item Then, at the end of the night,\\
As the rising of the sun drew near,\\
All craving was dried up,\\
So I sat down cross-legged.

\chapter*{Chapter Nine}
\addcontentsline{toc}{chapter}{Chapter Nine}
\markboth{Chapter Nine}{Chapter Nine}

\subsubsection*{9.1 Bhūta}

\item When a wise person fully understands \\
That old age and death—\\
To which an ignorant unawakened person is bound—\\
Are suffering; and they are mindful, practising \emph{jhāna}:\\
There is no greater pleasure than this.

\item When attachment, the carrier of suffering,\\
And craving, the carrier of the suffering \\
Of this mass of proliferation,\\
Are destroyed; and they are mindful, practising \emph{jhāna}:\\
There is no greater pleasure than this.

\item When the blissful eightfold way,\\
The supreme path, cleanser of all stains,\\
Is seen with wisdom; \\
And they are mindful, practising \emph{jhāna}:\\
There is no greater pleasure than this.

\item When one develops that peaceful state,\\
Sorrowless, stainless, unconditioned,\\
Cleanser of all stains, \\
And cutter of fetters and bonds:\\
There is no greater pleasure than this.

\item When the thunder-cloud rumbles in the sky,\\
And the rain falls in torrents \\
On the path of birds everywhere,\\
And a monk has gone to a mountain cave, \\
Practising \emph{jhāna}:\\
There is no greater pleasure than this.

\item When sitting on a riverbank covered in flowers,\\
Garlanded with many-coloured forest plants\\
One is truly happy, practising \emph{jhāna}:\\
There is no greater pleasure than this.

\item When it is midnight in a lonely forest,\\
And the sky rains, and the lions roar,\\
And a monk has gone to a mountain cave, \\
Practising \emph{jhāna}:\\
There is no greater pleasure than this.

\item When one’s own thoughts have stopped,\\
Meditating between two mountains, \\
Sheltered inside a cleft,\\
Without stress or heartlessness, practising \emph{jhāna}:\\
There is no greater pleasure than this.

\item When one is happy, destroyer of stains, \\
Heartlessness, and sorrow,\\
Without obstructions, entanglements, and darts,\\
And with all defilements annihilated, practising \emph{jhāna}:\\
There is no greater pleasure than this.

\chapter*{Chapter Ten}
\addcontentsline{toc}{chapter}{Chapter Ten}
\markboth{Chapter Ten}{Chapter Ten}

\subsubsection*{10.1 Kāḷudāyi}

\item \vleftofline{“}The trees are now crimson, venerable sir,\\
They’ve shed their foliage, and are ready to fruit.\\
They’re splendid, as if on fire;\\
Great hero, this period is full of flavour.

\item The blossoming trees are delightful,\\
Wafting their scent all around, in all directions,\\
They’ve shed their leaves and wish to fruit,\\
Hero, it is time to depart from here.

\item It is neither too hot nor too cold,\\
Venerable sir, it’s a pleasant season for travelling.\\
Let the Sākiyas and Koḷiyas see you,\\
Facing west as you cross the Rohiṇī river.

\item In hope, the field is ploughed;\\
The seed is sown in hope;\\
In hope, merchants travel the seas,\\
Carrying rich cargoes.\\
The hope that I stand for:\\
May it succeed!

\item Again and again, they sow the seed;\\
Again and again, the king of gods sends rain;\\
Again and again, farmers plough the field;\\
Again and again, grain is produced for the nation.

\item Again and again, the beggars wander,\\
Again and again, the donors give,\\
Again and again, when the donors have given,\\
Again and again, they go to their place in heaven.

\item A hero of vast wisdom purifies seven generations\\
Of the family in which they’re born.\\
Sakya, I believe you’re the king of kings,\\
Since you fathered the one who is truly called a sage.

\item The father of the great sage is named Suddhodana;\\
But the Buddha’s mother is named Māyā.\\
Having borne the Bodhisatta in her womb,\\
She rejoices in the heaven of the Thirty-Three.

\item When she died and passed away from here,\\
She was blessed with divine sensual pleasures;\\
Rejoicing in the five kinds of sensual pleasures,\\
Gotamī is surrounded by those hosts of gods.”

\item \vleftofline{“}I’m the son of the Buddha, \\
The incomparable Aṅgīrasa, the poised—\\
I bear the unbearable.\\
You, Sakya, are my father’s father;\\
Gotama, you are my grandfather in the Dhamma.”

\subsubsection*{10.2 Ekavihāriya}

\item If no-one else is found\\
In front or behind,\\
It’s extremely pleasant,\\
Dwelling alone in a forest grove.

\item Come now, I’ll go alone\\
To the wilderness praised by the Buddha.\\
It’s pleasant for a monk\\
Dwelling alone and resolute.

\item Alone and self-disciplined,\\
I’ll quickly enter the delightful forest,\\
Which gives joy to meditators,\\
And is frequented by rutting elephants.

\item In Sītavana, so full of flowers,\\
In a cool mountain cave,\\
I’ll bathe my limbs\\
And walk meditation alone.

\item When will I dwell alone,\\
Without a companion,\\
In the great wood, so delightful,\\
My task complete, free of defilements?

\item This is what I want to do:\\
May my wish succeed!\\
I’ll make it happen myself:\\
No-one can do someone else’s duty.

\item Fastening my armour,\\
I’ll enter the forest.\\
I won’t leave here\\
Until I have attained the end of defilements.

\item As the cool breeze blows\\
With fragrant scent,\\
I’ll split ignorance apart,\\
Sitting on the mountain-peak.

\item In a forest grove covered with blossoms,\\
In a cave so very cool,\\
I take pleasure in Giribbaja,\\
Happy with the happiness of freedom.

\item My intentions are fulfilled\\
Like the moon on the fifteenth day.\\
With the utter ending of all defilements,\\
Now there is no more rebirth \\
Into any state of existence.

\subsubsection*{10.3 Mahākappina}

\item If you’re prepared for the future,\\
Both the good and the bad,\\
Then those who look for your weakness,\\
Whether enemies or well-wishers, will find none.

\item One who has fulfilled, developed,\\
And gradually consolidated\\
Mindfulness of breathing\\
As taught by the Buddha:\\
They light up the world,\\
Like the moon freed from a cloud.

\item Yes, my mind is clean,\\
Measureless, and well-developed;\\
It is broken through and uplifted—\\
It radiates in every direction.

\item The wise person lives on\\
Even after loss of wealth;\\
But without gaining wisdom\\
Even a rich person doesn’t really live.

\item Understanding questions what is learned;\\
Understanding grows fame and reputation;\\
A person who has understanding\\
Finds happiness even among sufferings.

\item It’s not something just for today;\\
It isn’t incredible or astonishing.\\
When you’re born, you die—\\
What’s astonishing about that?

\item For anyone who is born,\\
Death always follows after living.\\
Everyone who is born here dies here;\\
Such is the nature of living beings.

\item The things that are useful for the living\\
Are of no use for the dead—\\
Not fame, not celebrity,\\
Not praise by ascetics and brahmins;\\
For the dead, there is only weeping.

\item And weeping impairs the eye and the body;\\
Complexion, health, and intelligence decline.\\
Your enemies rejoice;\\
But your well-wishers are not happy.

\item So you should wish \\
That those who stay in your family\\
Have understanding and learning,\\
And do their duty \\
Through the power of understanding,\\
Just as you’d cross a full river by boat.

\subsubsection*{10.4 Cūḷapanthaka}

\item My progress was slow,\\
I was despised in the past;\\
My brother turned me away,\\
Saying, “Go home now”.

\item Turned away at the gate\\
Of the Saṅgha’s monastery,\\
I stood there sadly,\\
Longing for the teaching.

\item Then Blessed One came\\
And touched my head.\\
Taking me by the arm,\\
He brought me into the Saṅgha’s monastery.

\item The teacher, out of compassion,\\
Gave me a foot-wiping cloth, saying:\\
\vleftofline{“}Focus your awareness \\
Exclusively on this clean cloth.”

\item After I had listened to his words,\\
I dwelt delighting in his teaching,\\
Practising \emph{samādhi}\\
For the attainment of the highest goal.

\item I know my past life;\\
My clairvoyance is clarified;\\
I’ve attained the three knowledges,\\
And fulfilled the Buddha’s instructions.

\item I, Panthaka, created a thousand\\
Images of myself,\\
And sat in the delightful mango grove\\
Until the time for the meal offering was announced.

\item Then the teacher sent to me\\
A messenger to announce the time.\\
When the time was announced,\\
I flew to him through the air.

\item I paid homage to the teacher’s feet,\\
And sat to one side.\\
When he knew I was seated,\\
The teacher received the offering.

\item Recipient of gifts from the whole world,\\
Receiver of sacrifices,\\
Field of merit for humanity,\\
He received the offering.

\subsubsection*{10.5 Kappa}

\item Filled with different kinds of dirt,\\
A great producer of dung,\\
Like a stagnant cesspool,\\
A great boil, a great wound,

\item Full of pus and blood,\\
Sunk in a toilet-pit,\\
Trickling with fluids:\\
This putrid body always oozes.

\item Bound by sixty tendons,\\
Coated with a fleshy coating,\\
Clothed in a jacket of skin,\\
This putrid body is worthless.

\item Held together by a skeleton of bones,\\
And bound by sinews;\\
It assumes postures\\
Due to a complex of many things.

\item We set out in the certainty of death,\\
In the presence of the king of death;\\
And having discarded the body right here,\\
A person goes where he likes.

\item Enveloped by ignorance,\\
Tied by the four ties,\\
This body is sinking in the flood,\\
Caught in the net of underlying tendencies.

\item Yoked with the five hindrances,\\
Afflicted by thought,\\
Accompanied by the root of craving,\\
Hidden by delusion.

\item So the body goes on,\\
Propelled by the mechanism of deeds.\\
But existence ends in perishing;\\
Separated, the body perishes.

\item Those blind, unawakened people\\
Who think of this body as theirs,\\
Swell the horrors of the charnel-ground,\\
And take up rebirth again in some state of existence.

\item Those who avoid this body,\\
Like a snake smeared with dung,\\
They expel the root of rebirth,\\
And realise \emph{nibbāna}, without defilements.

\subsubsection*{10.6 Vaṅgantaputtaupasena}

\item In order to go on retreat,\\
A monk should stay in lodgings\\
That are secluded and quiet,\\
Frequented by beasts of prey.

\item Having gathered scraps from rubbish heaps,\\
Cemeteries and streets,\\
And making an outer robe from them,\\
He should wear that coarse robe.

\item Humbling his mind,\\
A monk should walk for alms\\
From family to family without exception,\\
With sense doors guarded, well-restrained .

\item He should be content even with coarse food,\\
Not hoping for lots of flavours.\\
The mind that is greedy for flavours\\
Doesn’t delight in \emph{jhāna}.

\item With few wishes, content,\\
A sage should live secluded.\\
Socializing with neither\\
Householders nor the homeless.

\item He should appear\\
To be stupid or dumb;\\
A wise person would not speak overly long\\
In the midst of the Saṅgha.

\item He would not insult anyone,\\
And would avoid causing harm.\\
Restrained in accordance with the Pātimokkha,\\
He would eat in moderation.

\item Skilled in the arising of thought,\\
He would grasp well the character of the mind.\\
He would be devoted to practicing\\
Serenity and insight at the right time.

\item Though endowed with energy and perseverance,\\
And always devoted to meditation,\\
A wise person would not be too sure of themselves,\\
Until they have attained the end of suffering.

\item For a monk who dwells in this way,\\
Longing for purification,\\
All his defilements wither away,\\
And he attains \emph{nibbāna}.

\subsubsection*{10.7 (Apara) Gotama}

\item You should understand your own purpose,\\
And consider the teachings carefully,\\
As well as what’s appropriate,\\
For one who has entered the ascetic life.

\item Good friendship in the community,\\
Undertaking lots of training,\\
Listening well to the teacher—\\
This is appropriate for an ascetic.

\item Respect for the Buddha,\\
Reverence for the Dhamma as it really is,\\
Esteem for the Saṅgha—\\
This is appropriate for an ascetic.

\item Devotion to good conduct and resort,\\
A livelihood that is pure and blameless,\\
And settling the mind—\\
This is appropriate for an ascetic.

\item A pleasing manner in things that should be done,\\
And those that should be avoided;\\
Devotion to the higher mind—\\
This is appropriate for an ascetic.

\item Wilderness lodgings\\
Remote, with little noise,\\
Fit for use by a sage—\\
This is appropriate for an ascetic.

\item Ethics, learning,\\
Investigation of Dhamma as it really is,\\
And penetration of the truths—\\
This is appropriate for an ascetic.

\item Developing the perceptions\\
Of impermanence, non-self, and unattractiveness,\\
And displeasure with the whole world—\\
This is appropriate for an ascetic.

\item Developing the factors of awakening,\\
The bases for psychic power, \\
The spiritual faculties and powers,\\
And the noble eight-fold path—\\
This is appropriate for an ascetic.

\item A sage should abandon craving,\\
With defilements split apart, root and all,\\
They should live liberated—\\
This is appropriate for an ascetic.

\chapter*{Chapter Eleven}
\addcontentsline{toc}{chapter}{Chapter Eleven}
\markboth{Chapter Eleven}{Chapter Eleven}

\subsubsection*{11.1 Saṅkicca}

\item \vleftofline{“}Like an \emph{ujjuhāna}-bird in the rainy season,\\
Child, is there benefit for you in the grove?\\
The city of Verambhā is delightful for you—\\
Seclusion is for meditators.”

\item \vleftofline{“}Just as the wind in Verambhā\\
Scatters the clouds during the rainy-season,\\
So the city scatters\\
My perceptions connected with seclusion.

\item It’s all black and born of an egg—\\
The crow that lives in the charnel ground\\
Rouses my mindfulness,\\
Based on dispassion for the body.

\item Not protected by others,\\
Nor protecting others:\\
Such a monk sleeps happily,\\
Without longing for sensual pleasures.

\item The water is clear and the gorges are wide,\\
Monkeys and deer are all around;\\
Festooned with dewy moss,\\
These rocky crags delight me!

\item I’ve dwelt in the wilderness,\\
In caves and caverns,\\
And remote lodgings,\\
Frequented by beasts of prey.

\item \vleftofline{‘}May these beings be killed! \\
May they be slaughtered!\\
May they suffer!’—\\
I’m not aware of having any such\\
Ignoble, hateful intentions.

\item I’ve attended on the teacher\\
And fulfilled the Buddha’s instructions.\\
The heavy burden is laid down,\\
I’ve undone the attachment \\
To being reborn in any state of existence.

\item I’ve attained the goal\\
For the sake of which I went forth\\
From home life into homelessness—\\
The ending of all fetters.

\item I don’t long for death;\\
I don’t long for life;\\
I await my time,\\
Like a worker waiting for their wages.

\item I don’t long for death;\\
I don’t long for life;\\
I await my time,\\
Aware and mindful.”

\chapter*{Chapter Twelve}
\addcontentsline{toc}{chapter}{Chapter Twelve}
\markboth{Chapter Twelve}{Chapter Twelve}

\subsubsection*{12.1 Sīlava}

\item One should train just in virtue,\\
For in this world, when virtue is\\
Cultivated and well-trained,\\
It provides all success.

\item Desiring three kinds of happiness—\\
Praise, prosperity,\\
And to delight in heaven after passing away—\\
The wise should protect virtue.

\item The well-behaved have many friends,\\
Because of their self-restraint.\\
But one without virtue, of bad conduct,\\
Drives away their friends.

\item A person of bad behavior has\\
Ill-repute and infamy.\\
A person of virtue always has\\
A good reputation, fame, and praise.

\item Virtue is the starting point and foundation;\\
The mother at the head\\
Of all good qualities:\\
Therefore you should purify virtue.

\item Virtue is a boundary and a restraint,\\
An enjoyment for the mind;\\
The place where all the Buddhas cross over:\\
Therefore you should purify virtue.

\item Virtue is the matchless power;\\
Virtue is the ultimate weapon;\\
Virtue is the best ornament;\\
Virtue is a marvellous coat of armour.

\item Virtue is a mighty bridge;\\
Virtue is the unsurpassed scent;\\
Virtue is the best perfume,\\
That floats in all directions.

\item Virtue is the best provision;\\
Virtue is the unsurpassed supply for a journey;\\
Virtue is the best vehicle,\\
That takes you in all directions.

\item In this life they’re criticized;\\
After passing away they’re unhappy in a lower realm;\\
A fool is unhappy everywhere,\\
Because they are not endowed with virtues.

\item In this life they’re famous;\\
After passing away they’re happy in heaven;\\
A person with understanding is happy everywhere,\\
Because they are endowed with virtues.

\item Virtue is best in this life,\\
But person with understanding is supreme\\
Among humans and gods,\\
Conquering with virtue and understanding.

\subsubsection*{12.2 Sunīta}

\item I was born in a low-class family,\\
Poor, with little to eat.\\
My job was lowly—\\
I threw out the old flowers.

\item Shunned by people,\\
I was disregarded and treated with contempt.\\
I humbled my heart,\\
And paid respects to many people.

\item Then I saw the Buddha,\\
Honoured by the Saṅgha of monks,\\
The great hero,\\
Entering the capital city of Magadhā.

\item I dropped my carrying-pole\\
And approached to pay respects.\\
Out of compassion for me,\\
The supreme man stood still.

\item When I had paid respects at the teacher’s feet,\\
I stood to one side,\\
And asked the most excellent of all beings\\
For the going-forth.

\item Then the teacher, being sympathetic,\\
And having compassion for the whole world,\\
Said to me, “Come, monk!”\\
That was my full ordination.

\item Staying alone in the wilderness,\\
Without laziness,\\
I did what the teacher said,\\
As the conqueror had advised me.

\item In the first watch of the night,\\
I recollected my previous births.\\
In the middle watch of the night,\\
I purified my clairvoyance.\\
In the last watch of the night,\\
I tore apart the mass of darkness.

\item At the end of the night,\\
As the sunrise drew near,\\
Indra and Brahmā came\\
And paid homage me with hands in \emph{añjalī}.

\item \vleftofline{“}Homage to you, thoroughbred among men!\\
Homage to you, supreme among men!\\
Your defilements are ended—\\
You, sir, are worthy of offerings.”

\item When he saw me honored\\
By the assembly of gods,\\
The teacher smiled,\\
And said the following:

\item \vleftofline{“}By austerity and by the holy life,\\
By restraint and by taming:\\
By this one is a holy man,\\
This is the supreme holiness.”

\chapter*{Chapter Thirteen}
\addcontentsline{toc}{chapter}{Chapter Thirteen}
\markboth{Chapter Thirteen}{Chapter Thirteen}

\subsubsection*{13.1 Soṇakoḷivisa}

\item He who was special in the kingdom,\\
The attendant to the king of Aṅga,\\
Today is special in the Dhamma—\\
Soṇa has gone beyond suffering.

\item Five should be cut off, five should be abandoned,\\
Five should be developed further.\\
A monk who has gone beyond \\
Five attachments is called\\
\vleftofline{“}One who has crossed the flood.”

\item If a monk is insolent and negligent,\\
Concerned only with externals,\\
Their virtue, \emph{samādhi}, and understanding\\
Do not become fulfilled.

\item They disregard what should be done,\\
And do what shouldn’t be done.\\
For the insolent and the negligent,\\
Their defilements only grow.

\item Those that have properly undertaken\\
Constant mindfulness of the body,\\
Don’t practise what shouldn’t be done,\\
But consistently do what should be done.\\
Mindful and clearly aware,\\
Their defilements come to an end.

\item Go on the straight path that has been taught—\\
Don’t turn back.\\
Urge yourself on,\\
And realise \emph{nibbāna}.

\item When my energy was over-exerted,\\
The unsurpassed teacher in the world,\\
Made the simile of the lute for me;\\
The Seer taught the Dhamma,\\
And when I heard what he said,\\
I stayed joyfully in his teaching.

\item Practising serenity of mind,\\
I attained the supreme goal.\\
I’ve attained the three knowledges,\\
And fulfilled the Buddha’s instructions.

\item Committed to renunciation,\\
And seclusion of the heart,\\
Committed to non-harming,\\
And the end of grasping;

\item Committed to the end of craving,\\
And an unconfused heart;\\
When seeing the senses arise,\\
The mind is perfectly liberated.

\item For the monk who is perfectly liberated,\\
His mind at peace,\\
There’s nothing to add to what has been done;\\
And nothing further to be done.

\item Just as a solid rock\\
Is not moved by the wind,\\
So sights, tastes, sounds\\
Smells, touches, all of these,

\item As well as pleasant and unpleasant phenomena,\\
Don’t shake one who is poised,\\
Whose mind is firm and unfettered,\\
Contemplating vanishing.

\chapter*{Chapter Fourteen}
\addcontentsline{toc}{chapter}{Chapter Fourteen}
\markboth{Chapter Fourteen}{Chapter Fourteen}

\subsubsection*{14.1 Khadiravaniyarevata}

\item Since I’ve gone forth\\
From home life into homelessness,\\
I’m not aware of any intention\\
That is ignoble and hateful.

\item \vleftofline{“}May these beings be killed! \\
May they be slaughtered!\\
May they suffer!”—\\
I’m not aware of having any such intentions\\
In all this long period of time.

\item I have been aware of loving-kindness,\\
Measureless and well-developed,\\
Gradually built up,\\
Just as the Buddha taught.

\item I’m friend and comrade to all,\\
Compassionate to all beings,\\
Developing a mind of loving-kindness,\\
And always delighting in harmlessness.

\item Immovable, unshakable,\\
I gladden the mind.\\
I develop the sublime abidings,\\
Which bad men do not cultivate.

\item Having entered a meditation state without thought,\\
A disciple of the Buddha\\
Is at that moment blessed\\
With noble silence.

\item Just like a rocky mountain\\
Is unshakable and firmly grounded;\\
So when delusion ends,\\
A monk, like a mountain, doesn’t tremble.

\item To the blameless man\\
Who is always seeking purity,\\
Even a hair-tip of evil\\
Seems the size of a cloud.

\item Just like a frontier city,\\
Is guarded inside and out,\\
So you should ward yourselves—\\
Don’t let the moment pass you by.

\item I don’t long for death;\\
I don’t long for life;\\
I await my time,\\
Like a worker waiting for their wages.

\item I don’t long for death;\\
I don’t long for life;\\
I await my time,\\
Aware and mindful.

\item I’ve attended on the teacher\\
And fulfilled the Buddha’s instructions.\\
The heavy burden is laid down,\\
I’ve undone the attachment to being reborn \\
In any state of existence.

\item I’ve attained the goal\\
For the sake of which I went forth\\
From home life into homelessness—\\
The ending of all fetters.

\item Strive on with heedfulness:\\
This is my advice.\\
Come, I’ll realise \emph{nibbāna}—\\
I’m liberated in every way.

\subsubsection*{14.2 Godatta}

\item Just as a fine thoroughbred,\\
Yoked to a carriage, endures the load,\\
Oppressed by the heavy burden,\\
And yet doesn’t try to escape the yoke;

\item So too, those who are as filled with understanding\\
As the ocean is with water,\\
Don’t look down on others;\\
This is the noble Dhamma regarding living beings.

\item People who fall under the dominion of time,\\
Under the dominion of being reborn \\
In one state of existence after another,\\
Undergo suffering,\\
And those young men grieve in this life.

\item Elated by anything happy,\\
Downcast by anything suffering:\\
These both destroy the fool,\\
Who doesn’t see in accordance with reality.

\item But those who in suffering and in happiness,\\
And in the middle have overcome the weaver;\\
They stand like a royal pillar,\\
Neither elated nor downcast.

\item Not to gain or loss,\\
Not to fame or reputation,\\
Not to criticism or praise,\\
Not to suffering or happiness—

\item The wise cling to nothing,\\
Like a droplet on a lotus-leaf.\\
They are happy everywhere,\\
And unconquered everywhere.

\item There’s principled loss,\\
And there’s unprincipled gain.\\
Principled loss is better\\
Than unprincipled gain.

\item There’s the fame of the unintelligent,\\
And there’s the disrepute of the discerning.\\
Disrepute of the discerning is better\\
Than the fame of the unintelligent.

\item There’s praise by fools,\\
And there’s criticism by the discerning.\\
Criticism by the discerning is better\\
Than praise by fools.

\item There’s the happiness of sensual pleasures,\\
And there’s the suffering of seclusion.\\
The suffering of seclusion is better\\
Than the happiness of sensual pleasures.

\item There’s life without principles,\\
And there’s death with principles.\\
Death with principles is better\\
Than life without principles.

\item Those who have abandoned \\
Sensual pleasures and anger,\\
Their minds at peace regarding being reborn \\
In one state of existence or another,\\
They wander in the world unattached,\\
For them nothing is beloved or unloved.

\item Having developed the factors of awakening,\\
The spiritual faculties, and the powers,\\
I’ve attained ultimate peace:\\
\emph{Nibbāna} without defilements.

\chapter*{Chapter Fifteen}
\addcontentsline{toc}{chapter}{Chapter Fifteen}
\markboth{Chapter Fifteen}{Chapter Fifteen}

\subsubsection*{15.1 Aññāsikoṇḍañña}

\item \vleftofline{“}My confidence grew\\
As I heard the Dhamma, so full of flavor.\\
Dispassion was the Dhamma that was taught,\\
Without any grasping at all.”

\item \vleftofline{“}There are many pretty things\\
In the circle of this earth;\\
They disturb one’s thoughts, I believe,\\
Beautiful, provoking lust.

\item Just as a rain cloud would settle\\
The dust blown up by the wind;\\
So thoughts settle down\\
When seen with understanding.

\item All conditions are impermanent—\\
When this is seen with understanding,\\
One turns away from suffering:\\
This is the path to purity.

\item All conditions are suffering—\\
When this is seen with understanding,\\
One turns away from suffering:\\
This is the path to purity.

\item All phenomena are not-self—\\
When this is seen with understanding,\\
One turns away from suffering:\\
This is the path to purity.

\item The senior monk Koṇḍañña, who was awakened\\
Right after the Buddha, is keenly energetic.\\
He has abandoned birth and death,\\
And has perfected the spiritual life.

\item There are floods, snares, and strong posts,\\
And a mountain hard to crack;\\
Snapping the posts and snares,\\
Breaking the mountain so hard to break,\\
Crossing over to the far shore,\\
One practicing \emph{jhāna} is freed from Māra’s bonds.

\item A haughty and fickle monk,\\
Relying on bad friends,\\
Sinks down in the great flood,\\
Overcome by a wave.

\item But one who is humble and stable,\\
Controlled, with senses restrained,\\
Wise, with good friends,\\
Would put an end to suffering.

\item With knobbly knees,\\
Thin, with veins matted on his skin,\\
Eating and drinking in moderation—\\
This person’s spirit is undaunted.

\item Pestered by gadflies and mosquitoes\\
In the awesome wilderness;\\
One should mindfully endure,\\
Like an elephant at the head of the battle.

\item I don’t long for death;\\
I don’t long for life;\\
I await my time,\\
Like a worker waiting for their wages.

\item I don’t long for death;\\
I don’t long for life;\\
I await my time,\\
Aware and mindful.

\item I’ve attended on the teacher\\
And fulfilled the Buddha’s instructions.\\
The heavy burden is laid down,\\
I’ve undone the attachment to being reborn \\
In any state of existence.

\item I’ve attained the goal\\
For the sake of which I went forth\\
From home life into homelessness—\\
What use do I have for students?”

\subsubsection*{15.2 Udāyi}

\item An person who has become \\
Awakened as a human being,\\
Self-tamed, with \emph{samādhi},\\
Following the spiritual path,\\
Delights in peace of mind.

\item Revered by people,\\
Gone beyond all things,\\
Even the gods revere him;\\
So I’ve heard from the Arahant.

\item He has transcended all fetters,\\
And escaped from entanglements,\\
Delighting in the renunciation of sensual pleasures,\\
He is liberated like gold from stone.

\item That elephant outshines all,\\
As the Himālaya outshines other mountains.\\
Of all those named “elephant”,\\
He is truly named, and unsurpassed.

\item I’ll extol the elephant to you,\\
For he does nothing wrong.\\
The elephant’s front two feet \\
Are gentleness and harmlessness.

\item Mindfulness and awareness\\
Are the elephant’s other feet.\\
Faith is the great elephant’s trunk,\\
And equanimity is the white tusks.

\item Mindfulness is his neck, his head is understanding—\\
The investigation and reflection on phenomena—\\
His belly is the sacred hearth of the Dhamma,\\
His tail is seclusion.

\item Practicing \emph{jhāna}, delighting in the breath,\\
Serene inside himself.\\
The elephant is serene when walking,\\
The elephant is serene when standing,

\item The elephant is serene when lying down,\\
And when sitting, the elephant is serene.\\
The elephant is restrained everywhere:\\
This is the accomplishment of the elephant.

\item He eats blameless things,\\
He doesn’t eat blameworthy things.\\
When he gets food and clothes,\\
He avoids storing them up.

\item Having cut off all bonds,\\
Fetters large and small,\\
Wherever he goes,\\
He goes without longing

\item Just as a white lotus,\\
Fragrant and delightful,\\
Is born in water and grows there,\\
But the water does not stick to it;

\item So the Buddha is born in the world,\\
And lives in the world,\\
But the world does not stick to him,\\
As the water does not stick to the lotus.

\item A great blazing fire\\
Dies down when the fuel runs out;\\
When the coals have gone out\\
It’s said to be “quenched”.

\item This simile is taught by the discerning\\
To express the meaning clearly.\\
Great elephants will understand\\
What the elephant taught the elephant.

\item Free of desire, free of hatred,\\
Free of delusion, without defilements,\\
The elephant, abandoning their body,\\
Realises \emph{nibbāna} without defilements.

\chapter*{Chapter Sixteen}
\addcontentsline{toc}{chapter}{Chapter Sixteen}
\markboth{Chapter Sixteen}{Chapter Sixteen}

\subsubsection*{16.1 Adhimutta}

\item \vleftofline{“}Those that we previously killed,\\
Whether for sacrifice or for wealth,\\
Without exception were afraid:\\
They trembled and squealed.

\item But you aren’t frightened;\\
Your appearance is becoming more calm:\\
Why don’t you cry out\\
In such a fearful situation?”

\item \vleftofline{“}There isn’t any mental suffering\\
For one without expectations, village chief.\\
All fears are left behind\\
By one whose fetters are ended.

\item When attachment to life is ended,\\
In this very life as it is,\\
There is no fear of death,\\
It is just like laying down a burden.

\item I’ve lived the spiritual life well,\\
And developed the path well, too;\\
I have no fear of death\\
It is just like the ending of sickness.

\item I’ve lived the spiritual life well,\\
And developed the path well, too;\\
I’ve seen lives seen to be ungratifying,\\
Like one who has drunk poison, then vomited it out.

\item One who has gone beyond, without grasping,\\
Their duty completed, without defilements:\\
They are content at the end of life,\\
Just as one freed from execution.

\item Having realised the supreme Dhamma,\\
Without needing anything from the whole world,\\
One doesn’t grieve at death;\\
It is just like escaping from a burning house.

\item Whatever has come to pass,\\
Wherever life is obtained,\\
There is no-one who can wield power over all that:\\
So it was said by the great sage.

\item Whoever understands this\\
As it was taught by the Buddha\\
Doesn’t take hold of any kind of life,\\
It is just like grabbing a hot iron ball.

\item It doesn’t occur to me, ‘I had past lives’;\\
Nor does it occur to me, ‘I will have future lives’.\\
All conditions will disappear—\\
Why lament for that?

\item Seeing in accordance with reality\\
The bare arising of phenomena,\\
And the bare continuity of conditions,\\
There is no fear, village chief.

\item The world is like grass and wood:\\
When this is seen with understanding,\\
Not finding anything to be mine,\\
Thinking ‘it isn’t mine’, one doesn’t grieve.

\item I’m fed up with the body;\\
I don’t need another life.\\
This body will be broken up,\\
There won’t be another.

\item Do what you want\\
With my corpse.\\
I won’t be angry or attached\\
On that account.”

\item When they heard these words,\\
So astonishing that they gave them goose-bumps,\\
The young men laid down their swords\\
And said this:

\item \vleftofline{“}What have you practiced, Venerable?\\
Or who is your teacher?\\
Whose instructions do we follow\\
To gain the sorrowless state?”

\item \vleftofline{“}All-knowing, all-seeing,\\
The conqueror is my teacher.\\
He is a teacher of great compassion,\\
Healer of the whole world.

\item He taught this Dhamma,\\
Which leads to the end, unsurpassed.\\
Following his instructions,\\
You can gain the sorrowless state.”

\item When the bandits heard the good words of the sage,\\
They laid down their swords and weapons.\\
Some refrained from their deeds,\\
While others chose the going-forth.

\item When they had gone forth \\
In the teaching of the Fortunate One,\\
They developed the factors of awakening \\
And the spiritual powers,\\
And being wise, with joyful hearts, happy, \\
Their spiritual faculties complete,\\
They realised the state of \emph{nibbāna}, the unconditioned.

\subsubsection*{16.2 Pārāpariya}

\item While he was sitting alone\\
In seclusion, practicing \emph{jhāna},\\
An ascetic, the monk Pārāpariya\\
Had this thought:

\item \vleftofline{“}Following what system\\
What vow, what conduct,\\
May I do what I need to do for myself,\\
Without harming anyone else?

\item The faculties of human beings\\
Can lead to both welfare and harm.\\
Unguarded they lead to harm;\\
Guarded they lead to welfare.

\item By protecting the faculties,\\
Taking care of the faculties,\\
I can do what I need to do for myself\\
Without harming anyone else.

\item If your eye wanders\\
Among sights without check,\\
Not seeing the danger,\\
You’re not freed from suffering.

\item If your ear wanders\\
Among sounds without check,\\
Not seeing the danger,\\
You’re not freed from suffering.

\item If, not seeing the escape,\\
You indulge in smell,\\
You’re not freed from suffering,\\
Being infatuated by smells.

\item Recollecting the sour,\\
And the sweet and the bitter,\\
Captivated by craving for taste,\\
You don’t understand the heart.

\item Recollecting lovely\\
And pleasurable touches,\\
Full of desire, you experience\\
Many kinds of suffering because of lust.

\item Unable to protect\\
The mind from such mental phenomena,\\
Suffering follows them,\\
Because of all five.

\item This body is full of pus and blood,\\
As well as many carcasses;\\
But cunning people decorate it\\
Like a lovely painted casket.

\item You don’t understand that\\
The gratification of sweetness turns out bitter,\\
And attachments to those we love are suffering,\\
Like a razor smeared all over with honey.

\item Full of lust for the sight of a woman,\\
For the voice and the smells of a woman,\\
For a woman’s touch,\\
You experience many kinds of suffering.

\item All of a woman’s streams\\
Flow from five to five.\\
Whoever, being energetic,\\
Is able to curb these,

\item Purposeful and firm in Dhamma,\\
Would be clever and discerning;\\
Even while enjoying himself,\\
What he does is connected \\
With Dhamma and its purpose.

\item You should avoid a meaningless task\\
That is leading to decline.\\
Thinking, ‘This is not to be done’,\\
Is being diligent and discerning.

\item Whatever is meaningful,\\
A principled happiness,\\
Let one undertake and practice that:\\
This is the best happiness.

\item Coveting the possessions of others\\
By whatever means, whether high or low,\\
One kills, injures, and torments,\\
Violently plundering the possessions of others.

\item Just as a strong person when building\\
Knocks out a peg with a peg,\\
So the skilful person\\
Knocks out the faculties with the faculties.

\item Developing faith, energy, \emph{samādhi}\\
Mindfulness, and wisdom;\\
Destroying the five with the five,\\
The perfected one lives without worry.

\item Purposeful and firm in Dhamma,\\
Having fulfilled in every respect\\
The instructions spoken by the Buddha,\\
That person prospers in happiness.”

\subsubsection*{16.3 Telakāni}

\item For a long time, unfortunately,\\
Though I ardently contemplated the Dhamma,\\
I didn’t have peace of mind;\\
So I asked ascetics and holy men:

\item \vleftofline{“}Who has crossed over the world?\\
Whose attainment culminates in the deathless?\\
Whose teaching do I accept,\\
To understand the highest goal?

\item I was hooked inside,\\
Like a fish swallowing bait;\\
Bound like the demon Vepaciti\\
In Mahinda’s trap.

\item Dragging it along, I’m not freed\\
From grief and lamentation.\\
Who will free me from bonds in the world,\\
So that I may know awakening?

\item What ascetic or holy man\\
Points to the perishable?\\
Whose teaching do I accept\\
To wash away old age and death?

\item Tied up with uncertainty and doubt,\\
Secured by the power of pride,\\
Rigid as a mind overcome by anger;\\
The arrow of covetousness,

\item Propelled by the bow of craving,\\
Is stuck in my twice-fifteen ribs—\\
See how it stands in my breast,\\
Breaking my strong heart.

\item Speculative views are not abandoned,\\
They are sharpened by memories and intentions;\\
And pierced by this I tremble,\\
Like a leaf blown by wind.

\item Arising inside me,\\
My selfishness is quickly tormented,\\
Where the body always goes\\
With its six sense-fields of contact.

\item I don’t see a healer\\
Who could pull out my dart of doubt,\\
Without a lance\\
Or some other blade.

\item Without knife or wound,\\
Who will pull out this dart,\\
That is stuck inside me,\\
Without harming any part of my body?

\item He really would be the Lord of the Dhamma,\\
The best one to cure the damage of poison;\\
When I had fallen into deep waters,\\
He would give me his hand and bring me to the shore.

\item I’ve plunged into a lake,\\
And I can’t wash off the mud and dirt,\\
It’s full of fraud, jealousy, pride,\\
And dullness and drowsiness.

\item Like a thunder-cloud of restlessness,\\
Like a rain-cloud of fetters;\\
Intentions based on lust are winds\\
That sweep along a person with bad views.

\item The streams flow everywhere;\\
A weed springs up and remains;\\
Who will block the streams?\\
Who will cut the weed?”

\item \vleftofline{“}Venerable sir, build a dam\\
To block the streams;\\
Don’t let your mind-made streams\\
Cut you down suddenly like a tree.”

\item That is how the teacher whose weapon is wisdom,\\
The sage surrounded by the Saṅgha,\\
Was my shelter when I was full of fear,\\
Seeking the far shore from the near.

\item As I was being swept away,\\
He gave me a strong, simple ladder,\\
Made of the heartwood of Dhamma,\\
And he said to me: “Do not fear.”

\item I climbed the tower \\
Of the establishment of mindfulness\\
And looked back down,\\
At people delighting in identity,\\
Which in the past I’d obsessed over.

\item When I saw the path,\\
As I was embarking on the ship,\\
Without fixating on the self,\\
I saw the supreme landing-place.

\item The dart that arises in oneself,\\
And that which is caused \\
By attachment to future lives;\\
He taught the supreme path\\
For the stopping of these.

\item For a long time it had lain within me;\\
For a long time it was fixed in me:\\
The Buddha cast off the knot,\\
Curing the poison’s damage.

\subsubsection*{16.4 Raṭṭhapāla}

\item See this fancy puppet,\\
A heap of sores, a composite body,\\
Diseased, obsessed over,\\
Having no lasting stability.

\item See this fancy shape,\\
With its gems and earrings;\\
It is bones wrapped with skin,\\
Made pretty by its clothes.

\item Rouged feet\\
And powdered face\\
Is enough to delude a fool,\\
But not a seeker of the far shore.

\item Hair in eight braids\\
And eyeliner,\\
Is enough to delude a fool,\\
But not a seeker of the far shore.

\item Like a newly decorated makeup box,\\
The disgusting body all adorned\\
 Is enough to delude a fool,\\
But not a seeker of the far shore.

\item The hunter laid his trap,\\
But the deer didn’t get caught in the snare;\\
Having eaten the bait we go,\\
Leaving the deer-trapper to lament.

\item The hunter’s trap is broken,\\
And the deer didn’t get caught in the snare;\\
Having eaten the bait we go,\\
Leaving the deer-trapper to lament.

\item I see rich people in the world,\\
Who, because of delusion, \\
Don’t give away the wealth they have gained.\\
Greedily, they hoard their riches,\\
Yearning for ever more sensual pleasures.

\item A king who conquered the earth by force,\\
Ruling the land from sea to sea,\\
Unsatisfied with the near shore of the ocean,\\
Would still yearn for the further shore.

\item The king and most other people\\
Reach death while not free from craving.\\
As if lacking, they abandon the body;\\
For sensual pleasures offer \\
No satisfaction in this world.

\item Relatives lament, their hair let loose,\\
Saying “Ah! Alas! They’re not immortal!”\\
They take out the body wrapped in a shroud,\\
Heap up a pyre, and burn it.

\item It is poked with stakes while being burnt,\\
Wearing a single cloth, all wealth abandoned.\\
Neither kinsman nor friends nor companions\\
Can help you when you are dying.

\item Heirs take the riches,\\
But beings fare on in accord with their deeds.\\
Riches don’t follow you when you die;\\
Nor do children, wife, wealth, nor kingdom.

\item Longevity isn’t gained by riches,\\
Nor does wealth banish old age;\\
For the wise have said that this life is short,\\
It is not eternal, its nature is decay.

\item The rich and the poor feel its touch;\\
The fool and the wise feel it too;\\
But the fool lies as if struck down by their own folly,\\
While the wise don’t tremble at the touch.

\item Therefore wisdom is definitely better than wealth,\\
Since by wisdom \\
You can attain perfection in this life;\\
But if you stay unperfected, \\
Then because of delusion,\\
You’ll do evil deeds in life after life.

\item One person enters a womb and the world beyond,\\
Transmigrating from one life to the next;\\
While someone of little wisdom, placing faith in them,\\
Also enters a womb and the world beyond.

\item Just as a bandit caught at the entrance to a house\\
Is punished due to their own bad deeds;\\
So after passing away, in the world beyond\\
People are punished due to their own bad deeds.

\item Sensual pleasures are diverse, sweet, delightful,\\
But their variety of forms stress the mind;\\
Seeing danger in the kinds of sensual pleasure,\\
I went forth, O King.

\item As fruit falls from a tree, so people fall,\\
Young and old, when the body breaks up.\\
Seeing this, too, I went forth, O King;\\
Without doubt, the ascetic life is better.

\item Endowed with faith, I went forth,\\
Entering the conqueror’s teaching.\\
My going forth wasn’t wasted;\\
I eat food free of debt.

\item I saw sensual pleasures as burning,\\
Gold as a cutting blade,\\
Conception in a womb as suffering,\\
And the hells as very fearful.

\item Knowing this danger,\\
I was struck with awe.\\
I was stabbed, and then I became peaceful;\\
I’ve attained the end of defilements.

\item I’ve attended on the teacher\\
And fulfilled the Buddha’s instructions.\\
The heavy burden is laid down,\\
I’ve undone the attachment to being reborn \\
In any state of existence.

\item I’ve attained the goal\\
For the sake of which I went forth\\
From home life into homelessness—\\
The ending of all fetters.

\subsubsection*{16.5 Mālukyaputta}

\item When seeing a sight, \\
Mindfulness becomes confused,\\
If attention is focussed on the pleasant aspect.\\
Experiencing it with a mind full of desire,\\
One remains clinging to it.

\item Many feelings grow\\
Arising from sights.\\
The mind is damaged\\
By covetousness and cruelty.\\
Heaping up suffering like this,\\
Is said to be far from \emph{nibbāna}.

\item When hearing a sound, \\
Mindfulness becomes confused,\\
If attention is focussed on the pleasant aspect.\\
Experiencing it with a mind full of desire,\\
One remains clinging to it.

\item Many feelings grow\\
Arising from sounds.\\
The mind is damaged\\
By covetousness and cruelty.\\
Heaping up suffering like this,\\
Is said to be far from \emph{nibbāna}.

\item When smelling a smell, \\
Mindfulness becomes confused,\\
If attention is focussed on the pleasant aspect.\\
Experiencing it with a mind full of desire,\\
One remains clinging to it.

\item Many feelings grow\\
Arising from smells.\\
The mind is damaged\\
By covetousness and cruelty.\\
Heaping up suffering like this,\\
Is said to be far from \emph{nibbāna}.

\item When savouring a taste, \\
Mindfulness becomes confused,\\
If attention is focussed on the pleasant aspect.\\
Experiencing it with a mind full of desire,\\
One remains clinging to it.

\item Many feelings grow\\
Arising from tastes.\\
The mind is damaged\\
By covetousness and cruelty.\\
Heaping up suffering like this,\\
Is said to be far from \emph{nibbāna}.

\item When touching a touch, \\
Mindfulness becomes confused,\\
If attention is focussed on the pleasant aspect.\\
Experiencing it with a mind full of desire,\\
One remains clinging to it.

\item Many feelings grow\\
Arising from touches.\\
The mind is damaged\\
By covetousness and cruelty.\\
Heaping up suffering like this,\\
Is said to be far from \emph{nibbāna}.

\item When knowing a mental phenomenon, \\
Mindfulness becomes confused,\\
If attention is focussed on the pleasant aspect.\\
Experiencing it with a mind full of desire,\\
One remains clinging to it.

\item Many feelings grow\\
Arising from mental phenomena.\\
The mind is damaged\\
By covetousness and cruelty.\\
Heaping up suffering like this,\\
Is said to be far from \emph{nibbāna}.

\item Seeing a sight with mindfulness,\\
There is no desire for sights.\\
Experiencing it with a mind free of desire,\\
One doesn’t remain clinging to it.

\item As it is for someone who lives mindfully,\\
When repeatedly seeing a sight,\\
Feeling is ended, not added to.\\
Reducing suffering like this,\\
Is said to be in the presence of \emph{nibbāna}.

\item Hearing a sound with mindfulness,\\
There is no desire for sounds.\\
Experiencing it with a mind free of desire,\\
One doesn’t remain clinging to it.

\item As it is for someone who lives mindfully,\\
When repeatedly hearing a sound,\\
Feeling is ended, not added to.\\
Reducing suffering like this,\\
Is said to be in the presence of \emph{nibbāna}.

\item Smelling a smell with mindfulness,\\
There is no desire for smells.\\
Experiencing it with a mind free of desire,\\
One doesn’t remain clinging to it.

\item As it is for someone who lives mindfully,\\
When repeatedly smelling a smell,\\
Feeling is ended, not added to.\\
Reducing suffering like this,\\
Is said to be in the presence of \emph{nibbāna}.

\item Savouring a taste with mindfulness,\\
There is no desire for tastes.\\
Experiencing it with a mind free of desire,\\
One doesn’t remain clinging to it.

\item As it is for someone who lives mindfully,\\
When repeatedly savouring a taste,\\
Feeling is ended, not added to.\\
Reducing suffering like this,\\
Is said to be in the presence of \emph{nibbāna}.

\item Touching a touch with mindfulness,\\
There is no desire for touches.\\
Experiencing it with a mind free of desire,\\
One doesn’t remain clinging to it.

\item As it is for someone who lives mindfully,\\
When repeatedly touching a touch,\\
Feeling is ended, not added to.\\
Reducing suffering like this,\\
Is said to be in the presence of \emph{nibbāna}.

\item Knowing a mental phenomenon with mindfulness,\\
There is no desire for mental phenomena.\\
Experiencing it with a mind free of desire,\\
One doesn’t remain clinging to it.

\item As it is for someone who lives mindfully,\\
When repeatedly knowing a mental phenomenon,\\
Feeling is ended, not added to.\\
Reducing suffering like this,\\
Is said to be in the presence of \emph{nibbāna}.

\subsubsection*{16.6 Sela}

\item \vleftofline{“}Your body is perfect, you are radiant,\\
Handsome, lovely to behold,\\
Blessed One, you are golden coloured,\\
Your teeth are pure white, you are full of energy.

\item The characteristics\\
Of a handsome man,\\
The marks of a great man,\\
Are all in your body.

\item Your eyes are clear, your face is nice,\\
You are large, upright, and majestic.\\
In the middle of the Saṅgha of ascetics,\\
You shine like the sun.

\item You’re a good-looking monk,\\
With skin like gold;\\
With such excellent appearance,\\
What do you want with the ascetic life?

\item You’re worthy of being a king,\\
A wheel-rolling emperor, a bull among heroes,\\
Victorious in the four directions,\\
Lord of all India.

\item Warriors, lords, and kings\\
Are your followers\\
You are king above kings and lord of men—\\
Claim your kingship, Gotama!”

\item \vleftofline{“}Sela, I \emph{am} a king,\\
Said the Blessed One to Sela,\\
\vleftofline{“}The unsurpassed king of Dhamma.\\
By Dhamma I set the wheel rolling,\\
The wheel which cannot be rolled back.”

\item \vleftofline{“}You claim to be awakened,”\\
Said Sela the brahman,\\
\vleftofline{“}The unsurpassed king of Dhamma.\\
\vleftofline{‘}By Dhamma I set the wheel rolling,’\\
That is what you say, Gotama.

\item Who is the Blessed One’s general,\\
The disciple who follows the teacher?\\
Who keeps on rolling\\
The wheel of Dhama you rolled forth?”

\item \vleftofline{“}I rolled forth the wheel,”\\
Said the Blessed One to Sela,\\
\vleftofline{“}The unexcelled wheel of Dhamma.\\
Sāriputta, who follows the Tathāgata’s example,\\
Keeps it rolling on.

\item What’s to be known is known;\\
What’s to be developed is developed;\\
I’ve abandoned what’s to be abandoned;\\
Therefore, brahmin, I am a Buddha.

\item Dispel your doubt in me;\\
Make up your mind, brahman!\\
It’s always hard to gain\\
The sight of Buddhas.

\item I am one of those whose appearance\\
Is always hard to find in this world;\\
I am a Buddha, brahman,\\
The unexcelled remover of darts.

\item Holy, unequalled,\\
Crusher of Māra’s army;\\
Having subdued all enemies,\\
I rejoice, fearing nothing in any direction.”

\item \vleftofline{“}Listen, sirs, to what,\\
Is spoken by the seer.\\
Remover of darts, great hero,\\
Roaring like a lion in the jungle.

\item Holy, unequalled,\\
Crusher of Māra’s army;\\
Who could see him and not have faith,\\
Even one whose nature is dark?

\item Those who wish may follow me;\\
Those who don’t wish may go.\\
Right here, I’ll go forth,\\
In the presence of the glorious wise one.”

\item \vleftofline{“}If, sir, you adopt\\
The teaching of the Buddha,\\
We will also go forth\\
In the presence of the glorious wise one.”

\item These three hundred brahmans\\
With hands held in \emph{añjalī}, ask:\\
\vleftofline{“}May we live the holy life\\
In your presence, Blessed One?”

\item \vleftofline{“}The holy life is well proclaimed,”\\
Said the Buddha to Sela,\\
\vleftofline{“}Apparent in this very life, without delay,\\
In which the going forth isn’t in vain,\\
For one heedful in the training.”

\item \vleftofline{“}It’s the eighth day, o seer,\\
Since we went to you for refuge.\\
In seven days, Blessed One,\\
We were tamed in your teaching.

\item You are the Buddha, you are the teacher\\
You are the sage who has overcome Māra;\\
You have cut off the underlying tendencies,\\
And having crossed over yourself, \\
You bring people across.

\item You have transcended attachments,\\
Your defilements have been torn apart;\\
Without grasping, like a lion,\\
You’ve abandoned fear and dread.

\item These three hundred monks\\
Stand with hands in \emph{añjalī}:\\
Put out your feet, great hero,\\
Let these beings of power venerate the teacher.”

\subsubsection*{16.7 Kāḷigodhāputtabhaddiya}

\item I rode on an elephant’s neck,\\
Wearing delicate clothes.\\
I ate rice conjey\\
With pure meat sauce.

\item Today I am fortunate, persevering,\\
Happy with the scraps in my alms-bowl;\\
Bhaddiya, son of Godhā,\\
Practices \emph{jhāna} without grasping.

\item Wearing rags, persevering,\\
Happy with the scraps in my alms-bowl;\\
Bhaddiya, son of Godhā,\\
Practices \emph{jhāna} without grasping.

\item Living on alms-food, persevering,\\
Happy with the scraps in my alms-bowl;\\
Bhaddiya, son of Godhā,\\
Practices \emph{jhāna} without grasping.

\item Possessing only three robes, persevering,\\
Happy with the scraps in my alms-bowl;\\
Bhaddiya, son of Godhā,\\
Practices \emph{jhāna} without grasping.

\item Going on alms-round from house to house \\
Without exception, persevering,\\
Happy with the scraps in my alms-bowl;\\
Bhaddiya, son of Godhā,\\
Practices \emph{jhāna} without grasping.

\item Sitting alone, persevering,\\
Happy with the scraps in my alms-bowl;\\
Bhaddiya, son of Godhā,\\
Practices \emph{jhāna} without grasping.

\item Eating only what is placed in the alms-bowl, \\
Persevering,\\
Happy with the scraps in my alms-bowl;\\
Bhaddiya, son of Godhā,\\
Practices \emph{jhāna} without grasping.

\item Never eating too late, persevering,\\
Happy with the scraps in my alms-bowl;\\
Bhaddiya, son of Godhā,\\
Practices \emph{jhāna} without grasping.

\item Living in the wilderness, persevering,\\
Happy with the scraps in my alms-bowl;\\
Bhaddiya, son of Godhā,\\
Practices \emph{jhāna} without grasping.

\item Living at the foot of a tree, persevering,\\
Happy with the scraps in my alms-bowl;\\
Bhaddiya, son of Godhā,\\
Practices \emph{jhāna} without grasping.

\item Living in the open, persevering,\\
Happy with the scraps in my alms-bowl;\\
Bhaddiya, son of Godhā,\\
Practices \emph{jhāna} without grasping.

\item Living in a charnel ground, persevering,\\
Happy with the scraps in my alms-bowl;\\
Bhaddiya, son of Godhā,\\
Practices \emph{jhāna} without grasping.

\item Accepting whatever seat is offered, persevering,\\
Happy with the scraps in my alms-bowl;\\
Bhaddiya, son of Godhā,\\
Practices \emph{jhāna} without grasping.

\item Not lying down to sleep, persevering,\\
Happy with the scraps in my alms-bowl;\\
Bhaddiya, son of Godhā,\\
Practices \emph{jhāna} without grasping.

\item Having few wishes, persevering,\\
Happy with the scraps in my alms-bowl;\\
Bhaddiya, son of Godhā,\\
Practices \emph{jhāna} without grasping.

\item Content, persevering,\\
Happy with the scraps in my alms-bowl;\\
Bhaddiya, son of Godhā,\\
Practices \emph{jhāna} without grasping.

\item Secluded, persevering,\\
Happy with the scraps in my alms-bowl;\\
Bhaddiya, son of Godhā,\\
Practices \emph{jhāna} without grasping.

\item Not socializing, persevering,\\
Happy with the scraps in my alms-bowl;\\
Bhaddiya, son of Godhā,\\
Practices \emph{jhāna} without grasping.

\item Energetic, persevering,\\
Happy with the scraps in my alms-bowl;\\
Bhaddiya, son of Godhā,\\
Practices \emph{jhāna} without grasping.

\item Giving up a valuable bronze bowl,\\
And a precious golden one, too,\\
I took up a clay bowl:\\
This is my second initiation.

\item Formerly I lived in a citadel \\
Surrounded by high walls,\\
With strong battlements and gates,\\
And guarded by swordsmen—\\
And I trembled with fear.

\item Today I am fortunate, free of trembling,\\
With fear and dread abandoned.\\
Bhaddiya, son of Godhā,\\
Has plunged into the forest and practices \emph{jhāna}.

\item Established in all the practices of virtue,\\
Developing mindfulness and understanding,\\
Gradually I attained\\
The end of all fetters.

\subsubsection*{16.8 Aṅgulimāla}

\item \vleftofline{“}Ascetic, you’re walking, \\
But you say ‘I’m standing still’;\\
And I’m standing still, but you tell me I’m not.\\
I’m asking you this, ascetic:\\
Why are you standing still and I’m not?”

\item \vleftofline{“}Aṅgulimāla, I always stand still—\\
I’ve given up violence towards all living beings.\\
But you have no restraint towards living creatures;\\
That’s why I’m standing still and you’re not.”

\item \vleftofline{“}It’s a been a long time since an ascetic,\\
A great sage who I honour, \\
Has entered this great forest.\\
Now that I’ve heard your verse on Dhamma,\\
I’ll discard a thousand evils.”

\item With these words, \\
The bandit hurled his sword and weapons\\
Down a pit, a cliff, a chasm.\\
Right there, he venerated the Fortunate One’s feet,\\
And asked the Buddha for the going-forth.

\item Then the Buddha, the compassionate great sage,\\
The teacher of the world together with its gods,\\
Said to him, “Come, monk!”\\
Just this was enough for him to be a monk.

\item \vleftofline{“}Whoever was heedless before,\\
And afterwards is not,\\
Lights up the world,\\
Like the moon freed from a cloud.

\item One whose bad deed\\
Is blocked by skilful action,\\
Lights up the world,\\
Like the moon freed from a cloud.

\item The young monk\\
Who is devoted to the teaching of the Buddha,\\
Lights up the world,\\
Like the moon freed from a cloud.

\item May even my enemies hear a Dhamma talk!\\
May even my enemies \\
Devote themselves to Buddha’s teaching!\\
May even my enemies associate when they can,\\
With those who establish people in the Dhamma!

\item May even my enemies hear Dhamma at suitable times,\\
From those who speak on acceptance,\\
Praising acquiescence;\\
And may they practice accordingly!

\item They would definitely not harm\\
Me or anyone else;\\
But would attain the ultimate peace,\\
Looking after creatures both firm and fragile.

\item Irrigators lead water,\\
Fletchers shape arrows,\\
Carpenters shape wood;\\
The disciplined tame themselves.

\item Some tame with sticks,\\
With hooked poles or whips;\\
But the poised one tamed me\\
Without rod or sword.

\item My name is ‘Harmless’,\\
Though I used to be harmful.\\
Today my name is truthful,\\
As I don’t harm anyone.

\item I used to be a bandit,\\
The notorious Aṅgulimāla.\\
Swept away in a great flood,\\
I went to Buddha as a refuge.

\item I used to have blood on my hands,\\
The notorious Aṅgulimāla.\\
See my going for refuge—\\
I’ve undone the attachment to being reborn \\
In any state of existence.

\item I’ve done many such deeds\\
As lead to a bad destination.\\
I’ve experienced the result of my deeds,\\
So I enjoy my food free of debt.

\item Fools and unintelligent people\\
Devote themselves to heedlessness.\\
But the intelligent protect heedfulness\\
As their best treasure.

\item Don’t devote yourself to heedlessness,\\
Nor delight in sexual intimacy.\\
If you are heedful and practice \emph{jhāna}\\
You’ll attain the highest happiness.

\item It was welcome, not unwelcome,\\
The advice I got was good.\\
Of things which are shared,\\
I encountered the best.

\item It was welcome, not unwelcome,\\
The advice I got was good.\\
I’ve attained the three knowledges,\\
And fulfilled the Buddha’s instructions.

\item In the wilderness, at the foot of a tree,\\
In mountains, or in caves;\\
At that time, wherever I stood,\\
My mind was anxious.

\item But now I lie down happily and stand up happily,\\
I live my life happily,\\
Out of Māra’s reach;\\
The teacher had compassion for me.

\item I used to belong to the brahman caste,\\
Highborn on both sides,\\
Now I’m a son of the Fortunate One,\\
The teacher, the King of Dhamma.

\item I am free of craving, without grasping,\\
My sense-doors are guarded and well-restrained.\\
I’ve destroyed the root of misery,\\
And attained the end of defilements.

\item I’ve attended on the teacher\\
And fulfilled the Buddha’s instructions.\\
The heavy burden is laid down,\\
I’ve undone the attachment \\
To being reborn in any state of existence.”

\subsubsection*{16.9 Anuruddha}

\item Leaving my mother and father behind,\\
As well as sisters, kinsmen, and brothers;\\
Abandoning the five kinds of sensual pleasures,\\
Anuruddha practices \emph{jhāna}.

\item Surrounded by song and dance,\\
Awakened by cymbals and gongs,\\
He did not find purification,\\
While delighting in Māra’s domain.

\item But he has gone beyond all that,\\
And delights in the teaching of the Buddha.\\
Having crossed over the entire flood,\\
Anuruddha practices \emph{jhāna}.

\item Sights, sounds, tastes, smells;\\
Touches that please the mind.\\
Having crossed over these as well,\\
Anuruddha practices \emph{jhāna}.

\item The sage returned from alms-round,\\
Alone, without companion.\\
Seeking rags from the dust heap,\\
Anuruddha is without defilements.

\item The thoughtful sage\\
Selected rags from the dust heap;\\
He picked them up, washed, dyed, and wore them;\\
Anuruddha is without defilements.

\item The principles of someone\\
Who has many wishes and is not content,\\
Who socializes and is conceited,\\
Are wicked and corrupted.

\item But someone who is mindful, of few wishes,\\
Content and untroubled,\\
Delighting in seclusion, joyful,\\
Always resolute and energetic;

\item Their principles are skilful,\\
Leading to awakening;\\
They are without defilements—\\
So it was said by the great sage.

\item Knowing my thought,\\
The world’s unsurpassed teacher\\
Came up to me in his mind-made body,\\
Using his psychic powers.

\item When I had that thought\\
He taught me more.\\
The Buddha, \\
Delighting in freedom from proliferation,\\
Taught it to me.

\item Understanding the Dhamma,\\
I lived happily in the teaching.\\
I’ve attained the three knowledges,\\
And fulfilled the Buddha’s instructions.

\item For the last fifty-five years\\
I have not lain down to sleep;\\
Twenty-five years have passed\\
Since drowsiness was uprooted.

\item The poised one, with steady heart,\\
Was not breathing;\\
Imperturbable, committed to peace,\\
The seer has realised \emph{nibbāna}.

\item With a positive mind\\
He put up with painful feelings;\\
The liberation of his heart\\
Was like the quenching of a lamp.

\item Now these touches and the other four\\
Are the last to be experienced by the sage;\\
Nor will there be other mental phenomena\\
Since the Buddha realised \emph{nibbāna}.

\item Weaver of the web, now there are no future lives\\
In the company of gods.\\
Transmigration through births is finished,\\
Now there is no more rebirth \\
Into any state of existence.

\item Whoever in a moment knows the thousand-fold world,\\
Together with the Brahmā realm;\\
That monk, a master of psychic powers,\\
Knowing the passing away and rebirth of beings,\\
Sees even the gods at that time.

\item In the past I was Annabhāra,\\
A poor carrier of fodder.\\
I made an offering\\
To the renowned ascetic, Upariṭṭha.

\item Then I was born in the Sakyan clan,\\
Where I was known as “Anuruddha”.\\
Surrounded by song and dance,\\
I was awakened by cymbals and gongs.

\item Then I saw the Buddha\\
The teacher, without fear from any direction;\\
Filling my mind with confidence in him,\\
I went forth into homelessness.

\item I know my past life,\\
Where I used to live—\\
I was born as Sakka,\\
And stayed among the Tāvatiṃsa gods.

\item Seven times I was a king of men\\
Ruling a kingdom,\\
Victorious in the four directions,\\
Lord of all India.\\
Without violence or sword,\\
I governed by principle.

\item Seven here, seven there,\\
For fourteen transmigrations\\
I remember my past lives;\\
At that time I stayed in the realm of the gods.

\item I have gained complete tranquillity\\
In \emph{samādhi} with five factors;\\
Peaceful, serene,\\
My clairvoyance is purified.

\item Steady in \emph{jhāna} with five factors,\\
I know the passing away and rebirth of beings,\\
Their coming and going,\\
Their lives in this state and that.

\item I’ve attended on the teacher\\
And fulfilled the Buddha’s instructions.\\
The heavy burden is laid down,\\
I’ve undone the attachment to being reborn \\
In any state of existence.

\item In the Vajjian village of Veḷuva,\\
At the end of life,\\
Beneath a thicket of bamboos,\\
I’ll realise \emph{nibbāna} without defilements.

\subsubsection*{16.10 Pārāpariya}

\item While the ascetic practiced \emph{jhāna},\\
Seated in seclusion, unified,\\
In the forest full of flowers,\\
This thought came to him:

\item \vleftofline{“}The behaviour of the monks\\
These days seems different\\
From when the lord of the world,\\
The best of men, was still here.

\item Their robes were only for covering the private parts,\\
And protection from the cold and wind;\\
They ate in moderation,\\
Content with whatever they were offered.

\item Whether refined or rough,\\
Little or much,\\
They ate only for sustenance,\\
Without greed or gluttony.

\item They weren’t so very eager,\\
For the requisites of life,\\
Such as tonics and other necessities,\\
As they were for the end of defilements.

\item In the wilderness, at the foot of trees,\\
In caves and caverns,\\
Committed to seclusion,\\
They lived with that as their final goal.

\item They were used to simple things, \\
And were easy to look after,\\
Gentle, their hearts not stubborn,\\
Unsullied, not talkative,\\
Their minds were intent on the goal.

\item In this way they inspired confidence,\\
 In their movements, eating, and practice;\\
Their deportment was smooth\\
As a stream of oil.

\item With the ending of all defilements,\\
Those senior monks have now realised \emph{nibbāna};\\
They were great meditators and great benefactors—\\
There are few like them today.

\item With the ending\\
Of good principles and understanding,\\
The conqueror’s teaching,\\
Full of all excellent qualities, has fallen apart.

\item This is the season\\
For bad principles and defilements.\\
Those who are ready for seclusion\\
Are all that’s left of the true Dhamma.

\item As they grow, the defilements\\
Possess many people;\\
They play with fools, I believe,\\
Like demons with the mad.

\item Overcome by defilements,\\
They run here and there\\
Among the causes for defilement,\\
As if they had declared war on themselves.

\item Having abandoned true Dhamma,\\
They argue with each other;\\
Following wrong views\\
They think, ‘This is better.’

\item They cut off their wealth,\\
Children, and wife to go forth;\\
But then they do what they shouldn’t,\\
For the sake of a measly spoon of alms-food.

\item They eat until their bellies are full,\\
And then they lie to sleep on their backs.\\
When they wake again, they keep on talking,\\
The kind of talk that the teacher criticized.

\item Valuing all the arts and crafts,\\
They train themselves in them;\\
Not being calm inside,\\
They think, ‘This is the purpose of the ascetic life’.

\item They provide clay, oil, and talcum powder,\\
Water, lodgings, and food\\
For householders,\\
Expecting more in return.

\item As well as tooth-picks, wood-apples,\\
Flowers, food to eat,\\
Well-cooked alms-food,\\
Mangoes and myrobalans.

\item In medicine they are like doctors,\\
In business like householders,\\
In decoration like prostitutes,\\
In sovereignty like lords.

\item Cheats, frauds,\\
False witnesses, sly:\\
Using multiple plans,\\
They enjoy material things.

\item Pursuing shams, contrivances, and plans,\\
By this means\\
They accumulate a lot of wealth\\
For the sake of their livelihood.

\item They assemble the community\\
For business rather than Dhamma.\\
They teach the Dhamma to others\\
For gain, not for the goal.

\item Those outside the Saṅgha\\
Quarrel over the Saṅgha’s property.\\
They’re shameless, and do not care\\
That they live on someone else’s property.

\item Some who have a shaven head \\
And wear the outer robe,\\
Are not devoted to practice,\\
But wish only to be honored,\\
Infatuated with property and reverence.

\item When things have come to this,\\
It’s not easy these days\\
To realise what has not yet been realised,\\
Or to preserve what has been realised.

\item A person with mindfulness established\\
Could walk without shoes\\
Even in a thorny place;\\
That is how a sage should walk in the village.

\item Remembering the meditators of old,\\
And recollecting their conduct;\\
Even in the latter days,\\
It is still possible to realise the deathless.”

\item That is what the ascetic, whose faculties\\
Were fully developed, said in the \emph{sāla} tree grove.\\
The holy man, the sage, realised \emph{nibbāna}:\\
Ending more rebirth into any state of existence.

\chapter*{Chapter Seventeen}
\addcontentsline{toc}{chapter}{Chapter Seventeen}
\markboth{Chapter Seventeen}{Chapter Seventeen}

\subsubsection*{17.1 Phussa}

\item Seeing many who inspire confidence,\\
Personally developed and well-restrained,\\
The sage Paṇḍarasagotta\\
Asked the one known as Phussa:

\item \vleftofline{“}In future times,\\
What desire and motivation\\
And behaviour will people have?\\
Please answer my question.”

\item \vleftofline{“}Listen to my words\\
O sage known as Paṇḍarasa:\\
And remember them carefully,\\
I will describe the future.

\item In the future many will be\\
Angry and hostile,\\
Denigrating, stubborn, and treacherous,\\
Envious, and holding divergent views.

\item Thinking they understand the profundity of the Dhamma,\\
They remain on the near shore.\\
Superficial and disrespectful towards the Dhamma,\\
They have no respect for one another.

\item In the future,\\
Many dangers will arise in the world.\\
Fools will defile\\
The Dhamma that has been taught so well.

\item Though lacking good qualities,\\
The incompetent, the talkative,\\
And the unlearned,\\
Will be powerful in Saṅgha proceedings.

\item Though possessing good qualities,\\
The competent, the conscientious,\\
And the unbiased,\\
Will be weak in Saṅgha proceedings.

\item In the future, fools will accept\\
Gold and silver,\\
Fields and property, goats and sheep,\\
And male and female bonded servants.

\item Fools looking to find fault in others,\\
But bereft of virtues themselves,\\
Will wander about, insolent,\\
Like cantankerous beasts.

\item They’ll be arrogant,\\
Wrapped in robes of blue;\\
Deceitful, obstinate, chatty, caustic,\\
They’ll wander as if they were noble ones.

\item With hair sleeked back with oil,\\
Fickle, their eyes painted with eye-liner,\\
They’ll travel on the high-road,\\
Wrapped in robes of ivory color.

\item They’ll love white clothes,\\
And they’ll detest the deep-dyed ochre robe,\\
The banner of the \emph{arahants},\\
Which is worn without disgust by the free.

\item They’ll want lots of things,\\
And be lazy, lacking energy;\\
Weary of the forest,\\
They’ll stay in villages.

\item Being unrestrained, they’ll keep company with\\
Those who obtain lots of things,\\
And who always enjoy wrong livelihood,\\
Following their example.

\item They won’t respect those\\
Who don’t obtain lots of things,\\
And they won’t associate with the wise,\\
Even though they’re very amiable.

\item Disparaging their own banner,\\
Which is dyed the colour of copper,\\
Some will wear the white banner\\
Of the followers of other religions.

\item Then they’ll have no respect\\
For the ochre robe;\\
The monks will not reflect\\
On the nature of the ochre robe.

\item This awful lack of reflection\\
Was unthinkable to the elephant,\\
Who was overcome by suffering,\\
Pierced by an arrow, and injured.

\item Then the six-tusked elephant,\\
Seeing the deep-coloured banner of the \emph{arahants},\\
Straight away spoke these verses\\
Connected with the goal.

\item The impure one\\
Who will wear the ochre robe\\
Without taming and truth,\\
Isn’t worthy of the ochre robe.

\item Whoever has rejected impurities,\\
Endowed with virtues,\\
Possessing truth and taming,\\
They are truly worthy of the ochre robe.

\item Devoid of virtue, unintelligent,\\
Wild, doing what they like,\\
Their minds all over the place, indolent\\
They are not worthy of the ochre robe.

\item Whoever is endowed with virtue,\\
Free of lust, possessing \emph{samādhi},\\
Their heart’s intention pure,\\
They are truly worthy of the ochre robe.

\item The conceited, arrogant fool,\\
Who has no virtue,\\
Is worthy of a white robe—\\
What use is an ochre robe for them?

\item In the future, monks and nuns\\
With corrupt hearts, disrespectful,\\
Will disparage those\\
With hearts of loving-kindness.

\item Though trained in wearing the robe\\
By senior monks,\\
The unintelligent will not listen,\\
Wild, doing what they like.

\item With that kind of attitude to training,\\
Those fools won’t respect each another,\\
Or take any notice of their mentors,\\
Like a wild horse with its charioteer.

\item So, in the future,\\
This will be the practice\\
Of monks and nuns,\\
When the latter days have come.

\item Before this terrifying future arrives,\\
Be easy to admonish,\\
Kind in speech,\\
And respect one another.

\item Have hearts of loving-kindness and compassion,\\
And keep your precepts;\\
Be energetic, resolute,\\
And always strong in exertion.

\item Seeing heedlessness as fearful,\\
And heedfulness as security,\\
Develop the eight-fold path,\\
Realising the deathless state.”

\subsubsection*{17.2 Sāriputta}

\item \vleftofline{“}A mindful person is like one of good conduct,\\
Or like one who is peaceful;\\
A heedful person is like one of good intentions,\\
Who is practicing \emph{jhāna};\\
Happy inside, possessing \emph{samādhi},\\
Solitary, contented; that is what they call a monk.

\item When eating fresh or dried food,\\
One shouldn’t be overly satisfied.\\
A monk should wander mindfully,\\
With unfilled belly, taking limited food.

\item Four or five mouthfuls before you’re full,\\
Drink some water;\\
This is enough to live comfortably\\
For a resolute monk.

\item Covered by a suitable robe,\\
Which is for this purpose;\\
This is enough to live comfortably\\
For a resolute monk.

\item When sitting cross-legged,\\
The rain doesn’t fall on the knees;\\
This is enough to live comfortably\\
For a resolute monk.

\item When you’ve seen happiness as suffering,\\
And suffering as a dart,\\
You know there’s no difference between them—\\
With what are you bound to the world?\\
What would you become?

\item When you think, ‘May I not associate \\
With people of bad wishes,\\
Lazy, lacking energy\\
With little learning, disrespectful’—\\
With what are you bound to the world?\\
What would you become?

\item A wise person who is learned,\\
Endowed with virtues,\\
Devoted to serenity of heart—\\
Let them stand at the head.

\item Whoever is devoted to proliferation,\\
A wild animal delighting in proliferation,\\
Is deprived of \emph{nibbāna},\\
The unexcelled safety from the yoke.

\item Whoever has given up proliferation,\\
Delighting in the path free of proliferation,\\
Is blessed with \emph{nibbāna},\\
The unexcelled safety from the yoke.

\item Whether in the village or in the wilderness,\\
In lands low or high,\\
Wherever \emph{arahants} live\\
Is a delightful place.

\item The wilderness is delightful!\\
Where most people find no delight,\\
Those who are free of lust delight there,\\
As they are not seeking sensual pleasures.

\item When you see someone who sees your faults,\\
A wise person who rebukes you,\\
You should stick close to such an intelligent person,\\
As if they were revealing some hidden treasure.\\
Sticking close to such a person, \\
Things get better, not worse.

\item You should advise, you should admonish;\\
You should curb rudeness;\\
For such a person is loved by the mindful,\\
Not loved by the unmindful.

\item The Blessed One, the Buddha, the seer\\
Was teaching Dhamma to another.\\
While Dhamma was being taught\\
I listened attentively, to understand the meaning—

\item My listening wasn’t wasted,\\
I’m liberated, without defilements.

\item Not for knowledge of past lives,\\
Nor even for clairvoyance;\\
Not for psychic powers, \\
Or reading the minds of others,\\
Nor for knowing people’s passing away \\
And being reborn;\\
Not for purifying the power of clairaudience,\\
Did I have any resolve.”

\item \vleftofline{“}His only shelter is at the foot of a tree;\\
With shaved head, wrapped in the outer robe,\\
The senior monk who is foremost in wisdom,\\
Upatissa himself practices \emph{jhāna}.

\item Entering a meditation state without thought,\\
A disciple of the Buddha\\
Is at that moment blessed\\
With noble silence.

\item Just like a rocky mountain\\
Is unshakable and firmly grounded;\\
So when delusion ends,\\
A monk, like a mountain, doesn’t tremble.”

\item \vleftofline{“}To the blameless man\\
Who is always seeking purity,\\
Even a hair-tip of evil\\
Seems the size of a cloud.

\item I don’t long for death;\\
I don’t long for life;\\
I will lay down this body,\\
Aware and mindful.

\item I don’t long for death;\\
I don’t long for life;\\
I await my time,\\
Like a worker waiting for their wages.”

\item \vleftofline{“}Both before and after\\
It’s death, not the deathless,\\
Practice, don’t perish—\\
Don’t let the moment pass you by.

\item Just like a frontier city,\\
Guarded inside and out,\\
So you should ward yourselves\\
Don’t let the moment pass you by.\\
Those who pass up the moment\\
Grieve when they end up in hell.”

\item \vleftofline{“}Calm and quiet, \\
Wise in counsel, not restless;\\
He shakes off bad qualities\\
As the wind shakes leaves off a tree.

\item Calm and quiet, \\
Wise in counsel, not restless;\\
He plucks off bad qualities\\
As the wind plucks leaves off a tree.

\item Calm and sorrowless,\\
Clear and undisturbed,\\
Of good virtue and intelligent:\\
You should put an end to suffering.”

\item \vleftofline{“}Some householders, \\
And even some of those gone forth,\\
Are not to be trusted.\\
Even some who were good later become bad;\\
While some who were bad become good.

\item Sensual desire, ill-will\\
Dullness and drowsiness,\\
Restlessness, and doubt:\\
These are the five mental stains for a monk.

\item Whoever’s \emph{samādhi} does not waver,\\
Regardless of whether or not\\
They receive honours,\\
Is one who lives heedfully.

\item They regularly practice \emph{jhāna},\\
With subtle insight into views;\\
Delighting in the end of grasping,\\
They are said to be a good person.

\item Ocean, earth,\\
Mountains, wind—\\
These cannot compare\\
With the teacher’s magnificent liberation.

\item He is the senior monk who keeps \\
The Wheel of Dhamma rolling,\\
Possessing great knowledge and \emph{samādhi}.\\
Like earth, like water, like fire,\\
He is neither attracted nor repelled.

\item He has attained the perfection of wisdom,\\
He has great intelligence and great discernment;\\
He is not stupid, but appears stupid;\\
He always wanders, quenched.

\item I’ve attended on the teacher\\
And fulfilled the Buddha’s instructions.\\
The heavy burden is laid down,\\
I’ve undone the attachment to being reborn \\
In any state of existence.

\item Strive on with heedfulness:\\
This is my advice.\\
Come, now I’ll realise \emph{nibbāna},\\
I am liberated in every way.”

\subsubsection*{17.3 Ānanda}

\item \vleftofline{“}A wise person would not make friends\\
With a slanderous or hostile person,\\
With a miser, or one who delights \\
In the misfortunes of others;\\
Association with a bad person is harmful.

\item The wise would make friends \\
With the faithful, the pleasant,\\
Those with understanding, who are learned;\\
Association with a good person is blessed.

\item See this fancy puppet,\\
A heap of sores, a composite body,\\
Diseased, obsessed over,\\
Having no lasting stability.

\item See this fancy shape,\\
With its gems and earrings;\\
It is bones wrapped with skin,\\
Made pretty by its clothes.

\item Rouged feet\\
And powdered face\\
Is enough to delude a fool,\\
But not a seeker of the far shore.

\item Hair in eight braids\\
And eyeliner applied,\\
Is enough to delude a fool,\\
But not a seeker of the far shore.

\item Like a newly decorated make-up box,\\
The disgusting body all adorned\\
 Is enough to delude a fool,\\
But not a seeker of the far shore.

\item Gotama is learned, a brilliant speaker,\\
The attendant to the Buddha.\\
Unfettered, with burden put aside,\\
He lies down to sleep.

\item Unfettered, his defilements have ended,\\
He has transcended attachments, \\
And has attained \emph{nibbāna}.\\
He bears his final body,\\
Gone beyond birth and death.

\item Gotama, in whom the teachings of the Buddha,\\
The Kinsman of the Sun, are established,\\
Stands on the path\\
Leading to \emph{nibbāna}.

\item I learned 82,000 from the Buddha,\\
And 2,000 from the monks;\\
These 84,000\\
Are the teachings I have memorized.

\item A person of little learning\\
Ages like an ox—\\
Their flesh grows,\\
But their wisdom doesn’t.

\item A learned person who, on account of their learning,\\
Looks down on someone of little learning,\\
Seems to me like\\
A blind man holding a lamp.

\item You should stay close to a learned person—\\
Don’t lose what you’ve learned.\\
It is the root of the spiritual life,\\
So you should memorize the Dhamma.

\item Knowing the sequence \\
And meaning of the teaching,\\
Skilled in the interpretation of terms,\\
He makes sure it is well memorized,\\
And then examines the meaning.

\item Accepting the teachings, he becomes enthusiastic;\\
Making an effort, he scrutinizes the Dhamma;\\
Striving at the right time,\\
He is serene inside himself.

\item If you want to understand the Dhamma,\\
You should associate with the sort of person\\
Who is learned, and has memorized the Dhamma,\\
A wise disciple of the Buddha.

\item A monk who is learned, and has memorized the Dhamma,\\
A keeper of the great sage’s treasury,\\
Is a visionary for the entire world,\\
Venerable, and learned.

\item Delighting in Dhamma, enjoying Dhamma,\\
Reflecting on Dhamma,\\
Recollecting Dhamma,\\
He doesn’t decline in the true Dhamma.

\item When your body is pampered and heavy,\\
While your remaining time is running out;\\
Greedy for physical pleasure,\\
How can you find happiness as an ascetic?

\item Every direction is unclear!\\
The Dhamma does not occur to me!\\
With the passing of our good friend,\\
It all seems dark.

\item If your friend has passed away,\\
And your teacher is past and gone,\\
There’s no friend like\\
Mindfulness of the body.

\item The old have passed away,\\
And I don’t get on with the new.\\
Today I meditate alone\\
Like a bird snug in its nest.”

\item \vleftofline{“}Many international visitors\\
Have come to see.\\
Don’t block the audience,\\
Let the congregation see me.”

\item \vleftofline{“}Lots of international visitors\\
Have come to see.\\
The teacher grants them the opportunity,\\
The seer doesn’t stop them.

\item For the 25 years\\
Since I have been a trainee,\\
No sensual perception arose in me:\\
See the excellence of the Dhamma!

\item For the 25 years\\
Since I have been a trainee,\\
No malicious perception arose in me:\\
See the excellence of the Dhamma!

\item For 25 years\\
I attended on the Blessed One\\
With loving deeds,\\
Like a shadow that never left.

\item For 25 years\\
I attended on the Blessed One\\
With loving words,\\
Like a shadow that never left.

\item For 25 years\\
I attended on the Blessed One\\
With loving thoughts,\\
Like a shadow that never left.

\item While the Buddha was walking meditation,\\
I walked meditation behind him.\\
As he taught the Dhamma,\\
Knowledge arose in me.

\item I’m a trainee, who has more to do!\\
My mind is not perfected!\\
Yet the teacher, who was so compassionate to me,\\
Has passed into \emph{nibbāna}.

\item Then there was terror!\\
Then they had goose-bumps!\\
When the Buddha, endowed with all qualities,\\
Passed into \emph{nibbāna}.”

\item \vleftofline{“}Ānanda, who was learned, \\
And had memorized the Dhamma,\\
A keeper of the great sage’s treasury,\\
A visionary for the entire world,\\
Has passed into \emph{nibbāna}.

\item He was learned, and had memorized the Dhamma,\\
A keeper of the great sage’s treasury,\\
A visionary for the entire world,\\
When all was black, he dispelled the dark.

\item He is the sage who remembered the teachings,\\
And mastered their sequence, holding them firm.\\
The senior monk who memorized the Dhamma,\\
Ānanda was a mine of gems.”

\item \vleftofline{“}I’ve attended on the teacher\\
And fulfilled the Buddha’s instructions.\\
The heavy burden is laid down,\\
I’ve undone the attachment to being reborn \\
In any state of existence.”

\chapter*{Chapter Eighteen}
\addcontentsline{toc}{chapter}{Chapter Eighteen}
\markboth{Chapter Eighteen}{Chapter Eighteen}

\subsubsection*{18.1 Mahākassapa}

\item You shouldn’t live for the adulation of a following;\\
It turns your mind, and makes \emph{samādhi} hard to find.\\
Seeing that popularity is suffering,\\
You shouldn’t accept a following.

\item A sage should not visit respectable families\\
It turns your mind, and makes \emph{samādhi} hard to find.\\
One who’s eager and greedy for flavours,\\
Misses the goal that brings such happiness.

\item They know that this really is a bog,\\
This homage and veneration \\
Among respectable families.\\
Honor is a subtle dart, hard to extract,\\
And hard for a bad man to give up.

\item I came down from my lodging\\
And entered the city for alms.\\
I courteously stood by\\
While a leper ate.

\item With his putrid hand\\
He offered me a morsel.\\
Putting the morsel in my bowl,\\
His finger broke off right there.

\item Leaning against the foot of a wall,\\
I ate that morsel.\\
While eating, and afterwards,\\
I did not feel any disgust.

\item Anyone who makes use of\\
Leftovers for food,\\
Putrid urine as medicine,\\
The root of a tree as lodging,\\
And rags from the rubbish-heap as robes,\\
Is at home in any direction.

\item Where some have perished\\
While climbing the mountain,\\
There Kassapa ascends;\\
An heir of the Buddha,\\
Aware and mindful,\\
Relying on his psychic powers.

\item Returning from alms-round,\\
Kassapa ascends the mountain,\\
And practices \emph{jhāna} without grasping,\\
With fear and dread abandoned.

\item Returning from alms-round,\\
Kassapa ascends the mountain,\\
And practices \emph{jhāna} without grasping,\\
Quenched amongst those who burn.

\item Returning from alms-round,\\
Kassapa ascends the mountain,\\
And practices \emph{jhāna} without grasping,\\
His duty done, without defilements.

\item Strewn with garlands of the musk-rose tree,\\
These regions are delightful.\\
Lovely, resounding with the trumpeting of elephants:\\
These rocky crags delight me!

\item They look like blue-black storm clouds, glistening, \\
Cooled with the waters of clear-flowing streams,\\
And covered with ladybird beetles:\\
These rocky crags delight me!

\item Like the peak of a blue-black storm cloud,\\
Or like a fine peaked house,\\
Lovely, resounding with the trumpeting of elephants:\\
These rocky crags delight me!

\item The rain comes down on the lovely flats,\\
In the mountains frequented by sages.\\
Echoing with the cries of peacocks,\\
These rocky crags delight me!

\item It’s enough for me,\\
Desiring to practice \emph{jhāna}, resolute and mindful.\\
It’s enough for me,\\
A resolute monk, desiring the goal.

\item It’s enough for me,\\
A resolute monk, desiring ease,\\
It’s enough for me,\\
Desiring to practice, resolute and poised.

\item Covered with flowers of flax,\\
Like the sky covered with clouds,\\
Full of flocks of many different birds,\\
These rocky crags delight me!

\item Empty of householders,\\
Frequented by herds of deer,\\
Full of flocks of many different birds,\\
These rocky crags delight me!

\item The water is clear and the gorges are wide,\\
Monkeys and deer are all around;\\
Festooned with dewy moss,\\
These rocky crags delight me!

\item Music played by a five-piece band\\
Can never make you as happy,\\
As when, with unified mind,\\
You rightly discern the Dhamma.

\item Don’t get involved in lots of work,\\
Avoid people, and don’t try to get more requisites.\\
If you’re eager and greedy for flavours,\\
You’ll miss the goal that brings such happiness.

\item Don’t get involved in lots of work,\\
Avoid what doesn’t lead to the goal.\\
The body gets worn out and fatigued,\\
And when you suffer, you won’t find tranquillity.

\item You won’t see yourself\\
By merely reciting words,\\
Wandering stiff-necked\\
And thinking, “I’m better.”

\item The fool is no better,\\
But they think they are.\\
The wise don’t praise\\
Stiff-minded people.

\item Whoever is not affected\\
By the modes of conceit—\\
\vleftofline{“}I am better”, “I am not better”,\\
\vleftofline{“}I am worse”, or “I am the same”—

\item Poised, with such understanding,\\
Endowed with virtues,\\
And devoted to tranquillity of mind:\\
That is who the wise praise.

\item Whoever has no respect\\
For their companions in spiritual life\\
Is as far from true Dhamma\\
As the sky is from the earth.

\item Those whose conscience and shame\\
Are always rightly established,\\
Thrive in the spiritual life,\\
For them, there is no rebirth \\
In any state of existence.

\item If a monk who is haughty and fickle,\\
Wears rags from the rubbish-heap,\\
Like a monkey in a lion skin,\\
That doesn’t make him impressive.

\item But if they are humble and stable,\\
Controlled, with faculties restrained,\\
Then wearing rags \\
From the rubbish-heap is impressive,\\
Like a lion in a mountain cave.

\item These famous gods\\
Endowed with psychic powers,\\
All 10,000 of them,\\
Belong to the retinue of Brahmā.

\item They stand with hands in \emph{añjalī},\\
Honouring Sāriputta,\\
The general of the Dhamma, the hero,\\
The great meditator who is endowed with \emph{samādhi}.

\item \vleftofline{“}Homage to you, thoroughbred among men!\\
Homage to you, best among men!\\
We do not even understand\\
The basis of your \emph{jhāna}.

\item The profound domain of the Buddhas\\
Is truly amazing.\\
We do not understand them,\\
Though we’ve gathered here to split hairs.”

\item When he saw the company of gods\\
Paying homage to Sāriputta—\\
Who is truly worthy of homage—\\
Kappina smiled.

\item As far as this Buddha-field extends\\
I am outstanding in ascetic practices.\\
I have no equal,\\
Apart from the great sage himself.

\item I’ve attended on the teacher\\
And fulfilled the Buddha’s instructions.\\
The heavy burden is laid down,\\
I’ve undone the attachment to being reborn \\
In any state of existence.

\item Like a lotus flower unstained by water,\\
Gotama the immeasurable is unstained\\
By robes, lodgings, or food.\\
He inclines to renunciation,\\
And has escaped being reborn\\
In the three states of existence.

\item The great sage’s neck \\
Is the establishment of mindfulness;\\
Faith is his hands, and wisdom his head.\\
Having great knowledge,\\
He always wanders, quenched.

\chapter*{Chapter Nineteen}
\addcontentsline{toc}{chapter}{Chapter Nineteen}
\markboth{Chapter Nineteen}{Chapter Nineteen}

\subsubsection*{19.1 Tāḷapuṭa}

\item Oh, when will I stay in a mountain cave,\\
Alone, with no companion,\\
Discerning all states of existence as impermanent?\\
This hope of mine, when will it be?

\item Oh, when will I stay happily in the forest,\\
A sage wearing a torn robe, dressed in ochre,\\
Unselfish, without desire,\\
With greed, hatred, and delusion destroyed?

\item Oh, when will I stay alone in the wood,\\
Fearless, discerning this body as impermanent,\\
A nest of death and disease,\\
Oppressed by death and old age;\\
When will it be?

\item Oh, when will I live, \\
Having grasped the sharp sword of wisdom\\
And cut the creeper of craving \\
That tangles around everything,\\
The mother of fear, the bringer of suffering,\\
When will it be?

\item Oh, when will I, seated on the lion’s throne,\\
Swiftly grasp the sword of the sages,\\
Forged by wisdom, of fiery might,\\
And swiftly break Māra and his army? \\
When will it be?

\item Oh, when will I be seen striving in the assemblies\\
By those who are virtuous, poised, \\
Respecting the Dhamma,\\
Seeing things as they are, with faculties subdued?\\
When will it be?

\item Oh, when will I focus on my own goal \\
On Giribbaja mountain,\\
Free of oppression by laziness, hunger, thirst,\\
Wind, heat, insects, and reptiles?\\
When will it be?

\item Oh, when will I have \emph{samādhi} and mindfulness,\\
And with understanding attain the four truths,\\
That were realized by the great sage,\\
And are so very hard to see? When will it be?

\item Oh, when will I, devoted to tranquillity,\\
See with understanding the infinite sights,\\
Sounds, smells, tastes, touches, \\
And mental phenomena as burning? \\
When will it be?

\item Oh, when will I not be downcast\\
Because of criticism,\\
Nor elated because of praise?\\
When will it be?

\item Oh when will I discern the aggregates\\
And the infinite varieties of phenomena,\\
Both internal and external, as no more than\\
Wood, grass, and creepers? \\
When will it be?

\item Oh, when will the winter clouds rain freshly\\
As I wear my robe in the forest,\\
Walking the path trodden by the sages?\\
When will it be?

\item Oh, when will I rise up, \\
Intent on attaining the deathless,\\
Hearing in the mountain cave\\
The cry of the crested peacock in the forest?\\
When will it be?

\item Oh, when will I cross the Ganges, Yamunā,\\
And Sarasvatī rivers, the Pātāla country,\\
And the dangerous Baḷavāmukha sea,\\
By psychic power, without hindrance? \\
When will it be?

\item Oh, when will I be devoted to \emph{jhāna},\\
Rejecting entirely the signs of beauty,\\
Splitting apart desire for sensual pleasures,\\
Like an elephant that wanders without ties;\\
When will it be?

\item Oh, when will I realise the teaching of the great sage\\
And be content, like a poor person in debt,\\
Harassed by creditors, who finds a hidden treasure?\\
When will it be?

\item For many years you begged me,\\
\vleftofline{“}Enough of living in a house for you!”\\
Why do you not urge me on, mind,\\
Now I’ve gone forth as an ascetic?

\item Didn’t you beg me, mind,\\
\vleftofline{“}On Giribbaja, the birds with colourful wings,\\
Greeting the thunder, Mahinda’s voice,\\
Will delight you as you practice \emph{jhāna} in the forest”?

\item In my family circle, \\
Friends, loved ones, and relatives;\\
And in the world, \\
Sports and play, and sensual pleasures;\\
All these I have abandoned for the sake of this:\\
And even then you’re not content with me, mind!

\item This is mine alone, it doesn’t belong to others;\\
When it is time to don your armour, why lament?\\
Reflecting that all this is unstable,\\
I went forth, longing for the deathless state.

\item The methodical teacher, supreme among people,\\
Great physician, charioteer of tractable people, said,\\
\vleftofline{“}The mind sways like a monkey,\\
So it’s very hard to control if you are not free of lust.”

\item Sensual pleasures are diverse, sweet, delightful;\\
Ignorant unenlightened people are attached to them.\\
Seeking to be reborn in another state of existence, they wish for suffering;\\
Led on by their mind, they’re relegated to hell.

\item \vleftofline{“}Staying in the grove resounding with cries\\
Of peacocks and herons, \\
And liked by leopards and tigers,\\
Abandon concern for the body, without fail!”\\
So you used to urge me, mind.

\item \vleftofline{“}Develop the \emph{jhāna}s and spiritual faculties,\\
The powers, factors of awakening, \\
And \emph{samādhi} meditation;\\
Realise the three knowledges \\
In the teaching of the Buddha!”\\
So you used to urge me, mind.

\item \vleftofline{“}Develop the eight-fold path \\
For realizing the deathless,\\
Emancipating, \\
Plunging into the end of  all suffering,\\
And cleansing all defilements!”\\
So you used to urge me, mind.

\item \vleftofline{“}Properly reflect on the aggregates as suffering,\\
And abandon that from which suffering arises;\\
Make an end of suffering in this very life!”\\
So you used to urge me, mind.

\item \vleftofline{“}Properly discern that impermanence is suffering,\\
That emptiness is non-self, and that misery is death.\\
Uproot the wandering mind!”\\
So you used to urge me, mind.

\item \vleftofline{“}Bald, unsightly, accursed,\\
Seek alms amongst families, bowl in hand.\\
Devote yourself to the word of the teacher, \\
The great sage!”\\
So you used to urge me, mind.

\item \vleftofline{“}Wander the streets well-restrained,\\
With your mind unattached \\
To families and sensual pleasures,\\
Like the full moon when the night is clear!”\\
So you used to urge me, mind.

\item \vleftofline{“}Be a wilderness-dweller and an alms-eater,\\
One who lives in charnel grounds, a rag-robe wearer,\\
One who never lies down, \\
Always delighting in ascetic practices.”\\
So you used to urge me, mind.

\item Mind, when you urge me \\
Towards the impermanent and unstable,\\
You are acting just like a person who plants trees,\\
Then, when they are about to fruit,\\
Wishes to cut down the very same trees.

\item You, incorporeal mind, far-traveller, lone-wanderer:\\
I won’t do your bidding any more.\\
Sensual pleasures are suffering, painful, \\
And very dangerous;\\
I’ll wander with my mind focussed only on \emph{nibbāna}.

\item I didn’t renounce due to bad luck or shamelessness,\\
Nor because of a whim, nor banishment,\\
Nor for the sake of a livelihood;\\
It was because I agreed \\
To the promise you made, mind.

\item \vleftofline{“}Having few wishes, abandoning disparagment,\\
Stilling suffering: these are praised by good people.”\\
So you used to urge me, mind,\\
But now you continue with your old habits!

\item Craving, ignorance, the loved and unloved,\\
Pretty sights, pleasant feelings,\\
And the delightful kinds of sensual pleasure: \\
I’ve vomited them all;\\
And I can’t swallow back what I’ve vomited up.

\item I’ve done your bidding everywhere, mind!\\
For many births, \\
I haven’t done anything to upset you,\\
Yet you show your gratitude \\
By producing craving inside yourself!\\
For a long time I’ve transmigrated \\
In the suffering you’ve created.

\item Only you, mind, make us holy men;\\
You make us lords or royal sages;\\
Sometimes we become traders or workers;\\
Life as a god is also on account of you.

\item You alone make us titans;\\
Because of you we are born in hell;\\
Then sometimes we become animals,\\
Life as an ghost is also on account of you.

\item Come what may, you won’t betray me again,\\
Dazzling me with your ever-changing display;\\
You play with me like I’m mad—\\
But how have I ever failed you, mind?

\item In the past my mind wandered\\
How it wished, where it liked, as it pleased.\\
Now I’ll carefully guide it,\\
As a rutting elephant is guided \\
By a trainer with a hook.

\item The teacher willed that this world appear to me\\
As impermanent, unstable, insubstantial.\\
Mind, let me leap into the conqueror’s teaching,\\
Carry me over the great flood, so very hard to cross.

\item Things have changed, mind!\\
Nothing could make me return to your control!\\
I’ve gone forth in the teaching of the great sage,\\
Those like me don’t come to ruin.

\item Mountains, oceans, rivers, the earth;\\
The four directions, the intermediate directions, \\
Below and in the sky;\\
The three states of existence are all \\
Impermanent and troubled—\\
Where can you go to find happiness, mind?

\item Mind, what will you do to someone\\
Who has made the ultimate commitment?\\
Nothing could make me a follower\\
Under your control, mind;
There’s no way you’d touch a bellows\\
With a mouth open at each end;\\
Let alone the body flowing with its nine streams!

\item You’ve ascended the mountain peak, \\
Full of nature’s beauty,\\
Frequented by boars and antelopes,\\
A grove sprinkled with fresh water in the rainy-season;\\
And there you’ll be happy in your cave-home.

\item Peacocks with beautiful necks and crests,\\
Colourful tail-feathers and wings,\\
Crying out at the sweet-sounding thunder:\\
They’ll delight you \\
As you practice \emph{jhāna} in the forest.

\item When the sky has rained down,\\
And the grass is four inches high,\\
And the grove is full of flowers, like a cloud,\\
In the mountain cleft, like the fork of a tree, I’ll lie;\\
It will be as soft as cotton-buds.

\item I’ll act as a master does:\\
Let whatever I get be enough for me.\\
I’ll make you as supple,\\
As a good worker makes a cat-skin bag.

\item I’ll act as a master does:\\
Let whatever I get be enough for me.\\
I’ll control you with my energy,\\
As the trainer controls \\
A rutting elephant with a hook.

\item Now that you’re well-tamed and reliable,\\
I can use you, \\
Like a trainer uses a straight-running horse,\\
To practice the safe path,\\
Cultivated by those who take care of their minds.

\item I shall strongly fasten you to a meditation subject,\\
As an elephant is tied to a post with firm rope.\\
You’ll be well-guarded by me, \\
Well-developed by mindfulness,\\
And unattached to rebirth in all states of existence.

\item You’ll use understanding \\
To cut the follower of the wrong path,\\
Restrain them by practice, \\
And settle them on the right path;\\
And when you have seen the cause of suffering \\
Arise and pass away,\\
You’ll be an heir to the greatest teacher.

\item Under the sway of the four distortions, mind,\\
You led me as if all around the world;\\
And now you won’t associate \\
With the great sage of compassion,\\
The cutter of fetters and bonds?

\item Like a deer roaming free in the colourful forest,\\
I’ll ascend the lovely mountain wreathed in cloud,\\
And rejoice to be on that hill, free of folk—\\
There is no doubt you’ll perish, mind.

\item The men and women who live \\
Under your will and command,\\
Whatever pleasure they experience,\\
They are ignorant and fall under Māra’s control;\\
Loving life, they’re your disciples, mind.

\chapter*{Chapter Twenty}
\addcontentsline{toc}{chapter}{Chapter Twenty}
\markboth{Chapter Twenty}{Chapter Twenty}

\subsubsection*{20.1 Mahāmoggallāna}

\item \vleftofline{“}Living in the wilderness, eating only alms-food,\\
Happy with whatever scraps fall into the alms-bowl,\\
And serene inside:\\
Let us tear apart the army of death.

\item Living in the wilderness, eating only alms-food,\\
Happy with whatever scraps fall into the alms-bowl,\\
Let us smash the army of death,\\
Like an elephant smashing a reed hut.

\item Living at the foot of a tree, persevering,\\
Happy with whatever scraps fall into the alms-bowl,\\
And serene inside:\\
Let us tear apart the army of death.

\item Living at the foot of a tree, persevering\\
Happy with whatever scraps fall into the alms-bowl,\\
Let us crush the army of death,\\
Like an elephant crushing a reed hut.”

\item \vleftofline{“}With a skeleton as a hut,\\
Woven together with flesh and tendons—\\
Damn this stinking body!\\
Which cherishes other bodies.

\item You sack of dung wrapped in skin!\\
You demon with horns on your chest!\\
Your body has nine streams,\\
Which are flowing all the time.

\item With its nine streams,\\
Your body stinks, full of dung.\\
A monk seeking purity would avoid it altogether,\\
Like excrement.

\item If they knew you\\
Like I do,\\
They’d keep far away,\\
Like a cesspit in the rainy-season.”

\item \vleftofline{“}So it is, great hero!\\
As you say, ascetic!\\
But some sink here\\
Like an old bull in mud.”

\item \vleftofline{“}Whoever might think\\
Of making the sky yellow,\\
Or any other colour,\\
Would only be causing trouble for themselves.

\item This mind is like the sky:\\
Serene inwardly.\\
Evil-minded one, don’t attack me\\
Like a moth to a bonfire.”

\item \vleftofline{“}See this fancy puppet,\\
A heap of sores, a composite body,\\
Diseased, obsessed over,\\
Having no lasting stability.

\item See this fancy shape,\\
With its gems and earrings;\\
It is bones wrapped with skin,\\
Made pretty by its clothes.

\item Rouged feet\\
And powdered face\\
Is enough to delude a fool,\\
But not a seeker of the far shore.

\item Hair in eight braids\\
And eyeliner applied,\\
Is enough to delude a fool,\\
But not a seeker of the far shore.

\item Like a newly decorated makeup box,\\
The disgusting body all adorned\\
 Is enough to delude a fool,\\
But not a seeker of the far shore.

\item The hunter laid his trap,\\
But the deer didn’t get caught in the snare;\\
Having eaten the bait we go,\\
Leaving the deer-trapper to lament.

\item The hunter’s trap is broken,\\
And the deer didn’t get caught in the snare;\\
Having eaten the bait we go,\\
Leaving the deer-trapper to lament.”

\item \vleftofline{“}Then there was terror!\\
Then they had goose-bumps!\\
When Sāriputta, endowed with many qualities,\\
Passed into \emph{nibbāna}.

\item All conditions are impermanent,\\
Their nature is to rise and fall.\\
They arise, then they cease—\\
And their stilling is bliss.”

\item \vleftofline{“}Those who see the five aggregates\\
As other, not as self,\\
Penetrate a subtle thing,\\
Like a hair-tip with an arrow.

\item Those who see conditions\\
As other, not as self,\\
Pierce a fine thing,\\
Like a hair-tip with an arrow.”

\item \vleftofline{“}As if struck by a sword,\\
As if their head was on fire,\\
Mindful, a monk should go forth,\\
To abandon desire for sensual pleasures.

\item As if struck by a sword,\\
As if their head was on fire,\\
Mindful, a monk should go forth,\\
To abandon desire to be reborn \\
In any state of existence.”

\item \vleftofline{“}Encouraged by the developed one,\\
Bearing his final body,\\
I shook the palace of Migāra’s mother\\
With my big toe.”

\item \vleftofline{“}This isn’t something you can get by being slack;\\
This isn’t something that takes little strength:\\
The realization of \emph{nibbāna},\\
The release from all attachments.”

\item \vleftofline{“}This young monk,\\
The best of men,\\
Has vanquished Māra and his mount,\\
And bears his final body.”

\item \vleftofline{“}Lightning flashes down\\
On the cleft of Vebhāra and Paṇḍava.\\
But in the mountain cleft, the son of the inimitable\\
Is poised and absorbed in \emph{jhāna}.”

\item \vleftofline{“}Calm and quiet,\\
The sage in his secluded lodging,\\
The heir to the best of Buddhas,\\
Is honoured even by Brahmā.”

\item \vleftofline{“}Calm and quiet,\\
The sage in his secluded lodging,\\
The heir to the best of Buddhas:\\
Brahman, you should honor Kassapa!

\item Even if someone were to be born\\
A hundred times repeatedly in the human realm,\\
And always as a brahman,\\
A student who memorized the Vedas,

\item And if he were a teacher,\\
With mastery of the three Vedas:\\
Honoring such a person\\
Isn’t worth a sixteenth of that.

\item Whoever attains the eight emancipations\\
Forwards and backwards before breakfast,\\
And then goes on alms-round—

\item Don’t attack such a monk!\\
Don’t ruin yourself, brahman!\\
Have faith in the \emph{arahant}\\
Quickly venerate him with hands in \emph{añjalī},\\
Don’t let your head be split open!”

\item \vleftofline{“}If you think transmigration is the important thing,\\
You don’t see the Dhamma.\\
You’re following a twisted path,\\
A bad path that will lead you down.

\item Like a worm smeared with dung,\\
He is besotted with conditions.\\
Sunk in gain and honour,\\
Poṭṭhila goes on, hollow.”

\item \vleftofline{“}See Sāriputta coming!\\
It is good to see him.\\
Liberated in both ways,\\
Serene inside himself.

\item With dart removed and fetters ended,\\
With the three knowledges, destroyer of death,\\
Worthy of offerings,\\
An unsurpassed field of merit for people.”

\item These famous gods\\
Endowed with psychic powers,\\
All 10,000 of them,\\
Are ministers of Brahmā.\\
They stand with hands in \emph{añjalī},\\
Honouring Moggallāna.

\item \vleftofline{‘}Homage to you, thoroughbred among men!\\
Homage to you, best of men!\\
Since your defilements are ended,\\
You, sir, are worthy of offerings!’”

\item \vleftofline{“}Venerated by men and gods,\\
He has arisen, the transcender of death.\\
He is undefiled by conditions,\\
As a lotus-flower by water.

\item Knowing in an hour the thousand-fold world,\\
Including the Brahmā realm;\\
Having mastery of psychic powers, \\
And the knowledge \\
Of the passing away and rebirth of beings in time:\\
That monk sees the gods.”

\item \vleftofline{“}Sāriputta, the monk who has crossed over,\\
Would be supreme\\
Because of his wisdom, \\
Virtue, and peace.

\item But in a moment I can create the likenesses\\
Of ten million times 100,000 people!\\
I’m skilled in transformations;\\
I’m a master of physic powers.

\item A member of the Moggallāna clan, \\
Attained to perfection and mastery\\
In \emph{samādhi} and knowledge,\\
Wise in the teachings of the unattached,\\
With serene faculties, has burst his bonds,\\
Like an elephant bursts a rope of creeper.

\item I’ve attended on the teacher\\
And fulfilled the Buddha’s instructions.\\
The heavy burden is laid down,\\
I’ve undone the attachment \\
To being reborn in any state of existence.

\item I’ve attained the goal\\
For the sake of which I went forth\\
From home life into homelessness—\\
The ending of all fetters.”

\item \vleftofline{“‘}What kind of hell was that,\\
Where Dussī was boiled,\\
After attacking the disciple Vidhura,\\
Along with the brahmin Kakusandha?’

\item \vleftofline{‘}There were 100 iron spikes,\\
Each one individually causing pain:\\
That was the kind of hell\\
Where Dussī was boiled,\\
After attacking the disciple Vidhura\\
Along with the brahmin Kakusandha.’

\item \vleftofline{‘}Dark One, if you attack\\
A monk who knows this from their own experience,\\
A disciple of the Buddha,\\
You will fall into suffering.

\item \vleftofline{‘}Mansions that last for an aeon\\
Stand in the middle of a lake;\\
The colour of lapis lazuli,\\
Brilliant, sparkling, and shining;\\
Many nymphs of diverse colours\\
Dance there.

\item \vleftofline{‘}Dark One, if you attack\\
A monk who knows this from their own experience,\\
A disciple of the Buddha,\\
You will fall into suffering.

\item \vleftofline{‘}The one who, encouraged by the Buddha,\\
With the monastic Saṅgha looking on,\\
Shook the palace of Migāra’s mother\\
With their big toe:

\item \vleftofline{‘}Dark One, if you attack\\
A monk who knows this from their own experience,\\
A disciple of the Buddha,\\
You will fall into suffering.

\item \vleftofline{‘}The one who shook Vejayanta palace\\
With their big toe,\\
Relying on psychic power,\\
Inspiring awe among the gods:

\item \vleftofline{‘}Dark One, if you attack\\
A monk who knows this from their own experience,\\
A disciple of the Buddha,\\
You will fall into suffering.

\item \vleftofline{‘}The one who asked Sakka in Vejayanta palace:\\
\vleftofline{“}Friend, do you know the freedom\\
That comes with the end of craving?”\\
And to whom, when asked this question,\\
Sakka answered truthfully:

\item \vleftofline{‘}Dark One, if you attack\\
A monk who knows this from their own experience,\\
A disciple of the Buddha,\\
You will fall into suffering.

\item \vleftofline{‘}The one who asked Brahmā\\
In the Sudhamma Hall before the assembly:\\
\vleftofline{“}Friend, do you still have the same view\\
That you had in the past?\\
Or do you see the radiance\\
Of the Brahmā world passing away?”

\item \vleftofline{‘}And to whom, when asked this question,\\
Brahmā answered truthfully:\\
\vleftofline{“}Friend, I don’t have that view\\
That I had in the past.

\item \vleftofline{‘“}I see the radiance\\
Of the Brahmā world passing away.\\
So how could I say today\\
That I am permanent and eternal?”

\item \vleftofline{‘}Dark One, if you attack\\
A monk who directly knows this,\\
A disciple of the Buddha,\\
You will fall into suffering.

\item \vleftofline{‘}The one who through emancipation has touched\\
The peak of the mighty Mount Neru,\\
The forests of Pubbavideha,\\
And the people who live there:

\item \vleftofline{‘}Dark One, if you attack\\
A monk who directly knows this,\\
A disciple of the Buddha,\\
You will fall into suffering.

\item \vleftofline{‘}Though a fire doesn’t think\\
\vleftofline{“}I’ll burn the fool”\\
Still the fool who comes too close\\
To the fire gets burnt.

\item \vleftofline{‘}In the same way Māra,\\
Having attacked the Tathāgata,\\
You’ll burn yourself,\\
Like a fool touching the flames.

\item \vleftofline{‘}Having attacked the Tathāgatha,\\
Māra produced demerit.\\
Wicked one, do you imagine:\\
\vleftofline{“}My wickedness won’t bear fruit?”

\item \vleftofline{‘}For a long time you’ve piled up\\
The wickedness that you’ve created.\\
Keep away from the Buddha, Māra!\\
Give up hope in tricking the monks.’

\item That is how, in the Bhesekaḷā grove\\
The monk rebuked Māra.\\
That spirit, downcast,\\
Disappeared right there!”

\chapter*{The Great Chapter}
\addcontentsline{toc}{chapter}{The Great Chapter}
\markboth{The Great Chapter}{The Great Chapter}


\subsubsection*{21.1 Vaṅgīsa}

\item \vleftofline{“}Now that I’ve gone forth\\
From the home life into homelessness,\\
I’m assailed\\
By the reckless thoughts of the Dark One.

\item Even if a thousand mighty princes and great archers\\
Well trained, with strong bows,\\
Might completely surround me,\\
I would not flee.

\item And if women come,\\
Many more than that,\\
They won’t scare me:\\
I stand firm in Dhamma.

\item Only once did I personally hear\\
From the Buddha, Kinsman of the Sun,\\
About the path leading to \emph{nibbāna};\\
My mind was delighted with that teaching.

\item Wicked one, if you come near me\\
As I live like this,\\
I’ll act in such a way that you, Death,\\
Will not even see the path I travel.

\item Entirely abandoning likes and dislikes,\\
Along with thoughts attached to the household life,\\
He wouldn’t get entangled in anything,\\
He is a monk without entanglements.

\item On this earth and in the sky,\\
Whatever form you take when entering the world\\
Wears out, it is all impermanent;\\
Reflective people live understanding this.

\item People are bound in their attachments\\
To what is seen, heard, and thought.\\
Being imperturbable, expel desire for these things;\\
For one they call a sage does not cling to these things.

\item Attached to sixty kinds of wrong views \\
With their modes of thought,\\
Unenlightened people are fixed in wrong principles;\\
But that monk wouldn’t go to any sectarian group,\\
Still less would he take up corrupt ways.

\item Clever, and for a long time established in \emph{samādhi},\\
Free of deceit, disciplined, without envy,\\
The sage has realised the state of peace,\\
Since he has realized \emph{nibbāna}, he awaits his time.

\item Abandon conceit, Gotama!\\
Completely abandon the path to conceit;\\
Infatuated with the path to conceit,\\
You’ve had regrets for a long time.

\item Smeared by smears and slain by conceit,\\
People fall into hell.\\
When people slain by conceit are reborn in hell,\\
They grieve for a long time.

\item But a monk never grieves\\
If they practice rightly, a victor of the path.\\
They have renown and happiness,\\
And they rightly call him a ‘Seer of Dhamma’.

\item So don’t be hard-hearted, be energetic,\\
With hindrances abandoned, purified,\\
And with conceit abandoned completely,\\
Be at peace, and use knowledge to make an end.”

\item \vleftofline{“}I’ve got a burning desire for pleasure;\\
My mind is on fire!\\
Please, out of compassion, Gotama,\\
Tell me how to quench the flames.”

\item \vleftofline{“}Your mind is on fire\\
Because of a perversion of perception.\\
Avoid noticing the attractive aspect of things\\
That provokes lust.

\item Meditate on the unattractive,\\
Unified, in \emph{samādhi};\\
With mindfulness immersed in the body,\\
Make much of disenchantment.

\item Meditate on the signless,\\
Throw out the underlying tendency to conceit,\\
And when you have a breakthrough \\
In understanding conceit,\\
You will live at peace.”

\item \vleftofline{“}Speak only such words\\
As do not hurt yourself\\
Nor harm others.\\
Such speech is truly well spoken.

\item Speak only pleasing words,\\
Words received gladly;\\
Pleasing words are those\\
That don’t have bad effects on others.

\item Truth itself is the undying word:\\
This is an eternal principle.\\
Realists say that the Dhamma and its meaning\\
Are grounded in the truth.

\item The reliable words spoken by the Buddha\\
For realizing \emph{nibbāna},\\
And making an end of suffering:\\
This really is the best kind of speech.”

\item \vleftofline{“}His understanding is profound, he is wise,\\
He is skilled in knowing the path \\
And what is not the path;\\
Sāriputta, of great understanding,\\
Teaches Dhamma to the monks.

\item He teaches in brief,\\
Or he speaks at length,\\
His voice, which sounds like a myna bird,\\
Inspires intuition.

\item While he teaches,\\
The monks hear his sweet voice,\\
Sounding attractive,\\
Clear and mellifluous;\\
They listen joyfully\\
With hearts uplifted.”

\item \vleftofline{“}Today, on the fifteenth day \emph{uposatha},\\
500 monks have gathered together \\
To purify their precepts.\\
These sages without affliction \\
Have cut off their fetters and bonds,\\
They will not be reborn again \\
Into any state of existence.

\item Just as a wheel-rolling emperor\\
Surrounded by ministers\\
Travels all around this\\
Land that is circled by sea;

\item So disciples with the three knowledges,\\
Destroyers of death,\\
Attend upon the winner of the battle,\\
The unsurpassed caravan leader.

\item All are sons of the Blessed One—\\
There is no rubbish here.\\
I bow to the Kinsman of the Sun,\\
The destroyer of the dart of craving.

\item Over a thousand monks\\
Attend on the Fortunate One\\
As he teaches the immaculate Dhamma:\\
\emph{Nibbāna}, free of fear from any direction.

\item They hear the stainless Dhamma\\
Taught by the Buddha.\\
The Buddha is so brilliant,\\
Revered by the monastic Saṅgha.

\item Blessed One, you are called ‘elephant’,\\
Supreme among all sages.\\
You are like a great cloud\\
That rains on your disciples.

\item Setting out from his daytime dwelling\\
Wanting to see the teacher;\\
Great hero, your disciple,\\
Vaṅgisa bows at your feet.”

\item \vleftofline{“}Overcoming Māra’s devious path,\\
I wander with hard-heartedness dissolved.\\
See him, the liberator from bonds,\\
Unattached, \\
Teaching the Dhamma by analysing each section.

\item He has explained in many ways\\
The path to cross the flood.\\
Since the deathless has been explained,\\
The seers of Dhamma stand unshakable.

\item Like a piercing light,\\
He’s seen the transcendence of all states of rebirth;\\
Knowing it and witnessing it,\\
He taught it first to the group of five.

\item When Dhamma is well taught like this,\\
How could those who understand Dhamma \\
Be heedless?\\
Therefore you should train in the teaching \\
Of the Blessed One,\\
Heedful, and always reverent.”

\item \vleftofline{“}The senior monk who was awakened \\
After the Buddha\\
Koṇḍañña is keenly energetic,\\
And regularly gains the meditative states\\
Of happiness and seclusion.

\item Whatever can be realised\\
By a disciple following the teacher,\\
He has attained it all,\\
Diligent in training himself.

\item With great power and the three knowledges,\\
Skilled in reading the minds of others,\\
Koṇḍañña, the heir to the Buddha,\\
Bows at the teacher’s feet.”

\item \vleftofline{“}As the sage, who has gone beyond suffering,\\
Sits on the mountainside,\\
He is attended by disciples \\
With the three knowledges,\\
Destroyers of death.

\item Moggallāna, of great psychic power,\\
Searches with his mind,\\
Looking into their minds\\
For one liberated without attachments.

\item So they attend upon Gotama,\\
The sage gone beyond suffering,\\
Who is endowed with all attributes,\\
And with a multitude of qualities.”

\item \vleftofline{“}Just as, when the clouds have vanished,\\
The moon shines in the sky, stainless as the sun,\\
So Aṅgīrasa, great sage,\\
Your renown outshines the entire world.”

\item \vleftofline{“}We used to wander, drunk on poetry,\\
From village to village, from town to town;\\
Then we saw the Buddha,\\
Who has gone beyond all Dhammas.

\item He, the sage gone beyond suffering,\\
Taught me the Dhamma;\\
When we heard the Dhamma, we became confident—\\
Faith arose in us.

\item Hearing him speak of\\
The aggregates, the sense-bases,\\
And the elements, I understood.\\
I went forth into homelessness.

\item Truly, Tathāgatas arise\\
For the benefit of the many\\
Men and women\\
Who follow their teachings.

\item Truly, it is for their benefit\\
That the sage indeed realised awakening;\\
The monks and nuns, who see\\
The natural principles of the Dhamma.

\item The seer, the Buddha,\\
The Kinsman of the Sun,\\
Has well taught the four noble truths\\
Out of compassion for living beings.

\item Suffering, the origin of suffering,\\
The transcending of suffering ,\\
And the noble eight-fold path\\
That leads to the stilling of suffering.

\item As these things were spoken,\\
So I have seen them.\\
I’ve realized my own true goal,\\
The Buddha’s instruction is completed.

\item It was so welcome for me,\\
As I was in the presence of the Buddha.\\
Of things which are shared,\\
I encountered the best.

\item I’ve realised the perfection of direct knowledge;\\
I have supernormal hearing;\\
I have the three knowledges and psychic powers,\\
I’m skilled at reading the minds of others.”

\item \vleftofline{“}I ask the teacher unrivalled in understanding,\\
Who has cut off all doubts in this very life—\\
Has a monk died at Aggāḷava, who was\\
Well-known, famous, and attained to \emph{nibbāna}?

\item Nigrodhakappa was his name;\\
It was given to that brahman by you, Blessed One.\\
Yearning for freedom, energetic, \\
Firmly seeing the Dhamma,\\
He wandered in your honor.

\item O Sakyan, who sees all around,\\
All of us wish to know about that disciple.\\
Our ears are eager to hear,\\
For you’re truly the most excellent teacher.

\item Cut off our doubt, declare this to us;\\
Your understanding is vast, tell us of his \emph{nibbāna}!\\
You see all around, so speak among us,\\
Like the thousand-eyed Sakka \\
In the assembly of the gods!

\item Whatever ties there are, or paths to delusion,\\
Or things that are on the side of unknowing,\\
Or that are bases of doubt:\\
When it comes to the Tathāgata there are none;\\
Among people, his eye is the best.

\item For if no man were ever to disperse defilements,\\
Like the wind dispersing a mass of clouds,\\
Darkness would cover the whole world,\\
And even a lamp would not shine.

\item The wise are makers of light;\\
My hero, that is what I think of you.\\
We’ve come to you for your insight and knowledge:\\
Here in this assembly, declare to us about Kappāyana.

\item Swiftly send forth your sweet voice,\\
Like a goose stretching its neck, gently honking,\\
The sound is smooth, with a lovely tone:\\
Alert, we are all listening to you.

\item You have entirely abandoned birth and death;\\
Restrained and pure, speak the Dhamma!\\
Unenlightened people can’t fulfil all their wishes,\\
But Tathāgatas can achieve what they intend.

\item Your answer is definitive, and we will accept it,\\
For you have perfect understanding.\\
We raise our hands in \emph{añjalī} one last time,\\
Your understanding is unrivalled,\\
So do not knowingly confuse us.

\item Knowing the noble Dhamma from top to bottom,\\
Your energy is unrivalled, \\
So do not knowingly confuse us.\\
Like a man in the baking summer sun \\
Would long for water,\\
I long for the rain of your voice to fall on my ears.

\item Surely Kappāyana\\
Did not live the spiritual life in vain?\\
Did he realise \emph{nibbāna},\\
Or did he still have a remnant of defilement?\\
Let us hear what kind of liberation he had!”

\item \vleftofline{“}He cut off craving for mind and body \\
In this very life,\\
The river of darkness that had long lain within him.\\
He has entirely crossed over birth and death.”\\
So declared the Blessed One, the leader of the five.

\item \vleftofline{“}Now that I have heard your words,\\
Best of sages, I am confident.\\
My question, it seems, was not in vain,\\
The brahman did not deceive me.

\item As he spoke, so he acted;\\
He was a disciple of the Buddha.\\
He cut the net of death the illusionist,\\
So extended and strong.

\item Blessed One, Kappāyana saw\\
The starting point of grasping.\\
He has gone beyond the realm of death,\\
So very hard to cross.

\item God of gods, best of men, I bow to you;\\
And to your son,\\
Who follows your example, a great hero\\
An elephant, true son of an elephant.”

\end{enumerate}

\end{document}
